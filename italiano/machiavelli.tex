\documentclass[10pt,a4paper]{report}
\usepackage[utf8]{inputenc}
\usepackage[T1]{fontenc}
\usepackage{amsmath}
\usepackage{amsfonts}
\usepackage{amssymb}
\usepackage{makeidx}
\usepackage{graphicx}
\usepackage[left=1.00in, right=1.00in, top=1.00in, bottom=1.00in]{geometry}
\author{Tommaso Severini}
\title{Letteratura italiana - Niccolò Machiavelli}
\begin{document}
	\maketitle
	
	\chapter*{Niccolò Machiavelli}
	
	\section*{Contesto storico e sociale}
	
	La figura di Machiavelli si inquadra in un periodo storico noto come Rinascimento. Quest'ultimo è caratterizzato da una rinascita ed un rinnovato interesse per il mondo classico, da cui si prende ispirazione sia per gli ideali che per le opere stesso (da ricordare la Traslatio Studii, trasferimento del sapere classico cominciato il lavoro di Petrarca).
	Tutto ciò portò ad una fioritura del sapere dal punto di vista artistico e scientifico.
	
	\begin{itemize}
		\item Arte: si sviluppano diversi movimenti artistici nel campo della scultura, architettura, pittura e letteratura (in particolare \textbf{politica e storica}).
		\item Scienza: si perde il \textbf{principio di autorità in favore del metodo scientifico}. Ciò porta a particolari rivoluzioni soprattutto nell'ambito della medicina e dell'astronomia.
	\end{itemize}

	Ultimo grande aspetto da considerare è legato sempre alla crescita esponenziale del sapere e, in particoalre, riguardo alla sua diffusione. La grande richiesta di cultura portò allo sviluppo di tecniche sempre più veloci ed efficienti per esportare la cultura. La più celebre di queste innovazioni fu la \textbf{stampa a caratteri mobili di Gutenberg}.\\
	
	Per quanto riguarda la situazione geopolitica, essa è caratterizzata da periodi di relativa stabilità ma anche di cambiamento. Basti pensare alla caduta definitiva dell'impero bizantino nel 1453 o la pace di Lodi del 1454 che posero fine ad una serie di conflitti che hanno segnato la storia medievale.
	
	Nonostante ciò, la Firenze in cui visse Machiavelli tende a rappresentare un corso degli eventi molto diverso da quello che si può osservare in altre regioni d'Europa. La celeberrima Repubblica fu soggetta al potere della famiglia Medici, che però fu minato diverse volte dal tentativo fallito della teocrazia di Girolamo Savonarola e da diverse congiure contro i membri della famiglia fiorentina.
	
	\section*{Vita}
	
	Nato a Firenze il 3 maggio 1469, Niccolò Machiavelli è considerato uno dei più grandi esponenti del Rinascimento italiano, dando alla luce la \textbf{scienza politica moderna} con una serie di trattati. Tra di essi risiede anche l'\textit{opus magna} del Machiavelli: \textbf{Il principe}.
	
	Grazie alla sua istruzione umanistica e alla sua iniziale amicizia con la famiglia Medici, riuscì sempre ad aver un ruolo nella vita politica della Signoria, fino all'arrivo del tristemente famoso Girolamo Savonarola. Il frate riuscì, attraverso una congiura, ad esiliare la famiglia Medici da Firenze e ad ottenere il potere; il suo sogno era quello di fondare una teocrazia basati sui princìpi cristiani. L'utopia del Savonarola sfumò dopo la sua uccisione al rogo e la successiva re-istituzione della repubblica fiorentina. Poichè i Medici si trovavano ancora lontano da Firenze, il controllo della signoria fu assunto da Pier Soderini, che permise a Machiavelli di continuare a lavorare come funzionario e, in particolare, come diplomatico all'estero. Machiavelli fu ospite delle principali monarchie europee; queste esperienza gli permisero di apprendere gli aspetti più importante che un buon governante deve avere, fornendogli grande ispirazione per i suoi successivi trattati.
	
	Tutto ciò terminò con il ritorno dei Medici a Firenze. La potente famiglia riuscì ad ottenere di nuovo il controllo della signoria, ma, vedendo che il Machiavelli aveva deciso di assistere la città sotto il controllo di una persona esterna alla famiglia, si sentì tradita. Niccolò fu trattato come una spia, subendo diverse punizioni corporali e torture che culminarono con l'esilio del celeberrimo letterato.
	
	Fu proprio durante la sua condizione di esilio che il Machiavelli si trovò a scrivere le opere per cui tutt'oggi è ricordato: il Principe e la Mandragola. La prima è un trattato politico in cui l'autore descrive tutte le caratteristiche e strategie che un buon principe deve assumere e far proprie per il mantenimento del potere. La seconda invece è un'opera teatrale che, attraverso l'utilizzo di diversi personaggi, deve essere interpretata a seconda degli ideali politici  che lo stesso Machiavelli voleva trasmettere attraverso il Principe. 
	
	\section*{Il pensiero}
	
	I princìpi cardine che segnano il pensiero machiavelliano possono essere riassunti in 4 punti:
	
	\begin{itemize}
		\item Il fatto che l'agire politico non dipenda dai valori morali della società(creando spesso un'immagine spietata del principe). Ciò perchè il principe non agiscd per fini personale, ma guidato dal bene collettivo.
		\item Il pragmatismo è un aspetto fondamentale della condizione umana. L'uomo deve essere sempre in cerca della verità oggettiva, non di una rappresentazione (spesso distorta) di essa.
		\item La malvagità è intrinseca nell'uomo, fornendogli (tendenzialmente) un'attitudine maligna; è compito dello Stato controllare queste pulsioni da parte del popolo, in una continua ricerca per la convivenza civile.
		\item La fortuna, insieme alla virtù, è l'elemento principale che permette all'uomo di usufruire di diverse opportunità nel corso della sua vita. Come esempio, il Machiavelli spesso riporta quello del conte Valentino, figlio di Papa Alessandro VI Borgia, che grazie alle opportunità derivate dal potere del padre riuscì ad ottenere molto potere.
	\end{itemize}

\chapter*{Interpretazioni letterarie}

	\section*{Introduzione}
	
	Nel corso della storia, molti critici si sono cimentati nell'interpretazione del principe, fornendo vari punti di vista che mettono in luce nuove prospettivi e possibili motivazioni dell'autore. Una delle interpretazioni meno condivise dai critici dell '800 e del '900 vede ls pubblicazione del principe non come un suggerimento a dei futuri tiranni, ma quanto un avvertimento per la popolazione (eseguito nei minimi dettagli). Comprendere e assimilare le dinamiche utilizzate da un despota spietato è utile per fare in modo che ciò non possa accadere nella realtà.\\
	
	Oltre a questa interpretazione, è anche possibile pensare che Machiavelli non abbia scritto il Principe per permettere ai tiranni di ottenere il potere, ma per esortare la popolazione (e in particolare la famiglia Medici) a lottare per riunire l'Italia sotto un unico regno, privo di influenze straniere. Se ciò fosse il caso, questo trattato rappresenterebbe uno dei primi esempi di nazionalismo italiano, a prima vista anticipando di quasi 3 secoli il lavoro di Garibaldi e degli altri principali esponenti del nazionalismo. 
	
	Nonostante questa parvenza (sostenuta anche dal celebre \textbf{Antonio Gramsci}), lo stato che il Machiavelli immagina, non è lo stato nazionale che in futuro sarà conosciuto come Italia, bensì una repubblica fiorentina in grado di ottenere il controllo della penisola. Essa, infatti, fino a quel momento risultava frammentata in diverse signorie, ducati e regni indipendenti e spesso in conflitto tra di loro.
	
	\section*{La questione dell'"attualità"}
	
	In un'intervista, il filologo Francesco Baudi riporta diverse problematiche emerse nel corso della storia nel processo di attualizzazione del trattato ma machiavelliano. Cio che emerge è come segue:
	
	La problematica principale che emerge dall'attualizzazione del Principe riguardo il fatto che chi compie questa opera spesso agisce per confermari i suoi ideali politici. Ciò produce un'interpretazione spesso non oggettiva e spesso traviata dal cosiddetto \textit{confirmation bias}, che potenzialmente possono condurre due lettori dai punti di vista completamente opposti a giungere a conclusione discordanti analizzando lo stesso trattato (Gramsci e Mussolini). Secondo il Baudi, attualizzare l'opera può essere utile unicamente a fornirci un modo per comprendere il contesto storico e politico in cui Machiavelli ha vissuto.
	
	Ciò, secondo Baudi, non significa che il Principe non può essere "utile" nella scena politica contemporanea, ma che deve essere l'intelligenza laica, pessimista e spesso amara che devono farci dubitare di quelle ideologie che intendono cambiare il mondo, spesso con la violenza.
	
	\section*{Il "manifesto politico" di Gramsci}
	
	Il Gramsci, famoso fondatore del PCI e detenuto da Mussolini durante il regime fascista, si ritrova ad offrire un'interessante interpretazione all'opera di Machiavelli. Egli sostiene che le argomentazioni, rigorose e scientifiche, che Machiavelli offre nel suo trattato rappresentino un grido disperato del popolo ai governanti. Essi, infatti, si concludono e concretizzano con l'ìappello che il Machiavelli pone da parte del popolo alla famiglia Medici, esortandoli a formare il nuovo Stato.
	
	\section*{L'incertezza di Ferroni}
	
	Il Ferroni, nella sua analisi, spiega come l'interpretazione del Principe che vede l'opera come una guida che porta lo stato ad un'utopia attraverso l'amoralità e la violenza (come nel caso del Fascismo e del Comunismo del XX secolo) non tenga conto del contesto politico e sociale durante il quale l'opera fu redatta. 
	
	Essenzialmente, il Ferroni ci vuole far capire come il clima instabile e ricco di conflitti offuschi la mente dell'autore, fornendogli una percezione sbagliata del potere e del suo mantenimento, ma anche un punto di vista spesso contraddittorio e a prima vista utopistico.   
	
\end{document}