\documentclass[10pt,a4paper]{article}
\usepackage[utf8]{inputenc}
\usepackage[T1]{fontenc}
\usepackage{amsmath}
\usepackage{amsfonts}
\usepackage{amssymb}
\usepackage{makeidx}
\usepackage{graphicx}

\usepackage{hyperref}

\usepackage{xcolor}     % for colour
\usepackage{mdframed}   % for framing

\usepackage[left=1.00in, right=1.00in, top=1.00in, bottom=1.00in]{geometry}
\author{Tommaso Severini}
\title{Letteratura italiana - Petrarca}

\parindent 0ex
\begin{document}
	\maketitle
	
	\tableofcontents
	
	\newmdtheoremenv[%
	linecolor=gray,leftmargin=60,%
	rightmargin=40,
	backgroundcolor=gray!40,%
	innertopmargin=5pt,%
	font=\ttfamily]{estratto}{Extract}[section]
	
	\section{Voi ch'ascoltate in rime sparse il suono}
	\label{sec:Voi ch'ascoltate in rime sparse il suono}
	
	\begin{estratto}
		Voi ch’ascoltate in rime sparse il suono\\
		di quei sospiri ond’io nudriva ’l core\\
		in sul mio primo giovenile errore\\
		quand’era in parte altr’uom da quel ch’i’ sono,\\
		
		del vario stile in ch’io piango et ragiono\\
		fra le vane speranze e ’l van dolore,\\
		ove sia chi per prova intenda amore,\\
		spero trovar pietà, nonché perdono.\\
		
		Ma ben veggio or sì come al popol tutto\\
		favola fui gran tempo, onde sovente\\
		di me medesmo meco mi vergogno;\\
		
		et del mio vaneggiar vergogna è ’l frutto,\\
		e ’l pentersi, e ’l conoscer chiaramente\\
		che quanto piace al mondo è breve sogno.\\
	\end{estratto}

Petrarca si rivolge ai lettori in grado di comprendere per esperienza le pene amorose, con un topos che rimanda a molte \textbf{liriche dello Stilnovo} (Donne ch'avete intelletto d'amore), anche se l'effusione del sentimento da parte sua sarà spesso un soliloquio: il poeta chiede perdono per i lamenti da lui prodotti nell'illusione di un \textbf{amore infelice}, che ora (a distanza di anni e col bagaglio della raggiunta maturità) \textbf{egli giudica un "giovenile errore" e un "vaneggiare"}, poiché tutto quello che piace al mondo è un sogno destinato a finire presto. Petrarca è anche consapevole che l'aver perso tempo dietro a Laura lo ha distolto dalla sua "missione" di intellettuale impegnato e lo ha esposto alla derisione del volgo, tema tipicamente classico che \textbf{rimanda al carme 8 di Catullo} ("Miser Catulle, desinas ineptire / et quod vides perisse perditum ducas", "Povero Catullo, smetti di fare il pazzo e ritieni perduto ciò che hai visto che è andato perduto"). La paura di essere deriso dal volgo tornerà anche nel sonetto \nameref{sec:Solo et pensoso} (35), in cui l'autore rifugge la compagnia degli altri uomini per non svelare la sua condizione interiore.

\section{Benedetto sia 'l giorno, e 'l mese, et l'anno}

\begin{estratto}
	Benedetto sia 'l giorno, e 'l mese, e l'anno,\\
	e la stagione, e 'l tempo, e l'ora, e 'l punto,\\
	e 'l bel paese, e 'l loco ov'io fui giunto\\
	da' duo begli occhi che legato m'hanno;\\
	
	e benedetto il primo dolce affanno\\
	ch'i'ebbi ad esser con Amor congiunto,\\
	e l'arco, e le saette ond'i' fui punto,\\
	e le piaghe che 'nfin al cor mi vanno.\\
	
	Benedette le voci tante ch'io\\
	chiamando il nome de mia donna ho sparte,\\
	e i sospiri, e le lagrime, e 'l desio;\\
	
	e benedette sian tutte le carte\\
	ov'io fama l'acquisto, e 'l pensier mio,\\
	ch'è sol di lei, sì ch'altra non v'ha parte.\\
\end{estratto}

In questo sonetto Petrarca racconta il luogo e il modo in cui vide per la prima volta Laura. Non è una dichiarazione d’amore: \textbf{il poeta esprime infatti il suo dissidio interiore}. Racconta il momento in cui fu colpito dalle frecce dell’Amore. Benedice tutti i sentimenti che lo legano alla donna concludendo che Laura è l’unica donna che può averne parte.\\

Nel sonetto “Benedetto sia ‘l giorno e ’l mese e l’anno” si capisce che Petrarca associa l’incontro avvenuto con Laura ad un fatto non casuale, un evento guidato da una volontà ineluttabile, associandolo ad un miracolo;il sonetto viene rafforzato dall’enumerazione(elenco:e,e,e,….)e dalle ripetute anafore all’inizio delle quartine e terzine,le quali riprendono il \textbf{termine “benedetto”}. Viene ripreso lo stile stilnovistico,in quanto \textbf{ogni riferimento a Laura viene commemorato e idealizzato};ritorna l’ immagine della donna,creata dal sogno,dalla fantasia e \textbf{dalla memoria}. Il poeta benedice anche i momenti negativi della propria esistenza, sottolineando il concetto di donna,come essere superiore rispetto all’uomo,la quale \textbf{dona gentilezza d’animo}. Egli non mostra rimpianti nel descrivere le sue sofferenze,anzi sembra andarne fiero,forse perché dimostrare la propria sensibilità verso Laura rendeva il suo corteggiamento più limpido,accentuando l’idea di un amore inappagato. Chiede al Signore di aiutarlo a ritornare ad un’esistenza più ligia e di sconfiggere il diavolo che tramite la sua amata lo provoca a commettere peccati. \textbf{Il problema del rimorso e del pentimento sboccia in un’analisi in cui viene descritto il passato,come tempo della debolezza e dell’errore e il futuro come attesa della liberazione e del riscatto}. Il sonetto viene strutturato come una preghiera a Dio, in cui l’invocazione ad Egli viene seguita dal ricordo del tempo perduto nel vaneggiamento e nella colpa;l’opera si apre e si chiude facendo riferimento a due preghiere importanti(Padre nostro e il Miserere) per rafforzare l’idea spirituale.\\

\subsection{Sintesi}

Ed è proprio grazie a questi due testi che riusciamo a scorgere il conflitto interiore del Petrarca,il quale si trova continuamente combattuto fra il richiamo dei beni terreni e il bisogno a condurre una vita più pura indirizzata alla salvezza interiore. Una battaglia attraverso la quale il poeta non riuscirà a trovare pace,vedendo anche nella morte solamente tempesta,al contrario di Dante che, scrivendo la Commedia, riuscirà a giungere ad una purificazione.

\section{Solo et pensoso}
\label{sec:Solo et pensoso}

\begin{estratto}
	Solo et pensoso i più deserti campi\\
	vo mesurando a passi tardi et lenti,\\
	et gli occhi porto per fuggire intenti\\
	ove vestigio human l’arena stampi.\\
	
	Altro schermo non trovo che mi scampi\\
	dal manifesto accorger de le genti,\\
	perché negli atti d’alegrezza spenti\\
	di fuor si legge com’io dentro avampi:\\
	
	sì ch’io mi credo omai che monti et piagge\\
	et fiumi et selve sappian di che tempre\\
	sia la mia vita, ch’è celata altrui.\\
	
	Ma pur sì aspre vie né sì selvagge\\
	cercar non so ch’Amor non venga sempre\\
	ragionando con meco, et io co·llui.\\
\end{estratto}

L'autore presenta se stesso come afflitto dalle pene amorose e vergognoso della propria condizione che non vuole svelare agli altri, ragione che lo spinge a sfuggire la compagnia degli altri uomini e a passeggiare in luoghi remoti e solitari, nel tentativo (non riuscito) di non pensare continuamente a Laura: \textbf{vi è la ripresa di un tema già presente nel sonetto proemiale, dove Petrarca diceva di essere stato per lungo tempo la "favola" del popolo e in cui qualificava il suo amore non corrisposto come un "vaneggiare"}, il cui frutto era stato appunto la "vergogna". L'oggettivazione dell'interiorità con il dato esteriore del paesaggio desolato è una delle novità più interessanti della lirica petrarchesca e segna la massima distanza dalla tradizione precedente, in cui il \textbf{locus amoenus} fungeva soltanto da sfondo all'avventura amorosa e non di rado il dato spaziale era del tutto assente o appena accennato (tipica in questo senso l'essenzialità della Vita nuova, dove la città di Firenze non viene mai nominata).

\section{O cameretta, che già fosti un porto}

\begin{estratto}
	O cameretta che già fosti un porto\\
	a le gravi tempeste mie diurne,\\
	fonte se’ or di lagrime nocturne,\\
	che ’l dí celate per vergogna porto.\\
	
	O letticciuol che requie eri et conforto\\
	in tanti affanni, di che dogliose urne\\
	ti bagna Amor, con quelle mani eburne,\\
	solo ver ’me crudeli a sí gran torto!\\
	
	Né pur il mio secreto e ’l mio riposo\\
	fuggo, ma più me stesso e ’l mio pensero,\\
	che, seguendol, talor levommi a volo;\\
	
	e ’l vulgo a me nemico et odioso\\
	(chi ’l pensò mai?) per mio refugio chero:\\
	tal paura ò di ritrovarmi solo.\\
\end{estratto}

Nelle quartine l'autore si rivolge alla sua "cameretta" e al "letticciuol" attraverso il parallelismo "O... che", posto all'inizio delle due strofe, per dire che un tempo il raccoglimento interiore nelle \textbf{ore notturne era fonte per lui di pace e serenità rispetto alle "tempeste" del giorno}, mentre ora la solitudine gli causa ulteriore sofferenza per via dell'amore infelice per Laura, le cui mani "eburne" (d'avorio) e crudeli aiutano Amore a versare lacrime dalle "urne" (gli occhi di Petrarca). Nella prima quartina il poeta riprende la metafora della \textbf{vita come viaggio in un mare in tempesta, rispetto alla quale la sua camera rappresentava un porto sicuro} (immagine che ricorre anche in altri testi, specie nelle rime "in morte" di Laura come il sonetto 272; TESTO: La vita fugge, et non s'arresta una hora), mentre nella seconda l'antitesi ruota attorno ai termini "requie" e "conforto" che si oppongono agli "affanni", anche attraverso la crudeltà di Laura che respinge il corteggiamento di Petrarca (non mancano altre interpretazioni). Il v. 4 accenna alla "vergogna" provata dall'autore nel mostrare le proprie lacrime che perciò cerca di nascondere, tema che era già espresso nel sonetto proemiale in cui ammetteva di essere stato a lungo la "favola" del popolo (TESTO: \nameref{sec:Voi ch'ascoltate in rime sparse il suono}).

\end{document}