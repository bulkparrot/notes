\documentclass[10pt,a4paper]{article}
\usepackage[utf8]{inputenc}
\usepackage[T1]{fontenc}
\usepackage{amsmath}
\usepackage{amsfonts}
\usepackage{amssymb}
\usepackage{makeidx}
\usepackage{graphicx}

\usepackage{hyperref}

\usepackage{xcolor}     % for colour
\usepackage{mdframed}   % for framing

\usepackage[left=1.00in, right=1.00in, top=1.00in, bottom=1.00in]{geometry}
\author{Tommaso Severini}
\title{Letteratura italiana - Petrarca}

\parindent 0ex
\begin{document}
	\maketitle
	
	\tableofcontents
	
	\newmdtheoremenv[%
	linecolor=gray,leftmargin=60,%
	rightmargin=40,
	backgroundcolor=gray!40,%
	innertopmargin=5pt,%
	font=\ttfamily]{estratto}{Extract}[section]
	
	\section{Voi ch'ascoltate in rime sparse il suono}
	\label{sec:Voi ch'ascoltate in rime sparse il suono}
	
	\begin{estratto}
		Voi ch’ascoltate in rime sparse il suono\\
		di quei sospiri ond’io nudriva ’l core\\
		in sul mio primo giovenile errore\\
		quand’era in parte altr’uom da quel ch’i’ sono,\\
		
		del vario stile in ch’io piango et ragiono\\
		fra le vane speranze e ’l van dolore,\\
		ove sia chi per prova intenda amore,\\
		spero trovar pietà, nonché perdono.\\
		
		Ma ben veggio or sì come al popol tutto\\
		favola fui gran tempo, onde sovente\\
		di me medesmo meco mi vergogno;\\
		
		et del mio vaneggiar vergogna è ’l frutto,\\
		e ’l pentersi, e ’l conoscer chiaramente\\
		che quanto piace al mondo è breve sogno.\\
	\end{estratto}

Petrarca si rivolge ai lettori in grado di comprendere per esperienza le pene amorose, con un topos che rimanda a molte \textbf{liriche dello Stilnovo} (Donne ch'avete intelletto d'amore), anche se l'effusione del sentimento da parte sua sarà spesso un soliloquio: il poeta chiede perdono per i lamenti da lui prodotti nell'illusione di un \textbf{amore infelice}, che ora (a distanza di anni e col bagaglio della raggiunta maturità) \textbf{egli giudica un "giovenile errore" e un "vaneggiare"}, poiché tutto quello che piace al mondo è un sogno destinato a finire presto. Petrarca è anche consapevole che l'aver perso tempo dietro a Laura lo ha distolto dalla sua "missione" di intellettuale impegnato e lo ha esposto alla derisione del volgo, tema tipicamente classico che \textbf{rimanda al carme 8 di Catullo} ("Miser Catulle, desinas ineptire / et quod vides perisse perditum ducas", "Povero Catullo, smetti di fare il pazzo e ritieni perduto ciò che hai visto che è andato perduto"). La paura di essere deriso dal volgo tornerà anche nel sonetto \nameref{sec:Solo et pensoso} (35), in cui l'autore rifugge la compagnia degli altri uomini per non svelare la sua condizione interiore.

\section{Benedetto sia 'l giorno, e 'l mese, et l'anno}

\begin{estratto}
	Benedetto sia 'l giorno, e 'l mese, e l'anno,\\
	e la stagione, e 'l tempo, e l'ora, e 'l punto,\\
	e 'l bel paese, e 'l loco ov'io fui giunto\\
	da' duo begli occhi che legato m'hanno;\\
	
	e benedetto il primo dolce affanno\\
	ch'i'ebbi ad esser con Amor congiunto,\\
	e l'arco, e le saette ond'i' fui punto,\\
	e le piaghe che 'nfin al cor mi vanno.\\
	
	Benedette le voci tante ch'io\\
	chiamando il nome de mia donna ho sparte,\\
	e i sospiri, e le lagrime, e 'l desio;\\
	
	e benedette sian tutte le carte\\
	ov'io fama l'acquisto, e 'l pensier mio,\\
	ch'è sol di lei, sì ch'altra non v'ha parte.\\
\end{estratto}

In questo sonetto Petrarca racconta il luogo e il modo in cui vide per la prima volta Laura. Non è una dichiarazione d’amore: \textbf{il poeta esprime infatti il suo dissidio interiore}. Racconta il momento in cui fu colpito dalle frecce dell’Amore. Benedice tutti i sentimenti che lo legano alla donna concludendo che Laura è l’unica donna che può averne parte.\\

Nel sonetto “Benedetto sia ‘l giorno e ’l mese e l’anno” si capisce che Petrarca associa l’incontro avvenuto con Laura ad un fatto non casuale, un evento guidato da una volontà ineluttabile, associandolo ad un miracolo;il sonetto viene rafforzato dall’enumerazione(elenco:e,e,e,….)e dalle ripetute anafore all’inizio delle quartine e terzine,le quali riprendono il \textbf{termine “benedetto”}. Viene ripreso lo stile stilnovistico,in quanto \textbf{ogni riferimento a Laura viene commemorato e idealizzato};ritorna l’ immagine della donna,creata dal sogno,dalla fantasia e \textbf{dalla memoria}. Il poeta benedice anche i momenti negativi della propria esistenza, sottolineando il concetto di donna,come essere superiore rispetto all’uomo,la quale \textbf{dona gentilezza d’animo}. Egli non mostra rimpianti nel descrivere le sue sofferenze,anzi sembra andarne fiero,forse perché dimostrare la propria sensibilità verso Laura rendeva il suo corteggiamento più limpido,accentuando l’idea di un amore inappagato. Chiede al Signore di aiutarlo a ritornare ad un’esistenza più ligia e di sconfiggere il diavolo che tramite la sua amata lo provoca a commettere peccati. \textbf{Il problema del rimorso e del pentimento sboccia in un’analisi in cui viene descritto il passato,come tempo della debolezza e dell’errore e il futuro come attesa della liberazione e del riscatto}. Il sonetto viene strutturato come una preghiera a Dio, in cui l’invocazione ad Egli viene seguita dal ricordo del tempo perduto nel vaneggiamento e nella colpa;l’opera si apre e si chiude facendo riferimento a due preghiere importanti(Padre nostro e il Miserere) per rafforzare l’idea spirituale.\\

\subsection{Sintesi}

Ed è proprio grazie a questi due testi che riusciamo a scorgere il conflitto interiore del Petrarca,il quale si trova continuamente combattuto fra il richiamo dei beni terreni e il bisogno a condurre una vita più pura indirizzata alla salvezza interiore. Una battaglia attraverso la quale il poeta non riuscirà a trovare pace,vedendo anche nella morte solamente tempesta,al contrario di Dante che, scrivendo la Commedia, riuscirà a giungere ad una purificazione.

\section{Padre del ciel dopo i perduti giorni}

\begin{estratto}
	Padre del ciel, dopo i perduti giorni,\\
	dopo le notti vaneggiando spese,\\
	con quel fero desio ch’al cor s’accese,\\
	mirando gli atti per mio mal sì adorni,\\
	
	 piacciati omai col tuo lume ch’io torni\\
	ad altra vita et a più belle imprese,\\
	sì ch’avendo le reti indarno tese,\\
	il mio duro adversario se ne scorni.\\
	
	Or volge, Signor mio, l’undecimo anno\\
	ch’i’ fui sommesso al dispietato giogo\\
	che sopra i più soggetti è più feroce.\\
	
	Miserere  del mio non degno affanno;\\
	reduci i pensier’ vaghi a miglior luogo;\\
	ramenta lor come oggi fusti in croce.\\
\end{estratto}

L’incipit di questo sonetto riecheggia il Padre Nostro, la più semplice e umile preghiera cristiana, attraverso la quale Petrarca si pone al cospetto di Dio, confessandogli il peccato di aver sprecato i propri giorni e le proprie notti vaneggiando l’amore profano.
Emerge dal sonetto il dissidio che travaglia l’animo di Petrarca, il contrasto drammatico tra:
\begin{itemize}
	\item La passione d’amore;
	\item Gli alti ideali religiosi.
\end{itemize}

Il poeta è combattuto tra la sua profonda passione amorosa terrena, per Laura, e l’intimo desiderio di liberarsi da essa per perseguire i grandi ideali predicati dalla religione, e ciò avviene in occasione dell’anniversario del giorno fatale in cui si innamorò di Laura, 11 anni prima, che coincide con quello in cui ricorre la morte di Cristo: il venerdì santo.
L’amore per Laura è deprecato e rinnegato, infatti viene presentato come un sentimento negativo e dannoso, una passione che distoglie l’uomo da ideali più alti, per questo Petrarca usa sempre espressioni negative come: fero desio, mio mal, duro avversario.

\section{Solo et pensoso}
\label{sec:Solo et pensoso}

\begin{estratto}
	Solo et pensoso i più deserti campi\\
	vo mesurando a passi tardi et lenti,\\
	et gli occhi porto per fuggire intenti\\
	ove vestigio human l’arena stampi.\\
	
	Altro schermo non trovo che mi scampi\\
	dal manifesto accorger de le genti,\\
	perché negli atti d’alegrezza spenti\\
	di fuor si legge com’io dentro avampi:\\
	
	sì ch’io mi credo omai che monti et piagge\\
	et fiumi et selve sappian di che tempre\\
	sia la mia vita, ch’è celata altrui.\\
	
	Ma pur sì aspre vie né sì selvagge\\
	cercar non so ch’Amor non venga sempre\\
	ragionando con meco, et io co·llui.\\
\end{estratto}

L'autore presenta se stesso come afflitto dalle pene amorose e vergognoso della propria condizione che non vuole svelare agli altri, ragione che lo spinge a sfuggire la compagnia degli altri uomini e a passeggiare in luoghi remoti e solitari, nel tentativo (non riuscito) di non pensare continuamente a Laura: \textbf{vi è la ripresa di un tema già presente nel sonetto proemiale, dove Petrarca diceva di essere stato per lungo tempo la "favola" del popolo e in cui qualificava il suo amore non corrisposto come un "vaneggiare"}, il cui frutto era stato appunto la "vergogna". L'oggettivazione dell'interiorità con il dato esteriore del paesaggio desolato è una delle novità più interessanti della lirica petrarchesca e segna la massima distanza dalla tradizione precedente, in cui il \textbf{locus amoenus} fungeva soltanto da sfondo all'avventura amorosa e non di rado il dato spaziale era del tutto assente o appena accennato (tipica in questo senso l'essenzialità della Vita nuova, dove la città di Firenze non viene mai nominata).

\section{O cameretta, che già fosti un porto}

\begin{estratto}
	O cameretta che già fosti un porto\\
	a le gravi tempeste mie diurne,\\
	fonte se’ or di lagrime nocturne,\\
	che ’l dí celate per vergogna porto.\\
	
	O letticciuol che requie eri et conforto\\
	in tanti affanni, di che dogliose urne\\
	ti bagna Amor, con quelle mani eburne,\\
	solo ver ’me crudeli a sí gran torto!\\
	
	Né pur il mio secreto e ’l mio riposo\\
	fuggo, ma più me stesso e ’l mio pensero,\\
	che, seguendol, talor levommi a volo;\\
	
	e ’l vulgo a me nemico et odioso\\
	(chi ’l pensò mai?) per mio refugio chero:\\
	tal paura ò di ritrovarmi solo.\\
\end{estratto}

Nelle quartine l'autore si rivolge alla sua "cameretta" e al "letticciuol" attraverso il parallelismo "O... che", posto all'inizio delle due strofe, per dire che un tempo il raccoglimento interiore nelle \textbf{ore notturne era fonte per lui di pace e serenità rispetto alle "tempeste" del giorno}, mentre ora la solitudine gli causa ulteriore sofferenza per via dell'amore infelice per Laura, le cui mani "eburne" (d'avorio) e crudeli aiutano Amore a versare lacrime dalle "urne" (gli occhi di Petrarca). Nella prima quartina il poeta riprende la metafora della \textbf{vita come viaggio in un mare in tempesta, rispetto alla quale la sua camera rappresentava un porto sicuro} (immagine che ricorre anche in altri testi, specie nelle rime "in morte" di Laura come il sonetto 272; TESTO: La vita fugge, et non s'arresta una hora), mentre nella seconda l'antitesi ruota attorno ai termini "requie" e "conforto" che si oppongono agli "affanni", anche attraverso la crudeltà di Laura che respinge il corteggiamento di Petrarca (non mancano altre interpretazioni). Il v. 4 accenna alla "vergogna" provata dall'autore nel mostrare le proprie lacrime che perciò cerca di nascondere, tema che era già espresso nel sonetto proemiale in cui ammetteva di essere stato a lungo la "favola" del popolo (TESTO: \nameref{sec:Voi ch'ascoltate in rime sparse il suono}). Inoltre, è possibile notare un forte distacco tra passato e presente. Rispetto al sonetto "Solo et pensoso", dove Petrerca fuggiva dagli uomini per evitare il dolore, ora egli riconosce che il dolore che prova proviene da dentro di lui, \textbf{cercando di fuggire da se stesso}.

\section{Chiare, fresche e dolci acque}

\begin{estratto}
	Chiare, fresche et dolci acque,\\
	ove le belle membra\\
	pose colei che sola a me par donna;\\
	gentil ramo ove piacque\\
	(con sospir’ mi rimembra)\\
	a lei di fare al bel fiancho colonna;\\
	herba et fior’ che la gonna\\
	leggiadra ricoverse\\
	co l’angelico seno;\\
	aere sacro, sereno,\\
	ove Amor co’ begli occhi il cor m’aperse:\\
	date udïenzia insieme\\
	a le dolenti mie parole extreme.\\
	
	Da’ be’ rami scendea\\
	(dolce ne la memoria)\\
	una pioggia di fior’ sovra ’l suo grembo;\\
	et ella si sedea\\
	humile in tanta gloria,\\
	coverta già de l’amoroso nembo.\\
	Qual fior cadea sul lembo,\\
	qual su le treccie bionde,\\
	ch’oro forbito et perle\\
	eran quel dì a vederle;\\
	qual si posava in terra, et qual su l’onde;\\
	qual con un vago errore\\
	girando parea dir: Qui regna Amore.\\
	
	Quante volte diss’io\\
	allor pien di spavento:\\
	Costei per fermo nacque in paradiso.\\
	Così carco d’oblio\\
	il divin portamento\\
	e ’l volto e le parole e ’l dolce riso\\
	m’aveano, et sì diviso\\
	da l’imagine vera,\\
	ch’i’ dicea sospirando:\\
	Qui come venn’io, o quando?;\\
	credendo esser in ciel, non là dov’era.\\
	Da indi in qua mi piace\\
	questa herba sì, ch’altrove non ò pace\\
\end{estratto}

Il ricordo di Laura sulle rive del Sorga è una descrizione idilliaca e ricca di immagini tratte dalla tradizione classica, in cui Laura sembra più una divinità pagana che non la "donna-angelo" di ispirazione stilnovista: la donna siede morbidamente sull'erba con la "gonna / leggiadra", mentre dai rami degli alberi scende una pioggia di fiori simile a un "amoroso nembo" che si posano su di lei e sugli elementi del paesaggio con un leggiadro volteggiare, con una ripresa di immagini della mitologia classica (il dio Amore, la simbologia dei petali...) che, a differenza dei poeti precedenti, sono del tutto sganciate da qualunque spiritualizzazione, fanno da sfondo a un amore terreno e dalle implicazioni sensuali inequivocabili (al v. 9 l'"angelico seno" è proprio il seno di Laura appoggiato all'erba, per cui la donna è mostrata nella sua nudità e con la bellezza seducente del suo giovane corpo).\\

La canzone si fonda tutta sulla contrapposizione tra il passato e la memoria del precedente incontro con Laura e il presente, in cui Petrarca si sente prossimo alla morte per le sofferenze amorose e desidera essere sepolto in quel luogo che ama: all'inizio si rivolge agli elementi del paesaggio (le acque del fiume, il ramo, l'erba, i fiori, l'aria) pregandoli di ascoltare il suo lamento amoroso, quindi esprime il desiderio che Laura torni lì e pianga sulla sua tomba, invocando per lui il perdono divino, nella consapevolezza che il suo amore è frutto del peccato e da condannare sul piano morale.

\subsection{Metrica}

Metro: canzone formata da cinque stanze di tredici versi ciascuna (endecasillabi e settenari), con schema della rima abCabCcdeeDfF e un congedo il cui schema riprende gli ultimi tre versi della sirma (DfF). La lingua presenta numerosi latinismi, consueti nello stile petrarchesco, tra cui "et" (vv. 1, 7, 26 ecc.), "herba" (v. 7), "extreme" (v. 14), "gratia" (v. 17), "humile" (v. 44); alcune forme sono proprie della grafia del latino medievale, come "fiancho" (v. 6), "anchor" (v. 27), "boscho" (v. 68). Il lessico non presenta termini ricercati o preziosi, conformemente a tutta la lirica petrarchesca, e lo stile è piuttosto fluido e musicale, anche grazie alla prevalenza di versi settenari (contrariamente alla tradizione della poesia lirica e al modello della canzone dantesca).

\section{Erano i capei d'oro a l'aura sparsi}

\begin{estratto}
	Erano i capei d’oro a l’aura sparsi\\
	che ’n mille dolci nodi gli avolgea,\\
	e ’l vago lume oltra misura ardea\\
	di quei begli occhi, ch’or ne son sì scarsi;\\
	
	e ’l viso di pietosi color’ farsi,\\
	non so se vero o falso, mi parea:\\
	i’ che l’esca amorosa al petto avea,\\
	qual meraviglia se di sùbito arsi?\\
	
	Non era l’andar suo cosa mortale,\\
	ma d’angelica forma; e le parole\\
	sonavan altro, che pur voce humana.\\
	
	Uno spirto celeste, un vivo sole\\
	fu quel ch’i' vidi: e se non fosse or tale,\\
	piagha per allentar d’arco non sana.\\
\end{estratto}

Tutto il componimento è giocato sul contrasto tra la Laura del primo incontro, quand'era giovane e bellissima, e quella del presente, invecchiata e la cui bellezza esteriore è sfiorita: la prima è descritta coi tratti distintivi della donna-angelo dello Stilnovo, quindi dai capelli biondi ("capei d'oro"), con gli occhi pieni di un "vago lume", dotata di un incedere che la fa sembrare una "angelica forma" e di una voce superiore a quella umana, paragonata a uno spirito celeste e a un "vivo sole"; della seconda è detto solo che i suoi occhi sono "scarsi" della luminosità di un tempo, intendendo che la donna è invecchiata e reca sul volto i segni del tempo, cosa che tuttavia non fa diminuire l'amore di Petrarca per lei. L'invecchiamento di Laura è l'aspetto che più la allontana dallo stereotipo della donna-angelo stilnovista richiamato solo dalla descrizione esteriore, dal momento che essa è una donna umana priva di qualunque significato religioso e per cui il poeta prova un amore passionale, centrato soprattutto sulla sua bellezza fisica; il tema si ricollega a un brano del Secretum, in cui S. Agostino accusava Francesco di amare l'aspetto esteriore di Laura e lui ribatteva dicendo che anche adesso che lei è invecchiata i suoi sentimenti restano immutati.\\

Il primo incontro tra Petrarca e Laura è raccontato nel sonetto 3, Era il giorno ch'al sol si scoloraro, in cui il poeta sottolinea come l'innamoramento sia avvenuto il giorno dell'anniversario della morte di Cristo (6 apr. 1327) quando non gli sembrava necessario difendersi dai colpi di Amore: anche in quel testo è presente la metafora dell'arco e delle frecce richiamata qui nei vv. 13-14, col dire che Amore colpì Petrarca al cuore attraverso gli occhi di Laura mentre a lei, armata, non mostrò neppure l'arco.

\section{La vita fugge, et non s'arresta una hora}

\begin{estratto}
	La vita fugge, et non s’arresta una hora,\\
	et la morte vien dietro a gran giornate,\\
	et le cose presenti et le passate\\
	mi dànno guerra, et le future anchora;\\
	
	e ’l rimembrare et l’aspettar m’accora,\\
	or quinci or quindi, sí che ’n veritate,\\
	se non ch’i’ ò di me stesso pietate,\\
	i’ sarei già di questi penser’ fòra.\\
	
	Tornami avanti, s’alcun dolce mai\\
	ebbe ’l cor tristo; et poi da l’altra parte\\
	veggio al mio navigar turbati i vènti;\\
	
	veggio fortuna in porto, et stanco omai\\
	il mio nocchier, et rotte arbore et sarte,\\
	e i lumi bei che mirar soglio, spenti.\\
\end{estratto}

Il testo è centrato sul raffronto tra il rimpianto del passato e l'angosciosa incertezza del futuro, ovvero tra la consapevolezza che l'amore vano per Laura è stato moralmente un errore e il timore che ciò costi la salvezza spirituale al poeta: l'immagine iniziale della vita che fugge e della morte che la incalza "a gran giornate" (termine militare che anticipa la "guerra" del v. 4) dà l'idea di una coscienza tormentata dai dubbi e dalle ansie spirituali, presentando Petrarca come un uomo assediato e quasi circondato dalle sue angosce ("or quinci or quindi", v. 6). Il v. 8 allude in modo velato al proposito del suicidio ("i’ sarei già di questi penser’ fòra") dal quale è distolto dalla pietà che prova di se stesso, nonché dal timore di una futura punizione dopo la morte che sente ormai prossima e che prelude a un imminente giudizio divino.

\section{Introduzione al Canzoniere}

I Rerum vulgarium fragmenta ("Frammenti di cose volgari") sono una raccolta di 366 liriche di Francesco Petrarca scritte nell'arco di tutta la vita e messe insieme nella forma definitiva negli \textbf{ultimi anni prima della morte}. L'opera è anche impropriamente intitolata Canzoniere e, \textbf{a differenza della Vita nuova di Dante, non ha una cornice narrativa in prosa ma presenta una successione di poesie, tradizionalmente divise tra quelle In vita di madonna Laura (sino al sonetto 264) e quelle In morte di madonna Laura}, benché tale suddivisione non sia resa esplicita dall'autore. \textbf{L'amore per Laura è il tema dominante} della raccolta, ma non mancano altri argomenti come la critica alla corruzione della Curia papale di Avignone, la politica del tempo, mentre alcuni componimenti sono d'occasione e dedicati ad amici e potenti protettori del poeta. \textbf{L'ordine di pubblicazione delle poesie non rispecchia quello di composizione} e infatti il sonetto di apertura è stato certamente scritto tra gli ultimi, quando Laura era già morta e l'autore considera in maniera retrospettiva la sua vita sprecata nell'amore non corrisposto della donna. L'opera ci è stata tramandata da alcuni manoscritti tra cui specialmente il Codice Vaticano Latino 3196, che per buona parte è stato vergato di pugno dallo stesso Petrarca con tanto di annotazioni a margine e dunque del testo possediamo l'autografo (primo caso tra gli autori del Medioevo). Il titolo originale alludeva alla scarsa considerazione che l'autore riponeva in quest'opera, da lui giudicata inferiore agli scritti latini da cui si attendeva la fama, infatti le liriche vengono definite anche nugae, "cose di poco conto" (tale giudizio apparentemente svalutante è probabilmente di maniera e contrasta con l'impegno profuso da Petrarca nel continuo lavoro di rimaneggiamento della raccolta). L'opera è comunque il capolavoro riconosciuto del poeta ed è considerata come la prima raccolta lirica della \textbf{poesia "moderna", con una rappresentazione dell'amore basata molto sull'interiorità dell'autore e con una descrizione della donna amata come creatura terrena, con difetti e soggetta all'invecchiamento, molto lontana quindi dalla idealizzazione propria dello Stilnovo.}\\

Le liriche della raccolta erano da lui chiamate \textbf{nugae}, "cose da poco", definizione di maniera che forse non va intesa in senso spregiativo visto che era usata talvolta anche per le composizioni latine. La raccolta \textbf{non ha un vero e proprio schema narrativo ed è priva di qualunque cornice in prosa, distaccandosi così dai modelli precedenti della Vita nuova e del Convivio}, e se il tema centrale è la storia tormentata dell'amore di Petrarca per Laura non mancano temi d'occasione, come ringraziamenti ad amici e conoscenti o rime encomiastiche per i potenti protettori del poeta, così come liriche di argomento politico (specie le canzoni Spirto gentil e Italia mia) e sonetti di polemica contro la corruzione della Curia papale di Avignone, detta "avara Babilonia".

Petrarca si ispira largamente alla tradizione poetica precedente e certo tra i suoi modelli vi sono la lirica trobadorica e lo Stilnovo, per quanto la sua visione del mondo sia profondamente laica e lontana dall'idealizzazione del sentimento amoroso in chiave religiosa, dunque la donna amata è presentata come creatura seducente che distoglie dalla ricerca del bene e dalla virtù, né può fare da tramite tra uomo e Dio \textbf{come la donna-angelo degli stilnovisti}. Le fonti poetiche dell'autore sono perciò anche e soprattutto i poeti latini della classicità e fra questi vanno citati anzitutto gli elegiaci dell'età di Cesare e Augusto, specie \textbf{Catullo, Tibullo e Properzio} che erano maggiormente noti nel Trecento e la cui opera presenta varie analogie con la storia amorosa tracciata nel Canzoniere, dato che tutti e tre cantano le lodi di una donna nobile e bella, sposata e che si concede loro con un'alternanza di compiacenza e ripulsa (in particolare Catullo, la cui relazione con Lesbia-Clodia è un susseguirsi di rotture e riconciliazioni), mentre fonte di ispirazione è anche \textbf{l'Ovidio degli Amores e delle altre opere elegiache}, non più interpretato come fonte di miti da leggersi in chiave cristiana. Lo stesso \textbf{Catullo è d'altronde citato indirettamente nel sonetto proemiale della raccolta}, in cui Petrarca si rammarica del fatto che a causa dell'amore per Laura è stato lungamente la "favola" del popolo e ricorda il Carme 8 del poeta latino laddove diceva "Miser Catulle, desinas ineptire" ("Povero Catullo, smetti di fare il pazzo"; TESTO: \nameref{sec:Voi ch'ascoltate in rime sparse il suono}).

\end{document}