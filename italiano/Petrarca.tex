\documentclass[10pt,a4paper]{article}
\usepackage[utf8]{inputenc}
\usepackage[T1]{fontenc}
\usepackage{amsmath}
\usepackage{amsfonts}
\usepackage{amssymb}
\usepackage{makeidx}
\usepackage{graphicx}

\usepackage{hyperref}

\usepackage{xcolor}     % for colour
\usepackage{mdframed}   % for framing

\usepackage[left=1.00in, right=1.00in, top=1.00in, bottom=1.00in]{geometry}
\author{Tommaso Severini}
\title{Letteratura italiana - Petrarca}

\parindent 0ex
\begin{document}
	\maketitle
	
	\newmdtheoremenv[%
	linecolor=gray,leftmargin=60,%
	rightmargin=40,
	backgroundcolor=gray!40,%
	innertopmargin=5pt,%
	font=\ttfamily]{estratto}{Extract}[section]
	
	\section[Domanda 1]{Voi ch'ascoltate in rime sparse il suono}
	
	\begin{estratto}
		Voi ch’ascoltate in rime sparse il suono\\
		di quei sospiri ond’io nudriva ’l core\\
		in sul mio primo giovenile errore\\
		quand’era in parte altr’uom da quel ch’i’ sono,\\
		
		del vario stile in ch’io piango et ragiono\\
		fra le vane speranze e ’l van dolore,\\
		ove sia chi per prova intenda amore,\\
		spero trovar pietà, nonché perdono.\\
		
		Ma ben veggio or sì come al popol tutto\\
		favola fui gran tempo, onde sovente\\
		di me medesmo meco mi vergogno;\\
		
		et del mio vaneggiar vergogna è ’l frutto,\\
		e ’l pentersi, e ’l conoscer chiaramente\\
		che quanto piace al mondo è breve sogno.\\
	\end{estratto}

Petrarca si rivolge ai lettori in grado di comprendere per esperienza le pene amorose, con un topos che rimanda a molte \textbf{liriche dello Stilnovo} (Donne ch'avete intelletto d'amore), anche se l'effusione del sentimento da parte sua sarà spesso un soliloquio: il poeta chiede perdono per i lamenti da lui prodotti nell'illusione di un \textbf{amore infelice}, che ora (a distanza di anni e col bagaglio della raggiunta maturità) \textbf{egli giudica un "giovenile errore" e un "vaneggiare"}, poiché tutto quello che piace al mondo è un sogno destinato a finire presto. Petrarca è anche consapevole che l'aver perso tempo dietro a Laura lo ha distolto dalla sua "missione" di intellettuale impegnato e lo ha esposto alla derisione del volgo, tema tipicamente classico che \textbf{rimanda al carme 8 di Catullo} ("Miser Catulle, desinas ineptire / et quod vides perisse perditum ducas", "Povero Catullo, smetti di fare il pazzo e ritieni perduto ciò che hai visto che è andato perduto"). La paura di essere deriso dal volgo tornerà anche nel sonetto \nameref{sec:Solo et pensoso} (35), in cui l'autore rifugge la compagnia degli altri uomini per non svelare la sua condizione interiore.

\section{}


\section{Solo et pensoso}
\label{sec:Solo et pensoso}
\end{document}