\documentclass[10pt,a4paper]{article}
\usepackage[utf8]{inputenc}
\usepackage[T1]{fontenc}
\usepackage{amsmath}
\usepackage{amsfonts}
\usepackage{amssymb}
\usepackage{makeidx}
\usepackage{graphicx}
\usepackage[left=1.00in, right=1.00in, top=1.00in, bottom=1.00in]{geometry}
\author{Tommaso Severini}
\title{Letteratura italiana - Divina Commedia - Canti I III V}
\begin{document}
	\maketitle
	
	\section{Introduzione}
	
	\subsection{Metrica e forma}
	
	La "Commedia", titolo originale dell'opera, successivamente definita "Divina" da Boccaccio è un poema rigoroso dal punto di vista metrico e si basa sul numero, in particolare l'1 e il 3, che riprendono gli ideali cristiani di Unità e Trinità. Infatti, l'opera è divisa in 3 cantiche, di 33 canti ciascuna, di cui la prima presenta un canto in più di introduzione(Nel mezzo del cammin di nostra vita...), per un totale di 100 canti, 10 volte 10, che riprende sia il concetto di unità sia il numero perfetto. Questa opera è divisa in terzine endecasillabe e presenta una schema di rime incatenato(ABA BCB CDC ... YZY Z) 
	
	\subsection{La composizione}
	
	Nonostante l'epoca di composizione sia incerta, possiamo definire abbastanza precisamente il periodo di tempo in cui Dante compose ogni cantica della Commedia. A partire dal \textbf{1306}, anno in cui la composizione del Convivio si interruppe per presumibilmente iniziare a comporre l'\textit{Inferno}, che fu probabilmente completato nel \textbf{1309} e menzionato nel 1314 da Francesco da Barberino nel suo \textit{Documenti d'amore}. Il \textit{Purgatorio} fu composto tra il \textbf{1910 e il 1914} e divulgato un paio di anni dopo. Infine, l'ultima cantica della \textit{Commedia}, il \textit{Paradiso}, fu scritta tra il \textbf{1315 e il 1321}.
	
	\subsection{Genere}
	
	Nonostante sia impossibile individuare il genere di un'opera talmente vasta e che copre i più svariati degli argomenti, sappiamo poichè Dante abbia voluto definire la sua opus magna "Commedia". Infatti, egli, in una lettera, spiega che ha scelto questo titolo perchè, all'inizio dell'opera, il Dante personaggio vive momenti di terrore e paura, ma, \textbf{nel finale, vive un lieto fine}. La seconda ragione riguarda la distinzione che veniva fatta al tempo di Dante tra i generi teatrali; la tragedia era definita alta e sublime, mentre \textbf{la commedia era considerata medio-bassa}. Poichè questa opera è definita "piana e umile", essa viene definita una commedia.
	
	\subsection{Il viaggio delle anime e nella società}
	
	Il vero obiettivo di Dante era quello di illustrare lo "stato delle anime" del Medioevo attraverso una scelta narrativa rivoluzionaria. Tutto ciò, con l'aggiunta di alcuni elementi autobiografici, \textbf{trasforma l'opera in un viaggio nella coscienza} del Sommo Poeta.
	
	Per scrivere tutto ciò, Dante si ispira sicuramente alla letteratura didascalica medievale, ma riprende anche molti elementi religiosi e profezie bibliche per il \textbf{connotamento morale}. Anche le opere di Brunetto Latini fungono da fonte di ispirazione, come in altre opere di Dante, ma, soprattutto, il viaggio nell'inferno riprende molti elementi del viaggio nell'ade che avviene nell'\textit{Eneide} di \textbf{Virgilio}, che gli fungerà da guida nei primi due cantici. 
	
	Poichè questa opera riprende molti elementi risalenti el periodo in cui Dante è vissuto, la Commedia può essere usata per definire molti aspetti della società dell'epoca, ricevendo addirittura la descrizione di "Sintesi culturale del Medioevo".
	
	\subsection{Dante agens e auctor}
	
	
	
\end{document}