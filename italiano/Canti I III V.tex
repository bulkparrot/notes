\documentclass[10pt,a4paper]{article}
\usepackage[utf8]{inputenc}
\usepackage[T1]{fontenc}
\usepackage{amsmath}
\usepackage{amsfonts}
\usepackage{amssymb}
\usepackage{makeidx}
\usepackage{graphicx}
\usepackage[left=1.00in, right=1.00in, top=1.00in, bottom=1.00in]{geometry}
\author{Tommaso Severini}
\title{Letteratura italiana - Divina Commedia - Canti I III V}
\begin{document}
	\maketitle
	
	\section{Introduzione}
	
	\subsection{Metrica e forma}
	
	La "Commedia", titolo originale dell'opera, successivamente definita "Divina" da Boccaccio è un poema rigoroso dal punto di vista metrico e si basa sul numero, in particolare l'1 e il 3, che riprendono gli ideali cristiani di Unità e Trinità. Infatti, l'opera è divisa in 3 cantiche, di 33 canti ciascuna, di cui la prima presenta un canto in più di introduzione(Nel mezzo del cammin di nostra vita...), per un totale di 100 canti, 10 volte 10, che riprende sia il concetto di unità sia il numero perfetto. Questa opera è divisa in terzine endecasillabe e presenta una schema di rime incatenato(ABA BCB CDC ... YZY Z) 
	
	\subsection{La composizione}
	
	Nonostante l'epoca di composizione sia incerta, possiamo definire abbastanza precisamente il periodo di tempo in cui Dante compose ogni cantica della Commedia. A partire dal \textbf{1306}, anno in cui la composizione del Convivio si interruppe per presumibilmente iniziare a comporre l'\textit{Inferno}, che fu probabilmente completato nel \textbf{1309} e menzionato nel 1314 da Francesco da Barberino nel suo \textit{Documenti d'amore}. Il \textit{Purgatorio} fu composto tra il \textbf{1310 e il 1314} e divulgato un paio di anni dopo. Infine, l'ultima cantica della \textit{Commedia}, il \textit{Paradiso}, fu scritta tra il \textbf{1315 e il 1321}.
	
	\subsection{Genere}
	
	Nonostante sia impossibile individuare il genere di un'opera talmente vasta e che copre i più svariati degli argomenti, sappiamo poichè Dante abbia voluto definire la sua opus magna "Commedia". Infatti, egli, in una lettera, spiega che ha scelto questo titolo perchè, all'inizio dell'opera, il Dante personaggio vive momenti di terrore e paura, ma, \textbf{nel finale, vive un lieto fine}. La seconda ragione riguarda la distinzione che veniva fatta al tempo di Dante tra i generi teatrali; la tragedia era definita alta e sublime, mentre \textbf{la commedia era considerata medio-bassa}. Poichè questa opera è definita "piana e umile", essa viene definita una commedia.
	
	\subsection{Il viaggio delle anime e nella società}
	
	Il vero obiettivo di Dante era quello di illustrare lo "stato delle anime" del Medioevo attraverso una scelta narrativa rivoluzionaria. Tutto ciò, con l'aggiunta di alcuni elementi autobiografici, \textbf{trasforma l'opera in un viaggio nella coscienza} del Sommo Poeta.
	
	Per scrivere tutto ciò, Dante si ispira sicuramente alla letteratura didascalica medievale, ma riprende anche molti elementi religiosi e profezie bibliche per il \textbf{connotamento morale}. Anche le opere di Brunetto Latini fungono da fonte di ispirazione, come in altre opere di Dante, ma, soprattutto, il viaggio nell'inferno riprende molti elementi del viaggio nell'ade che avviene nell'\textit{Eneide} di \textbf{Virgilio}, che gli fungerà da guida nei primi due cantici. 
	
	Poichè questa opera riprende molti elementi risalenti el periodo in cui Dante è vissuto, la Commedia può essere usata per definire molti aspetti della società dell'epoca, ricevendo addirittura la descrizione di "Sintesi culturale del Medioevo".
	
	\subsection{Dante agens e auctor}
	
	 Proprio perchè la funzione del personaggio di Dante è etico-didascalica, egli rappresenta anche tutta l'umanità. Dante deve rappresentare il popolo cristiano e il suo viaggio che va dal bene al male. Oltre che su questo piano, esistono due "personaggi" che rappresentano il poeta: il Dante \textbf{agens}, che è ignaro di ciò che lo aspetta e che ha il compito di apprendere il più possibile da questo viaggio, e il Dante \textbf{auctor}, soggetto della scrittura dell'opera, onniscente e che ha il compito di rivelare la verità.
	 
	 \section{Allegorie e simboli}
	 
	 La Divina Commedia è definibile un poema didascalico-allegorico; infatti, vuole insegnare sulle grandi verità morali e religiose attraverso l'utilizzo di immagini che hanno significato simbolico.
	 
	 \subsection{Allegorie}
	 
	 Nonostante la Commedia narri solamente del viaggio di Dante attraverso i tre regni dell'aldilà, esso non va interpretato solamente così, ma bisogna considerare anche il suo significato allegorico. L'allegoria è una tecnica spesso usata nel medioevo poichè consentiva di poter rappresentare concetti astratte con immagini concrete, Quindi, il viaggio di Dante, va interpretato come il viaggio che ogni cristiano deve compiere per purificare la propria anima e poter andare in paradiso.
	 
	 \subsection{I quattro sensi delle scritture}
	 
	 Dante, nel Convivio, definisce 4 possibili interpretazioni per le Sacre Scritture:
	 \begin{itemize}
	 	\item Il senso letterale
	 	\item Il senso allegorico ricavato da quello letterale che conduce alle verità nascoste dell'opera
	 	\item Il senso etico, che spiega il comportamento che gli uomini dovrebbero assumere
	 	\item Il senso anagogico, ovvero quello spirituale, posseduto solo dalle sacre scritture, che correla ad ogni avvenimento delle Scritture al benessere dell'anima. 
	 \end{itemize}
 
 	Nel Medioevo, i padri della Chiesa hanno cercato di conciliare l'Antico Testamento con il Nuovo e sono riusciti a fare ciò attraverso il modello tipologico. Secondo questo modello, una delle due opere sarebbe solamente una prefigurazione della successiva, mentre l'altra completerebbe la precedente. 
 	
 	L'interpretazione figurale delle sacre scritture può essere applicata anche nella Commedia. Infatti, mentre alcuni elementi, come le tre fiere, sono prive di storia, e quindi rappresentano solo delle allegorie, altre figure, \textbf{come Virgilio, Beatrice e Catone}, presentano una storia e sono raffigurate come \textbf{prefigurazioni complete} nella Commedia. Ad esempio, Virgilio diviene guida, pozzo di scienza e di letteratura, tutte qualità presenti nel personaggio storico, ma completate e rese perfette. 
 	
 	\section{I luoghi}
 	
 	\subsection{Struttura dell'inferno}
 	
 	L'Inferno era rappresentato all'epoca di Dante come una cavità di forma conica interna alla Terra, allora concepita come divisa in due emisferi, uno di terre e l'altro di acque. La caverna infernale era nata dal ritrarsi delle terre inorridite al contatto con il corpo maledetto di Lucifero e delle sue schiere, cadute dal cielo dopo la ribellione a Dio. La voragine infernale aveva il suo ingresso esattamente sotto Gerusalemme, collocata al centro della semisfera occupata dalle terre emerse, ovvero dal continente euroasiatico. Agli antipodi di Gerusalemme, e quindi al centro della semisfera acquea, si ergeva l'isola montagnosa del Purgatorio, composta appunto dalle terre fuoriuscite dal cuore del mondo all'epoca della ribellione degli angeli. In cima al Purgatorio, Dante colloca il Paradiso terrestre del racconto biblico, il luogo terrestre più vicino al cielo.
 	
 	La struttura cosmologica della Commedia è strettamente connessa alla struttura dottrinale del poema, per cui la collocazione dei tre regni, e, al loro interno, l'ordine delle anime (ovvero delle pene e delle grazie), corrisponde a precisi intendimenti di ordine morale e teologico.
 	
 	In particolare, la topografia dell'Inferno comprende i seguenti luoghi:
 	\begin{itemize}
 \item	Un ampio vestibolo o Antinferno, dove vengono puniti coloro che nessuno vuole, né Dio né il demonio: gli ignavi.
 \item	Il fiume Acheronte, che separa il vestibolo dall'Inferno vero e proprio.
 \item	Una prima sezione costituita dal Limbo, immerso in una tenebra perenne.
 \item	Una serie di cerchi meno scoscesi in cui patiscono i peccatori incontinenti.
 \item	La città infuocata di Dite, le cui mura circondano la voragine finale.
 \item	Il cerchio dei violenti in cui scorre il fiume sanguigno del Flegetonte.
 \item	Un burrone scosceso, che dà all'ottavo cerchio, chiamato Malebolge: il cerchio dei fraudolenti.
 \item	Il pozzo dei Giganti.
 \item	Il lago ghiacciato di Cocito, dove sono immersi i traditori.
 \end{itemize}
 	
 	\subsection{Cosmologia dantesca}
 	
 	La struttura del Paradiso è costruita sul \textbf{sistema geocentrico di Aristotele e di Claudio Tolomeo}: al centro dell'universo sta la Terra, nella regione sublunare, e intorno ad essa nove sfere concentriche, responsabili del movimento dei pianeti. Mentre l'Inferno è un luogo presente sulla Terra, il Paradiso è un mondo immateriale, etereo, diviso in nove cieli: i primi sette prendono il nome dai corpi celesti del sistema solare (nell'ordine Luna, Mercurio, Venere, Sole, Marte, Giove, Saturno), gli ultimi due sono costituiti dalla sfera delle stelle fisse e dal Primo mobile. Il tutto è contenuto nell'Empireo, dove è possibile trovare la rosa candida degli angeli che adorano Dio.
 	
 \section{Canto I}
 
 	\subsection{Simboli e allegorie}
 	
 		\subsubsection{La selva}
 		
 		La selva, a differenza di altri personaggi, non presenta un significato figurale, perchè privo di una prefigurazione storica. Nonostante ciò, essa assume una \textbf{connotazione psicologica} e viene descritta come "oscura", "selvaggia" e "forte". Questa descrizione deve far comprendere l'effetto che la foresta ha sull'animo del poeta e aiuta il lettore a distaccarsi dal senso letterale dell'opera per avvicinarsi a quello allegorico. La selva, quindi, rappresenta la \textbf{debolezza morale e intellettuale degli uomini e l'inclinazione al peccato}.
 		
 		\subsubsection{Le fiere}
 		
 		Quando Dante sta per salire sul colle per uscire dalla selva, il suo percorso è ostacolato da tre fiere. Il significato di questi esseri, nonostante si sappia che hanno un significato allegorico, non è chiaro. Nonostante ciò, sono state formulate diverse ipotesi. 
 		\begin{itemize}
 			\item L'ipotesi più probabile è quella che vede rappresentare tre peccati capitali, quali lussuria, superbia e cupidigia.
 			\item Potrebbero rappresentare tre colpe della società del periodo di Dante
 			\item Potrebbero rappresentare le tre classi dei peccati delle anime dannate dell'inferno: incontinenti, violenti e fraudolenti.
 			\item Potrebbero anche rappresentare le più importanti potenze politiche dell'epoca, quali Firenze, Impero e Papato.  
 		\end{itemize}
 	
 		\subsubsection{Il colle}
 		
 		Attraverso numerose spie testuali, come l'utilizzo di parole che richiamano il bene (il sole che illumina la sommità del monte) e a riferimenti biblici ("Chi salirà sul monte del signore" Salmi), è facile osservare come il colle rappresenti il totale capovolgimento dal punto si vista etico e morale della selva. Esso, a differenza della selva, rappresenta la \textbf{vita secondo virtù} e \textbf{orientata al bene}.
 		
 		\subsubsection{Virgilio}
 		
		Virgilio, in questa opera, funge sia da personaggio-allegoria sia da personaggio-funzione. Egli, infatti, oltre a rappresentare la \textbf{ragione}, fa anche da guida a Dante attraverso l'inferno e il purgatorio. \textbf{Il suo ruolo è quello di rimuovere ostacoli e anche mostrare la verità}, utilizzando, ad esempio, profezie come quelle del Veltro, e guidando Dante (e l'intera comunità cristiana) attraverso questo arduo percorso spirituale. Essendo portatore di \textbf{giustizia, sapienza e religiosità}, in sua presenza il registro si eleva, con l'utilizzo di latinismi e diverse figure retoriche. La \textbf{profezia del Veltro} assume toni più scuri, con lessico appartenente al campo semantico dell'oscurità e spesso ambiguo, fino a rendere la profezia indecifrabile. Grazie al perfezionamento che Dante fornisce a Virgilio attraverso l'interpretazione figurale della Commedia, Virgilio si trova completato anche dal punto di vista spirituale, divenendo portatore esemplare dei \textbf{valori cristiani}, a lui sconosciuti in età pre-cristiana.
		
		\subsubsection{Il veltro}
		
		Nonostante sia impossibile decifrare il significato della profezia del Veltro a causa del suo linguaggio oscuro e antico, è possibile formulare ipotesi molto attendibili che darebbero la possibilità di spiegarne il significato.
		
		Secondo la teoria più accreditata, il Veltro non rappresenta altro che l'imperatore. Infatti, è facile notare come molte parole della profezia facciano riferimento all'impero, come \textbf{ordine, giustizia e autorità}, anche quando ci si riferisce a Dio. Nonostante ciò, l'imperatore di cui si ipotizza potrebbe rappresentare sia l'\textbf{ultimus imperator}, imperatore che otterrà il controllo sulle terre prima della seconda venuta di Cristo, sia lo storico Enrico VIII, che potrebbe permettere a Dante di poter ritornare nella sua amata Firenze. Nonostante il Veltro debba possedere le \textbf{doti trinarie} di \textbf{sapienza, amore e virtute}, Dante crede che il potere imperiale derivi da Dio, il quale può concedere all'imperatore queste doti per esercitare in maniera ottimale il potere.
		
	\section{Canto III}
	
		\subsection{I luoghi}
		
		\subsubsection{La prosopopea}
		
		Sopra la porta è riportata una scritta ch indica che è la porta stessa a parlare. La scritta comunica che, una volta vagliata la soglia, non è possibile tornare indietro e che non si può uscire dall'inferno. comunica anche il fatto che la porta è stata creata da Dio dopo la lotta con Lucifero. Questa concezione proviene dalla tradizione classica e religiosa.
		
		\subsection{I personaggi}
		
		\subsubsection{Gli ignavi} 
		
		Gli ignavi si trovano dopo la porta dell'inferno, ma prima del fiume Acheronte, quindi nell'\textbf{antinferno}. Essi sono una invenzione dantesca (Virgilio non ne parla nelle sue opere) nonostante Virgilio sia il modello fondamentale dell'inferno. Loro sono coloro che in vita non hanno mai preso posizioni. Poichè non hanno mai osato agire nè nel bene nè nel male, sono, secondo la legge del contrappasso, costretti a inseguire un'insegna mentre sono punti da vespe e mosconi. Dante operò scelte molto importanti, ma anche pericolose, perciò ritiene che chi rifiuta di prendere siano dei peccatori terribili, permettendo a personaggi malvagi di agire. Tra questi è presente una figura identificata come Celestino V,papa che rifiutò al pontificato facendo salire al trono Bonifacio VIII.
		
		\subsubsection{Caronte}
		
		Arrivato al \textbf{fiume Acheronte}, superato l'antinferno, Dante incontra il primo demone infernale, Caronte, trghettatore delle anime e figura ripresa dalla letteratura virgiliana. Nonostante ciò, il Caronte dantesco ha una connotazione psicologica molto più profonda rispetto a quello virgiliano. Egli, per prima cosa, cercherà di impedire al Sommo Poeta di accedere all'inferno, ma, successivamente, verrà zittito da Virgilio, il quale utilizzerà una frase formulare che verrà ripetuta molto volte nella Commedia: \textbf{Vuolsi così colà dove si puote ciò che si vuole, e più non dimandare}.
 	
\end{document}