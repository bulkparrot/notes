\documentclass[10pt,a4paper]{article}
\usepackage[utf8]{inputenc}
\usepackage[T1]{fontenc}
\usepackage{amsmath}
\usepackage{amsfonts}
\usepackage{amssymb}
\usepackage{makeidx}
\usepackage{graphicx}
\usepackage[left=1.00in, right=1.00in, top=1.00in, bottom=1.00in]{geometry}
\author{Tommaso Severini}
\title{Letteratura italiana - Boccaccio}
\begin{document}
	\maketitle

	\section{Vita}

	\subsection{Infanzia}

	Giovanni Boccaccio nasce a Firneze nel 1313 da Boccaccino di Chelino, mercante, con una donna di umili condizioni. Questo fatto influenzò la poetica di Boccaccio, che nelle sue opere afferma di essere figlio di una nobildonna parigina o dello stesso re di Frnacia. Riconosciuto solo nel 1320 dal padre, egli riceve una formazione classica.

	\subsection{Periodo napoletano}

	 Da questo momento in poi iniziano gli anni più belli della vita del poeta. Egli, infatti, si trasferisce con il padre nella corte dei Bardi, importante famiglia legata agli Angoini,e si dedica all'apprendimento della \textbf{tradizione stilnovista} grazie a Cino da Pistoia. In seguito, Boccaccio comincia a frequentare la corte angioina, ricca di libri e testi, dove studia la \textbf{letteratura cortese e romanza}. Scruve alcune opere minori tra cui il \textit{Teseida, il Filocolo, il Filostrato} e la \textit{Caccia di Diana}. Durnate questo periodo impara alcune nozioni grammaticali del greco(che successivamente lo spingerà a istituire una cattedra di greco all'università di Firenze).

	 \subsection{Periodo fiorentino}

	 Boccaccio è costretto a ritornare a Firenze a causa del padre. Ciò rende infelice Boccaccio. che si ritrova in un'ambiente troppo provinciale e ristretto, dove comunque cerca di accattivare la popolazione con alcune opere minori.
	 A causa della sua condizione economica, Boccaccio è costretto a cercare asilo tra le corti nobiliari emiliane. Infatti, tra il 1345 e il 1346 risiede a Ravenna dove viene in contatto con l'ambiente dove Dante aveva trascorso i suoi ultimi anni di vita.
	 Dopo di che, nel 1348 è costretto a tornare a Firenze, dove, durante l'epidemia di peste e in seguito alle morte di alcuni suoi cari, decide di scrivere la sua \textit{opus magna}: il \textbf{Decameron}, presumibilmente completato nel 1351.

	 \subsection{Periodo fiorentino-certaldese}

	 Questo periodo rappresenta sicuramente quello più buio nella vita di Boccaccio. A causa di una fallita congiura organizzata da dei suoi conoscenti perde gli incarichi al comune. Nonostante ciò, dopo qualche anno riassume i suoi ruoli al comune e dieviene ambasciatore nella corte papale ad Avignone. Durante i suoi ultimi anni di vita termina le sue opere in latino e rafforza la cultura umanistica a Firenze.

	 Egli curò un commento delle opere di Dante, sia in pubblico che in forma scritta.

	 \section{Stile e opere}

	 La produzione letteraria di Boccaccio può essere suddivisa in tre fasi, separate dalla cmposizione del \textbf{Decameron}. A differenza di Petrarca, le sue opere tendono ad assumere un carattere più narrativo che lirico, caratterizzato dalla \textbf{scomparsa dell'io dell'autore}.

	 \subsection{La cultura di Boccaccio}

	 Nonostante cerchi di scrivere e di esercitare la sua arte in quanto tale, Boccaccio cerca sempre di usare le sue opere per descrivere il contesto sociale in cui la vicenda prendeva luogo. Anche il fatto di essere nato a Firenze ma vissuto nelle corti signorili di Napoli gli permise di ottenere una \textbf{chiara visione della classe mercantile} insieme ad un esaustivo approfondimento delle opere cortesi, ma anche di quelle di \textbf{autori classici, come Virgilio, Ovidio e Stazio}. Oltre a ciò, una caratteristica fondamentale della produzione artistica di Boccaccio riguardo il suo \textbf{utilizzo dei più svariati generi narrativi}.

	 \section{Produzione artistica}

	 Nel primo periodo della sua vita, Boccaccio stende un'opera che modificherà fino alla fine della sua vita. Trattano principalmente il tema dell'amore, ma sono inclusi anche alcuni elementi autobiografici, entrambi raccontati attraverso la tradizione cortese.

	 Nel periodo fiornetino si dedica ad opere meno autobiografiche, ma più vicine alla relatà di Firenze e del pubblico comunale. Queste comprendono la "Commedia delle ninfe fiorentine", la "Ninfale fiesolano" e la "Elegia di Madonna Fiammetta".

	 Nell'ultimo periodo Boccaccio si dedica alla stesura di opere enciclopediche, ma anche di commenti delle opere dantesche, che espone anche in pubblico. Una opera importante da ricordare è sicuramente \textbf{Il Corbaccio}, che può essere descritta come un'invettiva contro la donna, che si conclude con l'avvicinamento dell'autore ad uno studio più ampio.

	 \section{Poetica boccacciana}

	 Come abbiamo già visto, la produzione artistica di Boccaccio può essere suddivisa in tre parti. Nonostante ciò, il poeta cerca di mantenere degli elementi fondanti fissi, che ci permettono di osservare le basi del pensiero del poeta.
	 \begin{itemize}
	 	\item Primo elemento da ricordare è sicuramente la funzione non solo consolatoria (le sue opere erano indirizzate a coloro che sperimentano sofferenze e affanni dell'esistenza umana) ma anche di mezzo che illustra la reale possibilità per l'uomo di raggiungere la \textbf{felicità}.
		\item Oltre a ciò, la letteratura deve divenire nutrimento essenziale dell'uomo, che, attraverso di essa, valorizza gli aspetti positivi della sua esistenza e permette di accettare le debolezze, creando una convivenza più felice e serena.
	 \end{itemize}
	 Nonostante ciò possa far apparire la letteratura molto simile alla filosofia, Boccaccio tiene a precisare la differenza tra queste ultime. Infatti, secondo l'autore, la letteratura prevede solo che la verità professata dal filosofo sia comunicata in modo esplicito, la letteratura nasconde sotto il \textbf{velo dell'invenzione} la verità che vuole trasmettere al lettore. \\

	  La poetica di Boccaccio ha sicuramente avuto uno stretto dialogo con le opere di altri 2 grandi esponenti della letteratura trecentesca: Dante, di cui Boccaccio fu copista e commentatore, e Petrarca, con cui il poeta condivide un progetto di valorizzazione della letteratura calssica (anticipando di qualche tempo l'umanesimo). Nonostante egli si ispiri a questi due grandi poeti e lui stesso si definisca più volte non più di un epigono, egli fu un grande innovatore: a partire dalla sua \textit{opus magna} il Decameron, per poi passare all'epica in volgare, al primo romanzo in prosa e all'inizio della tradizione bucolica.

		Nonostante tutto ciò il vero motore che alimenta l'evoluzione della letteratura boccacciana è la contrapposizione tra \textbf{letteratura come gioco e diletto} e \textbf{letteratura come strumento di elevazione morale}.

		Infatti, nelle opere precedenti al Decameron, è facile vedere come Petrarca scriva opere in cui si ha una concezione di letteratura che è fine a se stessa. In essa l'ispirazione letteraria e l'ispirazione amorosa coincidono e l'autore si rivolge alle donne, considerate un pubblico molto privilegiato. Ciò rende queste opere degli specchi in cui sono osservabili i pensieri ed i desideri dell'autore.

		Al contrario, nelle opere che seguono il Decameron, le muse ispiratrici riprendono la loro forma divina originale e Boccaccio si dedica all'elevamento morale dei suoi lettori. Tipico elemento di questa tipologia di opere è la centralità dell'allegoria, che sottolinea come la letteratura debba comunicare un inseganemento morale. Allo stesso tempo, l'avvicinamento di Boccaccio alla letteratura classica porta il poeta a considerare le opere a lui contemporanee come frivole.

		Il Decameron assume una posizione mediana tra i due stili appena descritti, ma senza nessun compromesso, anzi, in uno stato di equilibrio. Infatti, è possibile vedere come nel Decameron siano presenti sia elementi riconducibili alle prime opere, come la componente del desiderio e l'inventiva nelle novelle. Al contempo, è però evidente una forte componente di controllo e razionalizzazione, come la contestualizzazione delle novelle e le descrizioni che Boccaccio ci fornisce delle varie classi sociali. Oppure, come da un lato possiamo osservare la forte componente dell'amore, in particolare nella giornata IV, dall'altro è evidente la componente del giudizio morale che Boccaccio esprime sul comportamento degli uomini.

	 \section{Il decameron}

	 Opera composta tra il 1348 e il 1351, tratta della visone laica e terrena del mondo. Il suo titolo significa \textit{dieci giornate}, nel corso dei queli sono narrate 100 novelli da 10 giovani, 7 ragazze e 3 ragazzi. La \textbf{cornice} dell'opera è rappresentata dal contesto in cui essa prende luogo, ovvero la peste di Firenze che infuria nella città nel 1348.

	 Essa si presenta come un'opera rivoluzionaria, non più costituita da un lungo racconto ma da \textbf{novelle}, ma anche per la sua \textbf{pluralità dei temi}. In particolare, questa opera si differenzia da una precedente raccolta di storie, il Novellino, poichè le novelle sono organizzate in blocchi tematici e secondo una sequenza logica e poichè la situazione appena descritta si svolge in una cornice, al cui interno sono solo i ragazzi a "interagire" con le novelle.

		Le sue finalità sono raccolte già nel sottotitolo dell'opera stessa: "cognominato prencipe Galeotto". Ciò allude al personaggio di Galeeotto appartenente alle storie di Lancillotto e Ginevra, in cui egli svolge un ruolo da consigliere e consolatore delle anime afflitte dall'amore; chiaro riferimento al ruolo consolatore dell'opera nei confronti delle donne che soffrono per amore a cui l'opera è dedicata. \textbf{Per la prima volta la condizione della donna è messa al primo posto}.

		Il ruolo della cornice, ripreso da opere classiche ma anche appartenenti alla letteratura orientale, svolge un ruolo fondamentale dal punto di vista metaforico. Infatti, la fuga dei ragazzi è una metafora che rappresenta la lotta per il diritto alla vita e la villa in cui si rifugiano diviene un locus amoenus in cui si può convivere civilmente.
		Oltre a ciò, la cornice permette di strutturare l'intera opera seconda un rigore logico, che traforma il progredire dei giorni in un percorso ascensionale, che parte nel mondo dei mercanti e degli usurai (da Ser Ciappelletto) e si conclude nel mondo feudale (novella di Griselda). Ciò permette di visualizzare chiaramente \textbf{vizi e virtù degli uomini vissuti al tempo di Boccaccio}, ma senza pregiudizi da parte dell'autore.

\section{Novelle}

\subsection{Introduzione e dedica}

Nell'introduzione al suo capolavora, compare il narratore esterno rappresentato da Boccaccio stesso, che indica a chi sia indirizzata l'opera, quale sia il suo intento e perchè abbia deciso di utilizzare un particolare stile. Questo testo può essere suddiviso in tre sezioni:
\begin{itemize}
	\item Nella prima parte, Boccaccio spiega quale sia il suo intento, ovvero quello di consolare, come quando lui stesso fu aiutato dagli amici nei momenti di sconforto.
	\item Nella seconda parte, Boccaccio specifica il destinatario dell'opera, ovvero le donne che soffrono per amore. Infatti, l'autore riconosce la situazione della donna nel medioevo e descrive come esse sia costrette a nascondere le loro sofferenze.
	\item Nella terza parte, egli specifca il tipo di consolazione che le donne riceveranno: le cento novelle del Decameron, che allieteranno sia lo spirito che la mente.
\end{itemize}

La prima tematica che sicuramente emerge è quella della \textbf{letteratura come rimedio alla morte}, che si evince dal discorso che Boccaccio pronuncia dicendo che i "ragionementi" dei suoi amici lo hanno salvato dalla "morte".

Boccaccio è il primo autore a \textbf{descrivere e compatire la situazione della donna nel medioevo}, alla quale vengono private molte libertà, costrette in casa. Nonostante ciò, Boccaccio non crede che questa situazione si origini da un'intrinseca minorità della società, ma solamente un "peccato della sfortuna", un evento sfortunato. Per questo l'uomo di deve adoperare ed utilizzare il suo ingegno per mitigare la situazione.

\subsection{La peste e la "lieta brigata"}

In questo brano, Boccaccio descrive  la comparsa della peste nera a Firenze e la formazione con successiva fuga della "lieta brigata". Il brano è diviso in tre macrosequenze:
\begin{itemize}
	\item Nella prima sequenza, Boccaccio si scusa anticipatamente per dover rievocare la dolorosa epidemia di peste che aveva colpito la città di Firenze qualche anno prima.
	\item Nella seconda sequenza, Boccaccio descrive gli effetti dell'epidemia sulla popolazione: scomparsa di ogni norma di convivenza civile e riduzione di esseri umani allo stato di bestie.
	\item Nell'ultima sequenza, il poeta descrive la formazione della "lieta brigata" e la loro fuga nella villa dove passeranno i prossimi 10 giorni, raccontandosi novelle.
\end{itemize}

Questo evento, nonostante si presenti, come lo definisce Boccaccio, un "orrido cominciamento", esso è essenziale per comprendere uno dei principi fondamentali della poetica boccacciana. Infatti, questo evento permette di contestualizzare le successive 100 novelle in un contesto realistico. \textbf{Boccaccio non vuole rappresentare un mondo immaginario, ma il mondo reale, che viene affrontato attraverso la letteratura. Ciò permette di comprendere meglio la realtà e di organizzarla in modo razionale.}

Altro elemento importante riguarda la "lieta brigata". Essa, appena raggiunta la villa, cerca di ricostruire le norme per un vivere civile ma libero attraverso il racconto delle novelle.

\subparagraph{Sir Ciappelletto}

Il tema centrale della prima novella del Decameron è sicuramente la beffa, in questo caso portata all'estremo. Ciappelletto, infatti, non è solo riuscito a mettere a ridicolo la sacralità della confessione, ma anche di prendersi gioco di tutto il popolo, facendosi acclamare come santo.

Questo risvolto apparentemente positivo ha portato molti a pensare di questa novella come un exemplum rovescaito. Ovviamente, il lieto fine è solo apparente, perchè sarà la giustizia divina a definire il destino di Ciappelletto, ma questa visione è riservata solamente a chi professa una fede cristiana.

Nonostante questo testo sembri apparentemente semplice, in realtà esistono molte interpretazioni possibili:
\begin{itemize}
	\item La prima vede questa novellla come una parodia del modello di santità proposto dalla chiesa.
	\item La seconda vede l'ingegno e l'astuzia di Ciappelletto come una critica ai valori fondanti della classe mercantile vissuta nel periodo di Boccaccio, con cui quest'ultimo aveva avuto molta esperienza.
	\item Un'ultima interpretazione vede Ciappelletto, maestro di retorica e con interessi mondani, una rappresentazione del profilo di Boccaccio.
\end{itemize}



\end{document}
