\documentclass[10pt,a4paper]{article}
\usepackage[utf8]{inputenc}
\usepackage[T1]{fontenc}
\usepackage{amsmath}
\usepackage{amsfonts}
\usepackage{amssymb}
\usepackage{makeidx}
\usepackage{graphicx}
\usepackage[left=1.00in, right=1.00in, top=1.00in, bottom=1.00in]{geometry}
\author{Tommaso Severini}
\title{Letteratura italiana - Boccaccio}
\begin{document}
	\maketitle

	\section{Vita}

	\subsection{Infanzia}

	Giovanni Boccaccio nasce a Firneze nel 1313 da Boccaccino di Chelino, mercante, con una donna di umili condizioni. Questo fatto influenzò la poetica di Boccaccio, che nelle sue opere afferma di essere figlio di una nobildonna parigina o dello stesso re di Frnacia. Riconosciuto solo nel 1320 dal padre, egli riceve una formazione classica.

	\subsection{Periodo napoletano}

	 Da questo momento in poi iniziano gli anni più belli della vita del poeta. Egli, infatti, si trasferisce con il padre nella corte dei Bardi, importante famiglia legata agli Angoini,e si dedica all'apprendimento della \textbf{tradizione stilnovista} grazie a Cino da Pistoia. In seguito, Boccaccio comincia a frequentare la corte angioina, ricca di libri e testi, dove studia la \textbf{letteratura cortese e romanza}. Scruve alcune opere minori tra cui il \textit{Teseida, il Filocolo, il Filostrato} e la \textit{Caccia di Diana}. Durnate questo periodo impara alcune nozioni grammaticali del greco(che successivamente lo spingerà a istituire una cattedra di greco all'università di Firenze).

	 \subsection{Periodo fiorentino}

	 Boccaccio è costretto a ritornare a Firenze a causa del padre. Ciò rende infelice Boccaccio. che si ritrova in un'ambiente troppo provinciale e ristretto, dove comunque cerca di accattivare la popolazione con alcune opere minori.
	 A causa della sua condizione economica, Boccaccio è costretto a cercare asilo tra le corti nobiliari emiliane. Infatti, tra il 1345 e il 1346 risiede a Ravenna dove viene in contatto con l'ambiente dove Dante aveva trascorso i suoi ultimi anni di vita.
	 Dopo di che, nel 1348 è costretto a tornare a Firenze, dove, durante l'epidemia di peste e in seguito alle morte di alcuni suoi cari, decide di scrivere la sua \textit{opus magna}: il \textbf{Decameron}, presumibilmente completato nel 1351.

	 \subsection{Periodo fiorentino-certaldese}

	 Questo periodo rappresenta sicuramente quello più buio nella vita di Boccaccio. A causa di una fallita congiura organizzata da dei suoi conoscenti perde gli incarichi al comune. Nonostante ciò, dopo qualche anno riassume i suoi ruoli al comune e dieviene ambasciatore nella corte papale ad Avignone. Durante i suoi ultimi anni di vita termina le sue opere in latino e rafforza la cultura umanistica a Firenze.

	 Egli curò un commento delle opere di Dante, sia in pubblico che in forma scritta.

	 \section{Stile e opere}

	 La produzione letteraria di Boccaccio può essere suddivisa in due fasi, separate dalla cmposizione del \textbf{Decameron}. A differenza di Petrarca, le sue opere tendono ad assumere un carattere più narrativo che lirico, caratterizzato dalla \textbf{scomparsa dell'io dell'autore}.

	 

\end{document}
