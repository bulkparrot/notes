\documentclass[10pt,a4paper]{article}
\usepackage[utf8]{inputenc}
\usepackage[T1]{fontenc}
\usepackage{amsmath}
\usepackage{amsfonts}
\usepackage{amssymb}
\usepackage{makeidx}
\usepackage{graphicx}
\usepackage[left=1.00in, right=1.00in, top=1.00in, bottom=1.00in]{geometry}
\author{Tommaso Severini}
\title{Letteratura italiana - Boccaccio}
\begin{document}
	\maketitle

	\section{Vita}

	\subsection{Infanzia}

	Giovanni Boccaccio nasce a Firneze nel 1313 da Boccaccino di Chelino, mercante, con una donna di umili condizioni. Questo fatto influenzò la poetica di Boccaccio, che nelle sue opere afferma di essere figlio di una nobildonna parigina o dello stesso re di Frnacia. Riconosciuto solo nel 1320 dal padre, egli riceve una formazione classica.

	\subsection{Periodo napoletano}

	 Da questo momento in poi iniziano gli anni più belli della vita del poeta. Egli, infatti, si trasferisce con il padre nella corte dei Bardi, importante famiglia legata agli Angoini,e si dedica all'apprendimento della \textbf{tradizione stilnovista} grazie a Cino da Pistoia. In seguito, Boccaccio comincia a frequentare la corte angioina, ricca di libri e testi, dove studia la \textbf{letteratura cortese e romanza}. Scruve alcune opere minori tra cui il \textit{Teseida, il Filocolo, il Filostrato} e la \textit{Caccia di Diana}. Durnate questo periodo impara alcune nozioni grammaticali del greco(che successivamente lo spingerà a istituire una cattedra di greco all'università di Firenze).

	 \subsection{Periodo fiorentino}

	 Boccaccio è costretto a ritornare a Firenze a causa del padre. Ciò rende infelice Boccaccio. che si ritrova in un'ambiente troppo provinciale e ristretto, dove comunque cerca di accattivare la popolazione con alcune opere minori.
	 A causa della sua condizione economica, Boccaccio è costretto a cercare asilo tra le corti nobiliari emiliane. Infatti, tra il 1345 e il 1346 risiede a Ravenna dove viene in contatto con l'ambiente dove Dante aveva trascorso i suoi ultimi anni di vita.
	 Dopo di che, nel 1348 è costretto a tornare a Firenze, dove, durante l'epidemia di peste e in seguito alle morte di alcuni suoi cari, decide di scrivere la sua \textit{opus magna}: il \textbf{Decameron}, presumibilmente completato nel 1351.

	 \subsection{Periodo fiorentino-certaldese}

	 Questo periodo rappresenta sicuramente quello più buio nella vita di Boccaccio. A causa di una fallita congiura organizzata da dei suoi conoscenti perde gli incarichi al comune. Nonostante ciò, dopo qualche anno riassume i suoi ruoli al comune e dieviene ambasciatore nella corte papale ad Avignone. Durante i suoi ultimi anni di vita termina le sue opere in latino e rafforza la cultura umanistica a Firenze.

	 Egli curò un commento delle opere di Dante, sia in pubblico che in forma scritta.

	 \section{Stile e opere}

	 La produzione letteraria di Boccaccio può essere suddivisa in due fasi, separate dalla cmposizione del \textbf{Decameron}. A differenza di Petrarca, le sue opere tendono ad assumere un carattere più narrativo che lirico, caratterizzato dalla \textbf{scomparsa dell'io dell'autore}.

	 \subsection{La cultura di Boccaccio}

	 Nonostante cerchi di scrivere e di esercitare la sua arte in quanto tale, Boccaccio cerca sempre di usare le sue opere per descrivere il contesto sociale in cui la vicenda prendeva luogo. Anche il fatto di essere nato a Firenze ma vissuto nelle corti signorili di Napoli gli permise di ottenere una \textbf{chiara visione della classe mercantile} insieme ad un esaustivo approfondimento delle opere cortesi, ma anche di quelle di \textbf{autori classici, come Virgilio, Ovidio e Stazio}. Oltre a ciò, una caratteristica fondamentale della produzione artistica di Boccaccio riguardo il suo \textbf{utilizzo dei più svariati generi narrativi}.

	 \section{Produzione artistica}

	 Nel primo periodo della sua vita, Boccaccio stende un'opera che modificherà fino alla fine della sua vita. Trattano principalmente il tema dell'amore, ma sono inclusi anche alcuni elementi autobiografici, entrambi raccontati attraverso la tradizione cortese.

	 Nel periodo fiornetino si dedica ad opere meno autobiografiche, ma più vicine alla relatà di Firenze e del pubblico comunale. Queste comprendono la "Commedia delle ninfe fiorentine", la "Ninfale fiesolano" e la "Elegia di Madonna Fiammetta".

	 Nell'ultimo periodo Boccaccio si dedica alla stesura di opere enciclopediche, ma anche di commenti delle opere dantesche, che espone anche in pubblico. Una opera importante da ricordare è sicuramente \textbf{Il Corbaccio}, che può essere descritta come un'invettiva contro la donna, che si conclude con l'avvicinamento dell'autore ad uno studio più ampio.

	 \section{Il decameron}

	 Opera composta tra il 1348 e il 1351, tratta della visone laica e terrena del mondo. Il suo titolo significa \textit{dieci giornate}, nel corso dei queli sono narrate 100 novelli da 10 giovani, 7 ragazze e 3 ragazzi. La \textbf{cornice} dell'opera è rappresentata dal contesto in cui essa prende luogo, ovvero la peste di Firenze che infuria nella città nel 1348.

	 Essa si presenta come un'opera rivoluzionaria, non più costituita da un lungo racconto ma da \textbf{novelle}, ma anche per la sua \textbf{pluralità dei temi}.

		Le sue finalità sono raccolte già nel sottotitolo dell'opera stessa: "cognominato prencipe Galeotto". Ciò allude al personaggio di Galeeotto appartenente alle storie di Lancillotto e Ginevra, in cui egli svolge un ruolo da consigliere e consolatore delle anime afflitte dall'amore; chiaro riferimento al ruolo consolatore dell'opera nei confronti delle donne che soffrono per amore a cui l'opera è dedicata. \textbf{Per la prima volta la condizione della donna è messa al primo posto}.

		Il ruolo della cornice, ripreso da opere classiche ma anche appartenenti alla letteratura orientale, svolge un ruolo fondamentale dal punto di vista metaforico. Infatti, la fuga dei ragazzi è una metafora che rappresenta la lotta per il diritto alla vita e la villa in cui si rifugiano diviene un locus amoenus in cui si può convivere civilmente.
		Oltre a ciò, la cornice permette di strutturare l'intera opera seconda un rigore logico, che traforma il progredire dei giorni in un percorso ascensionale, che parte nel modo deo ,ercanti e degli usurai (da Ser Ciappelletto) e si conclude nel mondo feudale (novella di Griselda). Ciò permette di visualizzare chiaramente \textbf{vizi e virtù degli uomini vissuti al tempo di Boccaccio}, ma senza pregiudizi da parte dell'autore.


\end{document}
