\documentclass[10pt,a4paper]{article}
\usepackage[utf8]{inputenc}
\usepackage[T1]{fontenc}
\usepackage{amsmath}
\usepackage{amsfonts}
\usepackage{amssymb}
\usepackage{makeidx}
\usepackage{graphicx}

\usepackage{hyperref}

\usepackage{xcolor}     % for colour
\usepackage{mdframed}   % for framing

\usepackage[left=1.00in, right=1.00in, top=1.00in, bottom=1.00in]{geometry}
\author{Tommaso Severini}
\title{Letteratura italiana - Divina Commedia - Canti VI XIII e XXVI}
\begin{document}
	\maketitle
	
	\tableofcontents
	
	\newmdtheoremenv[%
	linecolor=gray,leftmargin=60,%
	rightmargin=40,
	backgroundcolor=gray!40,%
	innertopmargin=5pt,%
	font=\ttfamily]{estratto}{Extract}[section]
	
	\section{Canto VI}
	
	\subsection{Introduzione generale}
	
	Dopo essersi risvegliato a seguito dell'incontro con Paolo e Francesca, Dante si accorge di essersi ritrovato nel Cerchio III, quello dei golosi. Questo girone infernale è sorvegliato, ma, a differenza dei precedenti, non troviamo una figura umana che rappresenta uno degli impedimenti del percorso simbolico di Dante, bensì troviamo un figura bestiale, mossa solo dai suoi istinti animaleschi e completamente priva di ragione (tanto che Virgilio, per placare il guardiano, gli getta un pugno di terra in bocca): il cane a tre teste \textbf{Cerbero}. 
	
	\subsection{Pena e contrappasso}
	
	Il secondo peccato che Dante incontra nel suo viaggio ultraterreno è quello della gola, posto immediatamente dopo quello di lussuria. Come punizione per il loro insaziabile desiderio di cibo, essi sono puniti da una permanente e violente pioggia, grandine e neve, mentre si trovano sommersi in una fanghiglia e assordati dai latrati di Cerbero. \textbf{Il contrappasso è verificato sia per analogia che per contrasto}: per \textbf{analogia},la gola rende gli uomini simili ad animali, costretti a rotolarsi nel fango, mentre, per \textbf{contrasto}, i golosi in vita amarono ricercare cibi raffinati e ora sono costretti a nutrirsi di fango.
	
\end{document}