\documentclass[10pt,a4paper]{article}
\usepackage[utf8]{inputenc}
\usepackage[T1]{fontenc}
\usepackage{amsmath}
\usepackage{amsfonts}
\usepackage{amssymb}
\usepackage{makeidx}
\usepackage{graphicx}

\usepackage{hyperref}

\usepackage{xcolor}     % for colour
\usepackage{mdframed}   % for framing

\usepackage[left=1.00in, right=1.00in, top=1.00in, bottom=1.00in]{geometry}
\author{Tommaso Severini}
\title{Letteratura italiana - Divina Commedia - Canti VI XIII e XXVI}
\begin{document}
	\maketitle
	
	\tableofcontents
	
	\newmdtheoremenv[%
	linecolor=gray,leftmargin=60,%
	rightmargin=40,
	backgroundcolor=gray!40,%
	innertopmargin=5pt,%
	font=\ttfamily]{estratto}{Extract}[section]
	
	\section{Canto VI}
	
	\subsection{Introduzione generale}
	
	Dopo essersi risvegliato a seguito dell'incontro con Paolo e Francesca, Dante si accorge di essersi ritrovato nel Cerchio III, quello dei golosi. Questo girone infernale è sorvegliato, ma, a differenza dei precedenti, non troviamo una figura umana che rappresenta uno degli impedimenti del percorso simbolico di Dante, bensì troviamo un figura bestiale, mossa solo dai suoi istinti animaleschi e completamente priva di ragione (tanto che Virgilio, per placare il guardiano, gli getta un pugno di terra in bocca): il cane a tre teste \textbf{Cerbero}. Ci troviamo al venerdì 8 aprile 1300.
	
	\subsection{Pena e contrappasso}
	
	Il secondo peccato che Dante incontra nel suo viaggio ultraterreno è quello della gola, posto immediatamente dopo quello di lussuria. Come punizione per il loro insaziabile desiderio di cibo, essi sono puniti da una permanente e violente pioggia, grandine e neve, mentre si trovano sommersi in una fanghiglia e assordati dai latrati di Cerbero. 
	
	\textbf{Il contrappasso è verificato sia per analogia che per contrasto}: per \textbf{analogia},la gola rende gli uomini simili ad animali, costretti a rotolarsi nel fango, mentre, per \textbf{contrasto}, i golosi in vita amarono ricercare cibi raffinati e ora sono costretti a nutrirsi di fango. Questa animalità rispecchia quella del peccato di gola, che presenta caratteri bestiali e, al contempo, riprende tratti pagani e filosofici. Infatti, la severa punizione di Dante (nessun'altra pena è così "spiacente") deriva sia dalle parole di san Paolo, che afferma come \textbf{il dio dei pagani sia il ventre}, ma anche da quelle di Aristotele, che, come anche Sallustio, condanna coloro che \textbf{"obbediscono al ventre"}.
	
	La pioggia che colpisce i dannati di questo girone è descritta in modo molto specifico, rendendola uno degli aspetti simbolici del canto. Infatti, questa pena presenta sia una connotazione psicologica che morale: \textbf{"etterna"} indica l'eternità e l'irrevocabilità della pena, \textbf{"fredda"} indica il raggelamento dell'anima dovuto all'egoismo di questo peccato e \textbf{"greve"} indica la materialità del peccato di gola.
	
	\subsection{Personaggi}
	
	\subsubsection{Cerbero}
	
	Personaggio della mitologia classica, figlio di Tifeo ed Echidna, già presente nell'Ade pagano con l'aspetto di cane a tre teste quale custode dell'ingresso degli Inferi (Ercole, in una delle sue fatiche, lo trascinò fuori dall'Ade tirandolo per una catena). Il mostro è descritto da Virgilio nel libro VI dell'Eneide, mentre si oppone alla discesa agli Inferi  di Enea ed è ammansito dalla Sibilla che gli getta un'offa (focaccia) di miele intrisa di erbe soporifere. Cerbero, che in Virgilio ha dei serpenti attorcigliati al collo, la afferra con fame rabbiosa ed è forse il motivo per cui nella tradizione medievale era talvolta interpretato come immagine del peccato di gola.
	Dante, infatti, lo pone a custodia del III Cerchio (golosi), dove è strumento di punizione in quanto graffia e scuoia gli spiriti con i suoi artigli (Inf., Canto VI). Il mostro è descritto con occhi rossi, i peli del muso sporchi e neri, il ventre largo e le zampe artigliate; emette latrati che assordano i dannati e ciò acuisce il loro tormento. Appena vede i due poeti si avventa contro di loro, ma Virgilio gli getta in gola una manciata di terra che placa la sua fame (in modo quindi analogo all'episodio dell'Eneide, salvo che qui la rappresentazione del mostro è chiaramente demoniaca). 
	
	\subsubsection{Ciacco}
	
	Poco sappiamo di lui, a parte le notizie fornite da Dante e da Boccaccio nel Decameron (IX, 8), dove lo definisce un «uomo ghiottissimo quanto alcun altro fosse giammai... per altro assai costumato e tutto pieno di belli e piacevoli motti». Il nome poteva forse essere un soprannome spregiativo col senso di «porco», ma potrebbe essere anche un nome proprio. Probabilmente era un parassita che a Firenze veniva invitato ai banchetti per allietare i commensali con le sue facezie, quindi doveva essere ben noto ai lettori contemporanei della Commedia.
	
	Nel Canto VI dell'Inferno Ciacco riconosce Dante come fiorentino e gli chiede se lo riconosce, cosa impossibile dato il suo aspetto stravolto. Poi si presenta e Dante gli pone tre domande sul destino politico di Firenze: cosa succederà alle fazioni in lotta, se vi sono cittadini giusti, quali sono le cause della discordia. Ciacco risponde profetizzando la vittoria dei Neri, dicendo che i giusti sono pochissimi e indicando le cause delle divisioni in superbia, invidia e avarizia.
	
	Dante chiede inoltre notizie sul destino escatologico di altri fiorentini illustri del passato e Ciacco risponde che sono tutti dannati nel profondo dell'Inferno. Dopo aver pregato Dante di ricordarlo ai vivi una volta tornato sulla Terra, tace e torna a sdraiarsi con gli altri dannati, nel fango del III Cerchio.
	
	\subsection{Tematiche}
	
	\subsubsection{Tema politico}
	
	Questo canto assume sicuramente una connotazione politica, che Dante gli conferisce evidenzaindo la \textbf{lotta tra gueli Bianchi e Neri} e le sue \textbf{cause scatenanti: invidia, superbia e avarizia} (concetti alla base della società mercantile che si stava formando agli inizi del '300). In particolare, Dante si accanisce contro il peccato di avarizia, che egli cosnidera essere il peggiore tra i peccati capitali. Il poeta, infatti, decide di enfatizzare ciò iniziando il canto con l'incontro con cerbero, che rappresenta l'avarizia del peccato di gola, e di chiuderlo con la visione di Pluto, guardiano del cerchio degli avari.
	
	\subsubsection{Tema morale}
	
	L'evocazione delle cinque grandi figure fiorentine del passato\textbf{(Farinata degli Uberti, Tegghiaio Aldobrandi, Iacopo Rusticucci, Arrigo e Mosca dei Lamberti)} serve da prete4sto per introdurre uno dei temi ricorrenti dell'opera: la \textbf{magnanimità delle gesta}. Infatti, dal punto di vista di Dante, anche se un uomo compie buone azioni solo dal punto di vista civico, ciò non garantisce che si ritroverà in paradiso; Dante giudica i dannati secondo le virtù cardinali: carità, umiltà...
	
	\subsubsection{Tema dottrinale}
	
	Come ultimo aspetto, Dante enfatizza la condizione delle anime dopo la morte e, in particoalre, dopo il giudizio universale. Con ciò, il Sommo Poeta ci vuole ricordare l'ineluttabilità della morte e del giudizio universale (quasi a ricordare un "memento mori") e l'insignificanza della vita in confronto alla vita eterna.
	
\end{document}