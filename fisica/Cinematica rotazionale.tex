\documentclass[10pt,a4paper]{article}
\usepackage[utf8]{inputenc}
\usepackage[T1]{fontenc}
\usepackage{amsmath}
\usepackage{amsfonts}
\usepackage{amssymb}
\usepackage{makeidx}
\usepackage{graphicx}
\usepackage{mdframed}
\usepackage{xcolor}
\usepackage[left=1.00in, right=1.00in, top=1.00in, bottom=1.00in]{geometry}
\author{Tommaso Severini}
\title{Fisica - Cinetica rotazionale}
\begin{document}
	\maketitle
	
		%% mdframed definizione
	\mdfdefinestyle{theoremstyle}{%
		linecolor=orange,linewidth=2pt,%
		frametitlerule=true,%
		frametitlebackgroundcolor=gray!20,
		innertopmargin=\topskip,
	}
	\mdtheorem[style=theoremstyle]{definition}{Definition}
	
	\section{Energia cinetica rotazionale}
	
	Sappiamo che qualsiasi corpo che si muove di moto traslatorio possiede un'energia cinetica. Ad esempio, un corpo di massa \textit{m} e in movimento a una velocità costante \textit{v}, la sua energia cinetica sarà data dall'equazione $\frac{1}{2}mv^2$.
	Nel caso di un corpo in rotazone, però, ogni particella che fa parte del corpo si muove a velocità tangenziale diversa a seconda della sia distanza dal raggio. Nonostante ciò, la velocità angolare è la stessa per tutti i corpi e, provando a sostituire la velocità tangenziale \textit{v} con la sua espressione in funzione della velocità angolare $\omega$, otteniamo la seguente formula
	
	\begin{equation}
		K = \frac{1}{2}mv^2 = \frac{1}{2}m (\omega r)^2 = \frac{1}{2}mr^2\omega^2
	\end{equation} 

	Notiamo che questa forma differisce di poco dalla forma classica dell'energia cinetica traslatoria, con la sola differenza della velocità angolare al posto di quella tangenziale e la sostituzione della massa del corpo con la grandezza $mr^2$, che definisce la tendenza di un corpo a subire cambiamenti nel suo moto. Questa "inerzia di movimento" prende il nome di \textbf{momento di inerzia I}. Infatti, più I è grande maggior sarà la resistenza del corpo alla variazione del suo moto.  
	
	\begin{definition}[Energia cinetica di rotazione]
		$\frac{1}{2}I \omega^2$ dove I è il momento di inerzia del corpo, mentre $\omega$ è la velocità angolare del corpo.
	\end{definition}

	\section{Momento di inerzia}
	
	Considerando un corpo non puntiforme, possiamo calcolare il suo momento di inerzia dividendolo in masse più piccole e trovando il momento di inerzia di ognuno per poi sommarli. In formule:
	
	\begin{definition}[Momento di inerzia]
		$I = \sum m_i r_i^2$
	\end{definition}
	
\end{document}