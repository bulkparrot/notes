\documentclass[10pt,a4paper]{article}
\usepackage[utf8]{inputenc}
\usepackage[T1]{fontenc}
\usepackage{amsmath}
\usepackage{amsfonts}
\usepackage{amssymb}
\usepackage{makeidx}
\usepackage{graphicx}
\usepackage[left=1.00in, right=1.00in, top=1.00in, bottom=1.00in]{geometry}
\author{Tommaso Severini}
\title{Operone lac}
\begin{document}
	\maketitle
	
	In biologia, si definisce operone un insieme di geni regolati in modo strettamente correlato. Il più famoso tra questi è \textbf{l'operone lac}, noto per essere stato il primo esempio scoperto di controllo dell'espressione genica nei procarioti. La sua scoperta, da parte di  François Jacob e Jacques Monod, li ha portato a vincere il premio Nobel per la medicina nel 1965. Questo operone contiene i tre geni necessari al batterio Escherichia Coli per metabolizzare il lattosio, zucchero contenuto nel latte: lacZ, lacY e lacA. Questi geni codificano rispettivamente le proteine $\beta-galattosidasi$, che catalizza l'idrolisi dei residui terminali di $\beta-D-galattosio$ nelle molecole di lattosio scindendo lo zucchero i monosaccaridi, la $\beta-galattoside permeasi$, che permette alle molecole di lattosio di penetrare attraverso la membrana cellulare, e la galattoside acetiltransferasi, che fornisce un gruppo acetile al $\beta-D-galattoside$ creato durante la digestione del lattosio.
\end{document}