\documentclass[10pt,a4paper]{article}
\usepackage[utf8]{inputenc}
\usepackage[T1]{fontenc}
\usepackage{amsmath}
\usepackage{amsfonts}
\usepackage{amssymb}
\usepackage{makeidx}
\usepackage{graphicx}
\usepackage[left=1.00in, right=1.00in, top=1.00in, bottom=1.00in]{geometry}
\author{Tommaso Severini}
\title{Biologia - Regolazione genica negli eucarioti}
\begin{document}
	\maketitle
	
	Nella riproduzione delle cellule eucariote, le cellule si \textbf{differenziano} a seconda della posizione grazie a raffinati meccanismi dell'espressione genica. Questa differenziazione porta alla formazione di tessuti molto diversi tra loro.
	
	\section{Spiralizzazione}
	
	Il ripiegamento del DNA svolge un ruolo cruciale nell'espressione genica. Osserviamo la struttura della cromatina: il doppio filamento di DNA forma un complesso istonico avvolgendosi attorno a delle particolari proteine dette \textbf{istoni}. Il luoghi in cui gli istoni si raggruppano sono detti \textbf{nucleosomi}. Al momento della replicazione, il complesso istonico assume una conformazione ancora più spiralizzata, detta \textbf{cromosoma}. 
	
\end{document}