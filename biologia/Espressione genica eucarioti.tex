\documentclass[10pt,a4paper]{article}
\usepackage[utf8]{inputenc}
\usepackage[T1]{fontenc}
\usepackage{amsmath}
\usepackage{amsfonts}
\usepackage{amssymb}
\usepackage{makeidx}
\usepackage{graphicx}
\usepackage[left=1.00in, right=1.00in, top=1.00in, bottom=1.00in]{geometry}
\author{Tommaso Severini}
\title{Biologia - Regolazione genica negli eucarioti}
\begin{document}
	\maketitle
	
	Nella riproduzione delle cellule eucariote, le cellule si \textbf{differenziano} a seconda della posizione grazie a raffinati meccanismi dell'espressione genica. Questa differenziazione porta alla formazione di tessuti molto diversi tra loro.
	
	\section{Spiralizzazione}
	
	Il ripiegamento del DNA svolge un ruolo cruciale nell'espressione genica. Osserviamo la struttura della cromatina: il doppio filamento di DNA forma un complesso istonico avvolgendosi attorno a delle particolari proteine dette \textbf{istoni}. Il luoghi in cui gli istoni si raggruppano sono detti \textbf{nucleosomi}. Al momento della replicazione, il complesso istonico assume una conformazione ancora più spiralizzata, detta \textbf{cromosoma}.
	
	\section{Elementi regolatori}
	
	Negli eucarioti, come nei batteri, una sequenza di DNA che indica alla RNA polimerasi dove iniziare la trascrizione di un gene è detto \textbf{promotore}. La trascrizione a partire dai pr
	
	Transcription from a particular promoter
	is controlled by DNA-binding proteins, termed transcription
	factors, that are equivalent to bacterial repressors
	and activators. However, the DNA control elements in eukaryotic
	genomes that bind transcription factors often are located
	much farther from the promoter they regulate than is
	the case in prokaryotic genomes. In some cases, transcription
	factors that regulate expression of protein-coding genes in
	higher eukaryotes bind at regulatory sites tens of thousands of
	base pairs either upstream (opposite to the direction of transcription)
	or downstream (in the same direction as transcription)
	from the promoter. As a result of this arrangement,
	transcription from a single promoter may be regulated by
	binding of multiple transcription factors to alternative control
	elements, permitting complex control of gene expression.
	For example, alternative transcription-control elements
	regulate expression of the mammalian gene that encodes
	transthyretin (TTR), which transports thyroid hormone in
	blood and the cerebrospinal fluid that surrounds the brain 
	
\end{document}