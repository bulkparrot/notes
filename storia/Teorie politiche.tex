\documentclass[10pt,a4paper]{article}
\usepackage[utf8]{inputenc}
\usepackage[T1]{fontenc}
\usepackage{amsmath}
\usepackage{amsfonts}
\usepackage{amssymb}
\usepackage{makeidx}
\usepackage{graphicx}
\usepackage[left=1.00in, right=1.00in, top=1.00in, bottom=1.00in]{geometry}
\author{Tommaso Severini}
\title{Storia - Principali teorie politiche}
\begin{document}
	\maketitle
	
	\section*{Termini fondamentali}
	\begin{itemize}
		\item Stato di natura: Ipotetica condizione in cui gli uomini non sono associati tra loro enè disciplinati da leggi. Rappresenta un potenziale sfondo di violenze che si nasconde al di sotto della società civile
		\item Contrattualismo: Origine dello Stato da un contratto sociale
		\item Giusnaturalismo: Tutti gli uomini possiedono diritti inalienabili dalla nascita (VITA, LIBERTA', PROPRIETA')
		\item Società civile: Unione di uomini in comunità che vivono pacificamente.
		\item Stato: organizzazione del potere mirata al mantenimento della pace.
	\end{itemize}
\end{document}