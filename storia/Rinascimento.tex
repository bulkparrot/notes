\documentclass[11pt]{report}
\usepackage[margin=1in]{geometry}
\usepackage{amsfonts,amsmath,amssymb}
\usepackage[italian]{babel}
\usepackage{fancyhdr}
\usepackage{graphicx}
\usepackage{pgfplots}
\usepackage{wrapfig}
\pgfplotsset{width=10cm,compat=1.9}

\title{La colonizzazione delle Americhe}
\author{Tommaso Severini}

\pagestyle{fancy}
\fancyhead{}
\fancyfoot{}
\fancyhead[L]{LA COLONIZZAZIONE}
\fancyhead[R]{Tommaso Severini}
\fancyfoot[C]{\thepage}

\parindent 0ex

\begin{document}
	
	
	\begin{titlepage}
		\begin{center}
			\Large\textbf{ III B Storia}
			
			\Large\textbf{Lavoro di approfondimento}
			\vfill
			\rule{400pt}{0.1pt}\\
			
			\huge\textbf{La scoperta di un nuovo mondo}
			\rule{400pt}{0.1pt}
			\vfill
			{\small Tommaso Severini\\
			21 aprile 2021}
		\end{center}
	\end{titlepage}	
	
	\begin{abstract}
		Il processo di colonizzazione delle Americhe comportò la scoperta, l'esplorazione, la conquista e l'occupazione di svariati territori nel "Nuovo Mondo" da parte di varie nazioni europee, come l'Inghilterra, la Spagna, la Francia e il Portogallo. 
		
		Benché lo scopo fondamentale fosse quello di espandere i propri affari commerciali, civilizzando e propagando la fede cattolica in queste nuove terre, il processo di colonizzazione produsse una sistematica distruzione delle culture delle popolazioni locali.\\
		
		Nonostante questo fenomeno riguardi particolarmente i paesi occidentali, intendo cominciare a descrive il trascorrere degli eventi a partire dalla caduta della capitale dell'impero bizantino, Costantinopoli. Questo evento ebbe una grande influenza sul rinascimento,epoca storica durante la quale ebbe inizio il processo di colonizzazione.
	\end{abstract}
	
	\chapter{La caduta dell'impero romano d'Oriente}
	
	Ci troviamo nel periodo storico che va dal 1202 al 1204. Papa Innocenzo III fa finalmente cominciare la sua spedizione per riconquistare Gerusalemme, controllata dai Musulmani, attraverso la conquista del Sultanato Egiziano di Ayyubid, il più potente stato islamico dell'epoca. 
	
	Nonostante questi ambiziosi obiettivi, a causa di una sequenza di eventi economici e politici, l'esercito Crociato decide di rinunciare alla guerra contro il Sultanato. Durante il loro viaggio verso la Terra Santa, l'esercito Crociato, a causa di motivi economici, assedia la città di Zara, citta posizionata sulla costa adriatica dell'attuale Croazia. Questo evento scatena l'ira di Innocenzo III, che decide di scomunicare l'esercito da lui stesso inviato. Ciò, ovviamente, porta i Crociati ad accettare l'aiuto del principe bizantino Alexios Angelos, intento a riconquistare il potere dopo la deposizione di suo padre dal potere. Nell'agosto del 1203 l'esercito Crociato aiuterà Alexios nell'assedio di Costantinopoli, che porterà all'inconorazione di quest'ultimo come co-imperatore. Purtroppo, però, il regno di Alexios non dura a lungo, deposto nel gennaio dell'anno seguente da un'insurrezione popolare. 
	
	Capendo che non otterranno il supporto di Alexios come speravano, i Crociati decidono di conquistare la città, impossessandosi dell'enorme ricchezza della città. Questa conquista comporta una grande crisi per l'impero romano d'Oriente, che viene diviso in 3 regni principali: l'impero di Nicea, il Despotato dell'Epiro e l'Impero di Trebisonda. Nonostante ciò, anche i conquistatori cristiani riusciranno ad ottenere il controllo su una vasta porzione dell'impero bizantino, tanto da formare l'Impero Latino di Costantinopoli.
	
	\chapter{La scoperta del Nuovo Mondo}
	
	\paragraph*{La minaccia dell'Impero Ottomano}
	
	La caduta di Costantinopoli e la seguente instaurazionedell'impero turco ebbe delle rimarcabili conseguenze sull'economia degli stati europei. Innanzitutto, le piazzeforti instaurate da alcuni stati, come la Serenissima Repubblica di Venezia e la Repubblica di Genova, sulle coste orientali del Mediterraneo risultano ormai impraticabili. Oltretutto, il controllo di territori come la penisola anatolica da parte degli Ottomani impediva ai paesi europei di mantenere facilmente rapporti stabili con l'Oriente.
	
	Tutto ciò costrinse le principali potenze mercantili a riorganizzarsi per poter negoziare con le Indie e la Cina. 
	
	\paragraph*{Le prime esplorazioni portoghesi}
	
	Uno dei primi paesi a prendere provvedimenti per ristabilire i suoi rapporti con i paesi orientali fu il Regno di Portogallo. Infatti, poichè il governo portoghese stava cercando da tempo di impossessarsi delle miniere d'oro africane, decise di finanziare quest'opera per cercare anche nuove rotte attorno al continente africano per giungere in Oriente.
	
	\begin{wrapfigure}{r}{2.5in}
		\includegraphics[width=2.5in]{"vasco da gama"}
		\caption{{\small Prima esplorazione di Vasco da Gama (1497-1499)}}
	\end{wrapfigure} 
	
	Le prime vere e proprie esplorazioni iniziarono nel 1419 lungo la costa occidentale africana e con il supporto del principe Enrico il Navigatore. Uno dei più grandi obiettivi raggiunti dai portoghesi che oggi ancora ricordiamo è sicuramente il raggiungimento dell'estremità più meridionale dell'Africa, oggi conosciuto come Capo di Buona Speranza, da parte di Bartolomeo Diaz nel 1488. Successivi progressi ottenuti dallo sforzo della marina portoghese furono la scoperta della prima rotta commerciale verso l'India, più precisamente Calicut, da parte di Vasco da Gama nel 1498 e l'esplorazione dei territori nipponici nel 1542, 44 anni dopo il loro arrivo in India.
	
	Nonostante ciò, l'egemonia portoghese si rivela poco duratura. In parte, il governo portoghese ha sofferto varie complicazioni già sulla terra ferma. La rivalità tra Spagna e Portogallo diviene più intensa dopo l'unione delle due corone spagnole, Regno di Aragona e Regno di Castiglia, nel 1580. D'altra parte, si forma una certa rivalità anche con gli imperi Asiatici, le cui temporanee debolezze erano state scruttate dai navigatori occidentali. In Giappone, ad esempio, dopo che lo shogunate Tokugawa riuscì a ristabilire il suo potere, egli fa espellere gli Iberici e nel 1639 abolì il Cristianesimo, poichè considerato un pericolo alla stabilità dello stato. \cite{o2005philip}
	
	\section*{La scoperta delle Americhe}
	
	\paragraph*{Il viaggio di Colombo}
	
	Uno dei paesi che ha portato ad una scoperta fondamentale non corso dell'Età delle scoperte fu sicuramente la Spagna. Infatti, dopo aver terminato la \textit{reconquista}, progressiva espansione dei regni cristiani del nord della penisola iberica nei territori del califfato islamico di al-Andalus, questa nazione ebbe libero accesso alle cosste atlantiche, ma le isole atlantiche e la costa africana risultava sbarrata dal regno portoghese. Per questo motivo la Spagna si ritrova costretta a espandersi verso Occidente attraverso l'oceano atlantico.\\
	
	Questi eventi portarono i re cattolici, Isabella di Castiglia e Ferdinando d'Aragona, ad accettare la proposta di un navigatore genovese conosciuto in Spagna come \textit{Cristobal Colon}, ma nel resto del mondo come Cristoforo Colombo. Quest'ultimo propose ai sovrani spagnoli una rotta alternativa verso le Indie, però ricca di errori di calcolo, soprattutto dovuti alla sbagliata conversione tra le unità di misura. Furono proprio questi errori a portare Colombo a pensare che la circonferenza della Terra fosse molto minore di quanto lo è in realtà.
	
	Le famose tre caravelle di Colombo, la \textit{Niña} insieme alla \textit{Pinta} e alla \textit{Santa Maria}, salparono dal porto di Palos, Spagna, il 3 agosto 1492 e approdarono in territorio americano il 12 ottobre 1492, più precisamente a San Salvador, nelle odierne isole Bahamas. Da quel momento, quella parte di mondo divenne terra spagnola, almeno secondo la volontà di Colombo stesso che ne prese possesso, senza nessuna consultazione con gli abitanti, a nome dei reali di Spagna. Seguirono a questo altri tre viaggi nel periodo tra il 1493 e il 1500, in cui Colombo continuò l'esplorazione dei Caraibi, raggiungendo a sud le foci dell'Orinoco e ad ovest Panama.
	
	\paragraph*{Il trattato di Tordesillas}
	
	Le nuove scoperte e i nuovi territori occupati da Colombo non tardarono a creare una disputa tra le altre potenze marittime, come il Portogallo. Nel 1481, dopo essere salito al soglio pontificio, Papa Sisto IV aveva garantito tutte le terre a sud delle isole Canarie al Regno di Portogallo. Nonostante ciò, nel 1493, il Papa spagnolo Alessandro VI pubblica la bolla \textit{Inter Caetera} che da tutte le terre a ovest di un meridiano a ovest delle isole di Capo Verde alla Spagna e tutte le terre ad est del suddetto meridiano al Portogallo. Ciò, ovviamente, non soddisfaceva il sovrano portoghese che a causa di questo trattato avrebbe perso tutte le sue colonie in Brasile; ciò spinge Giovanni II di Portogallo ad aprire delle trattative con i Re Cattolici per spostare il meridiano più ad ovest, sostenendo che il meridiano proposto dal Papa avrebbe limitato in controllo spagnolo sulle Indie.\cite{parry2010age} 
	
	\paragraph*{I \textit{conquistadores}}
	
	In occasione del trattato di Torsedillas il Papa incaricò ufficialmente la Spagna e il Portogallo di evangelizzare i territori scoperti. Ciò diede la possibilità agli Stati iberici di portare la "buona novella" si dal punto di vista religioso che dal punto di vista militare: nascono i \textit{conquistadores}.
	
	\bibliographystyle{naturemag}
	{\small \bibliography{nuovo_mondo.bib}}
	
\end{document}
