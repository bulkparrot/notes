\documentclass[11pt]{report}
\usepackage[margin=1in]{geometry}
\usepackage{amsfonts,amsmath,amssymb}
\usepackage[italian]{babel}
\usepackage{fancyhdr}
\usepackage{graphicx}
\usepackage{pgfplots}
\usepackage{wrapfig}
\pgfplotsset{width=10cm,compat=1.9}

\title{La colonizzazione delle Americhe}
\author{Tommaso Severini}

\pagestyle{fancy}
\fancyhead{}
\fancyfoot{}
\fancyhead[L]{LA COLONIZZAZIONE}
\fancyhead[R]{Tommaso Severini}
\fancyfoot[C]{\thepage}



\parindent 0ex

\begin{document}
	

	
	
	\begin{titlepage}
		\begin{center}
			\Large\textbf{ III B Storia}
			
			\Large\textbf{Lavoro di approfondimento}
			\vfill
			\rule{400pt}{0.1pt}\\
			
			\huge\textbf{L'età delle Scoperte}
			\rule{400pt}{0.1pt}
			\vfill
			{\small Tommaso Severini\\
			21 aprile 2021}
		\end{center}
	\end{titlepage}	
	
	\begin{abstract}
		Il processo di colonizzazione delle Americhe comportò la scoperta, l'esplorazione, la conquista e l'occupazione di svariati territori nel "Nuovo Mondo" da parte di varie nazioni europee, come l'Inghilterra, la Spagna, la Francia e il Portogallo. 
		
		Benché lo scopo fondamentale fosse quello di espandere i propri affari commerciali, civilizzando e propagando la fede cattolica in queste nuove terre, il processo di colonizzazione produsse una sistematica distruzione delle culture delle popolazioni locali.\\
		
		Nonostante questo fenomeno riguardi particolarmente i paesi occidentali, intendo cominciare a descrive il trascorrere degli eventi a partire dalla caduta della capitale dell'impero bizantino, Costantinopoli. Questo evento ebbe una grande influenza sul rinascimento,epoca storica durante la quale ebbe inizio il processo di colonizzazione.
	\end{abstract}
	
	\chapter{La caduta dell'impero romano d'Oriente}
	
	Ci troviamo nel periodo storico che va dal 1202 al 1204. Papa Innocenzo III fa finalmente cominciare la sua spedizione per riconquistare Gerusalemme, controllata dai Musulmani, attraverso la conquista del Sultanato Egiziano di Ayyubid, il più potente stato islamico dell'epoca. 
	
	\paragraph*{La cancellazione della Crociata}Nonostante questi ambiziosi obiettivi, a causa di una sequenza di eventi economici e politici, l'esercito Crociato decide di rinunciare alla guerra contro il Sultanato. Durante il loro viaggio verso la Terra Santa, l'esercito Crociato, a causa di motivi economici, assedia la città di Zara, citta posizionata sulla costa adriatica dell'attuale Croazia. Questo evento scatena l'ira di Innocenzo III, che decide di scomunicare l'esercito da lui stesso inviato. Ciò, ovviamente, porta i Crociati ad accettare l'aiuto del principe bizantino Alexios Angelos, intento a riconquistare il potere dopo la deposizione di suo padre dal potere. Nell'agosto del 1203 l'esercito Crociato aiuterà Alexios nell'assedio di Costantinopoli, che porterà all'inconorazione di quest'ultimo come co-imperatore. Purtroppo, però, il regno di Alexios non dura a lungo, deposto nel gennaio dell'anno seguente da un'insurrezione popolare. 
	
	\begin{wrapfigure}{r}{3.5in}
		\includegraphics[width=3.5in]{"LatinEmpire2"}
		\caption{{\small La divisione dei territori dopo la Quarta Crociata (1204)}}
	\end{wrapfigure} 
	
	Capendo che non otterranno il supporto di Alexios, i Crociati decidono di conquistare la città, impossessandosi della sua enorme ricchezza. Questa conquista comporta una grande crisi per l'impero romano d'Oriente, che viene diviso in 3 regni principali: l'impero di Nicea, il Despotato dell'Epiro e l'Impero di Trebisonda. Nonostante ciò, anche i conquistatori cristiani riusciranno ad ottenere il controllo su una vasta porzione dell'impero bizantino, tanto da formare l'Impero Latino di Costantinopoli.
	
	\paragraph*{La "rinascita" bizantina}É l'imperatore della Nicea, Michele VIII Paleologo, che riconquista Costantinopoli nel 1261. L'impero così riformato, però, presentava enormi problematiche: la sua economia stava diventando sempre più debole a causa dei commerci Veneziani e Genovesi che avevano il pieno controllo sulle principali rotte marittime. Oltre a ciò, l'impero si ritrova confinato dai suoi rivali, che stanno diventando sempre più forti e che sono intenti a conquistare uno dei porti commerciali più importanti del mediterraneo; a nord erano situate le popolazioni Slavo-Balcaniche e nella penisola anatolica si trovava il minaccioso impero Ottomano.
	
	Già a partire dalla metà del XIV secolo, il territorio greco risultava occupato dai Serbi, poi sconfitti e allontanati dal potere turco. Dal 1354 i Turchi Ottomani giunsero in Europa, entrando in un particolare rapporto diplomatico con ciò che poteva essere ormai considerata la città-stato di Costantinopoli.\cite{o2005philip}
	
	La conquista turca di Gallipoli nel 1354 presagiva già la
	grande invasione ottomana dell'Europa. Nel 1365
	Adrianopoli era diventata la capitale ottomana, rinominata Edirne.
	Avanzando in Serbia e culminando nella battaglia del Kosovo
	Polje nel 1389, l'impero pone fine all'espansione serba. Allo stesso
	tempo gli Ottomani consolidano il loro controllo dell'Asia Minore,
	e nasce una marina ottomana, che solca le acque del Mediterraneo, l'Egeo e l'Adriatico; Molti dei suoi ammiragli erano europei rinnegati e non molto esperti. Il primo assedio di Costantinopoli avviene nel 1391, ma è prontamente fermato dai Mongoli di Tamerlano, che sconfiggono l'esercito turco nella dura battaglia di Ankara.
	
	\paragraph*{La caduta di Costantinopoli}
	
	Era ovvio che Costantinopoli stesse vivendo sul filo del rasoio. Nonostante ciò, essa rimaneva il centro della sapienza e della cultura, oggi ancora visibile nei resti della città di Mistra, nel Peloponneso.
	
	\begin{wrapfigure}{r}{3in}
		\includegraphics[width=3in]{"costantinopoli"}
		\caption{{\small Costantinopoli nel periodo bizantino}}
	\end{wrapfigure} 
	
	Nel frattempo, i Turchi Ottomani stavano costruendo la loro macchina da guerra. A partire dal XIV secolo, ci furono vittorie ottomane solo grazie all'aiuto dei Balcani e di altri mercenari. Questa volta, c'è un reclutamento di truppe molto più ingente, intenzionato a produrre il maggior numero possibile delle migliori truppe scelte tra le fila ottomane, i Giannizzeri.\\
	
	Pronto per un secondo tentativo di assedio, l'esercito ottomano, guidato da Maometto II, circonda la città di Costantinopoli nel 1451. Provvisti degli migliori esempi di artiglieria bellica dell'epoca, le mitologiche mura teodosiane, che difendevano la città da più di un millennio, crollano sotto il fuoco turco nel giorno 29 marzo 1453.\\
	
	Questa data rappresenta l'ultimo giorno del glorioso Impero Romano, ma anche la nascita di un Impero Ottomano maturo, pronto a espandersi fino al XVII secolo e a rimanere in vita fino al 1922. 
	
	\chapter{La scoperta del Nuovo Mondo}
	
	\paragraph*{La minaccia dell'Impero Ottomano}
	
	La caduta di Costantinopoli e la seguente instaurazione dell'impero turco ebbe delle rimarcabili conseguenze sull'economia degli stati europei. Innanzitutto, le piazzeforti instaurate da alcuni stati, come la Serenissima Repubblica di Venezia e la Repubblica di Genova, sulle coste orientali del Mediterraneo risultano ormai impraticabili. Oltretutto, il controllo di territori come la penisola anatolica da parte degli Ottomani impediva ai paesi europei di mantenere facilmente rapporti stabili con l'Oriente.
	
	Tutto ciò costringe le principali potenze mercantili a riorganizzarsi per poter negoziare con l'India e la Cina. 
	
	\paragraph*{Le prime esplorazioni portoghesi}
	
	Uno dei primi paesi a prendere provvedimenti per ristabilire i suoi rapporti con i paesi orientali fu il Regno di Portogallo. Infatti, poichè il governo portoghese stava cercando da tempo di impossessarsi delle miniere d'oro africane, esso decide di finanziare quest'opera per cercare anche nuove rotte attorno al continente africano per giungere in Oriente.
	
	\begin{wrapfigure}{r}{2.5in}
		\includegraphics[width=2.5in]{"vasco da gama"}
		\caption{{\small Prima esplorazione di Vasco da Gama (1497-1499)}}
	\end{wrapfigure} 
	
	Le prime vere e proprie esplorazioni iniziarono nel 1419 lungo la costa occidentale africana e con il supporto del principe Enrico il Navigatore. Uno dei più grandi obiettivi raggiunti dai portoghesi che oggi ancora ricordiamo è sicuramente il raggiungimento dell'estremità più meridionale dell'Africa, oggi conosciuto come Capo di Buona Speranza, da parte di Bartolomeo Diaz nel 1488. Successivi progressi ottenuti dallo sforzo della marina portoghese furono la scoperta della prima rotta commerciale verso l'India, più precisamente Calicut, da parte di Vasco da Gama nel 1498 e l'esplorazione dei territori nipponici nel 1542, 44 anni dopo il loro arrivo in India.
	
	Nonostante ciò, l'egemonia portoghese si rivela poco duratura. In parte, il governo portoghese ha sofferto varie complicazioni già sulla terra ferma. La rivalità tra Spagna e Portogallo diviene più intensa dopo l'unione delle due corone spagnole, Regno di Aragona e Regno di Castiglia, nel 1580. D'altra parte, si forma una certa rivalità anche con gli imperi Asiatici, le cui temporanee debolezze erano state sfruttate dai navigatori occidentali. In Giappone, ad esempio, dopo che lo shogunate Tokugawa riusce a ristabilire il suo potere, egli fa espellere gli Iberici e nel 1639 abolisce il Cristianesimo, poichè considerato un pericolo alla stabilità dello stato. \cite{o2005philip}
	
	\section*{La scoperta delle Americhe}
	
	\paragraph*{Il viaggio di Colombo}
	
	Uno dei paesi che ha portato ad una scoperta fondamentale non corso dell'era moderna fu sicuramente la Spagna. Infatti, dopo aver terminato la \textit{reconquista}, progressiva espansione dei regni cristiani del nord della penisola iberica nei territori del califfato islamico di al-Andalus, questa nazione ebbe libero accesso alle coste atlantiche, ma le isole e la costa africana risultavano sbarrate dal regno portoghese. Per questo motivo la Spagna si ritrova costretta a espandersi verso Occidente attraverso l'Oceano Atlantico.\\
	
	\begin{wrapfigure}{r}{2.5in}
		\includegraphics[width=2.5in]{"colombo"}
		\caption{{\small Ritratto di Cristoforo Colombo, di Sebastiano del Piombo (1519)}}
	\end{wrapfigure} 
	
	Questi eventi portano i re cattolici, Isabella di Castiglia e Ferdinando d'Aragona, ad accettare la proposta di un navigatore genovese conosciuto in Spagna come \textit{Cristobal Colon}, ma nel resto del mondo come Cristoforo Colombo. Quest'ultimo propone ai sovrani spagnoli una rotta alternativa verso le Indie, però ricca di errori di calcolo, soprattutto dovuti alla sbagliata conversione tra le unità di misura. Furono proprio questi errori a portare Colombo a pensare che la circonferenza della Terra fosse molto minore di quanto lo è in realtà.
	
	Le famose tre caravelle di Colombo, la \textit{Niña} insieme alla \textit{Pinta} e alla \textit{Santa Maria}, salpano dal porto di Palos, Spagna, il 3 agosto 1492 e approdano in territorio americano il 12 ottobre 1492, più precisamente a San Salvador, nelle odierne isole Bahamas. Da quel momento, quella parte di mondo diviene terra spagnola, almeno secondo la volontà di Colombo stesso che ne prende possesso, senza nessuna consultazione con gli abitanti, a nome dei reali di Spagna. Seguono a questo altri tre viaggi nel periodo tra il 1493 e il 1500, in cui Colombo continua l'esplorazione dei Caraibi, raggiungendo a sud le foci dell'Orinoco e ad ovest Panama.
	
	\paragraph{La scoperta di Vespucci}
	
	Nonostante i fondamentali viaggi di Colombo, il primo esploratore ad intuire che quelle terre non fossero l'Oriente ma un nuovo continente fu, nel 1507, il fiorentino Amerigo Vespucci con le sue esplorazioni lungo le coste del Brasile e dell'Argentina. Vespucci viaggia inizialmente nel Nuovo Mondo nel 1497 e probabilmente tocca terra nell'attuale penisola della Guayira (Colombia). Ciò si deduce dalle sue lettere a Lorenzo il Popolano sulla cultura degli indigeni nativi.\cite{robertson1917vespucci}
	
	\paragraph*{La prima circumnavigazione del globo}
	
	Alcuni anni dopo la scoperta del continente americano, nel 1519, un navigatore portoghese conosciuto con il nome di Ferdinando Magellano organizza la prima spedizione spagnola verso le Indie, passando da Ovest, che sarebbe durata fino al 1522. Questa proposta è inizialmente mostrata al re di Portogallo, Re Manuele I, che però decide di non finanziare l'azzardata spedizione. Per questo motivo, Magellano si reca presso la corte di re Carlo I di Spagna che desiderava aprire rotte commerciali verso le isole Maluku, famose per le loro spezie. All'esploratore portoghese è affidata una flotta composta da 5 vascelli, che lo avrebbe dovuto condurre verso l'Oriente. Nonostante diverse tempeste ed ammutinamenti, Magellano riesce ad attraversare il famoso stretto che ancora oggi porta il suo nome, permettendogli di raggiungere una massa d'acqua che Magellano definisce "mare pacifico" (oggi conosciuto come Oceano Pacifico) e le isole Filippine, dove trova la sua morte.\cite{hartig1910magellan} Nonostante ciò, la spedizione arriva presso le isole Maluku e una nave riesce a ritornare in Spagna dopo aver circumnavigato il continente africano, ma, aspetto più importante, avendo circumnavigato l'intero globo per la prima volta.
	
	\paragraph*{Il trattato di Tordesillas}
	
	Le nuove scoperte e i nuovi territori occupati da Colombo non tardarono a creare una disputa tra le altre potenze marittime, come il Portogallo. Nel 1481, dopo essere salito al soglio pontificio, Papa Sisto IV garantisce tutte le terre a sud delle isole Canarie al Regno di Portogallo. Nonostante ciò, nel 1493, il Papa spagnolo Alessandro VI pubblica la bolla \textit{Inter Caetera} che dà tutte le terre a ovest di un meridiano a ovest delle isole di Capo Verde alla Spagna e tutte le terre ad est del suddetto meridiano al Portogallo. Ciò, ovviamente, non soddisfa il sovrano portoghese che a causa di questo trattato  perderebbe tutte le sue colonie in Brasile; ciò spinge Giovanni II di Portogallo ad aprire delle trattative con i Re Cattolici per spostare il meridiano più ad ovest, sostenendo che il meridiano proposto dal Papa avrebbe limitato in controllo spagnolo sulle Indie.\cite{parry2010age} 
	
	\begin{figure}[h]
		\centering
		\includegraphics[width=0.7\textwidth]{"tordesillas"}	
		\caption{{\small Il trattato di Tordesillas del 1494}}
	\end{figure}
	
	\section*{Il colonialismo}
	
	\paragraph*{I \textit{conquistadores}}
	
	In occasione del trattato di Torsedillas il Papa incarica ufficialmente la Spagna e il Portogallo di evangelizzare i territori scoperti. Ciò dà la possibilità agli Stati iberici di portare la "buona novella" sia dal punto di vista religioso che dal punto di vista militare: nascono i \textit{conquistadores}. Immediatamente dopo la firma del trattato, i colonizzatori spagnoli cominciano a posizionare delle fortezze in territori insulari, come il Porto Rico, Cuba e Santo Domingo. Tutto ciò, ovviamente, porta alla quasi completa distruzione e riduzione in schiavitù delle popolazioni indigene. Queste ultime subiscono un forte  shock culturale, provocato dall'enorme avanzamento tecnologico degli europei rispetto agli \textit{Indios}, che vennero in contatto con tecnologie a loro sconosciute: la ruota, i cavalli, le armature metalliche, le armi da fuoco e la lingua spagnola. Oltre a ciò, queste popolazioni sperimentano anche gli aspetti più negativi della società "civilizzata" europea: l'influenza, il morbillo, la peste bubbonica, la malaria e il vaiolo.\cite{crosby1967conquistador} Gli unici a beneficiare da questa situazione sono, ovviamente, la Spagna e il Portogallo, che importano dalle Americhe enormi quantità di metalli preziosi e di piante ancora sconosciute in Europa, come il mais, le patate e il cacao. Tutto ciò crea nuove rotte commerciali e maggiori beni da scambiare, in particolare generi di lusso, che permettono ai due stati iberici di divenire due delle più importanti potenze economiche dell'epoca. 
	
	\begin{wrapfigure}{r}{2.5in}
		\includegraphics[width=2.5in]{"conquistadores"}
		\caption{{\small  \textit{Conquistadores} in preghiera prima di una battaglia}}
	\end{wrapfigure}
	
	Anche altre persone provenienti dalle più svariate classi sociali decidono di approfittare di questa opportunità: ammiragli e marinai che si arricchiscono derubando gli Indios dei loro beni e riducendoli in schiavitù. Anche diversi criminali comuni si imbarcano verso queste terre sconosciute, in cerca di una nuova vita o, più semplicemente, di beni da scambiare con il Vecchio Continente. Anche i predicatori ricevono un grande incentivo per convertire il maggior numero possibile di persone, che non erano mai venute in contatto con qualsiasi altro tipo di religioni oltre a quelle animistiche, secondo cui ogni fenomeno o cosa dell'universo sono dotati di anima e vivono di una loro vita, spesso creduta divina e degna di culto.
	
	\paragraph*{Il Regno di Spagna}
	
	L'intera opera brutale e cruenta di colonizzazione spagnola si sviluppa in modo simile all'organizzazione feudale del medioevo europeo, fondata, quindi, sulla fiducia tra i coloni e sulle \textit{encomiendas}, affidamenti terrieri. Ogni europeo riceve uno di questi appezzamenti di terra, in cui sono inclusi anche gli Indiani che abitano in quella zona, e ad ognuno è affidato il compito di gestire, organizzare e convertire il suo possedimento. Nonostante gli indigeni fossero considerati degli uomini liberi dal punto di vista giuridico, essi vengono sfruttati e fatti lavorare in condizioni alienanti e disumane.
	
	\paragraph*{Il Regno di Portogallo}
	
	Nel frattempo, mentre il dominio spagnolo si caratterizzava dalla conquista dei territori continentali, i Portoghesi organizzano un sistema coloniale alquanto differente. Esso si organizza secondo caratteri commerciali: ricco di centri di gestione e porti lungo le coste, che avevano il carattere di empori e scali commerciali, in particolare atti a commerciare con l'India. \\
	
	Uno dei più importanti \textit{conquistadores} spagnoli è certamente Alfonso de Albuquerque, che stabilì una vasta rete commerciale nell'oceano Indiano; infatti, dopo la scoperta della rotta verso l'Oriente da parte di Vasco da Gama, egli conquista diversi luoghi strategici come i territori dell'attuale Sri Lanka e dell'Indonesia, andando contro le flotte ottomane e indiane. Oltre a ciò, i Portoghesi fondano anche alcune colonie in Cina, come Macao, che rimase colonia portoghese fino al 1999.\\
	
	Una delle poche regioni in cui avviene una vera e propria colonizzazione continentale è il Brasile, a cui il Regno può accedere grazie al trattato di Torsedillas.
	
	 \subparagraph*{Cortez e Pizarro} Uno dei più importanti \textit{conquistadores} che ricordiamo è Hernan Cortes, che si accanì contro le popolazioni azteche, governate da Moctezuma II. Ciò li porta al dominio di altre regione dell'America centrale e alcuni territori oggi appartenenti agli Stati Uniti d'America. 
	
	Successivamente a Cortes, anche Francisco Pizarro continua a sottomettere le popolazioni \textit{pre-colombiane}, ormai così soprannominate, e ad espandere il dominio spagnolo lungo le coste occidentali dell'America Latina, colonizzando vasti territori oggi situati in Perù e Cile.
	
	\section{La denuncia sociale}
	
	Nonostante il gran numero di schiavisti e di atrocità presenti e compiute nelle Americhe da parte degli Spagnoli, solo alcuni uomini di Chiesa decidono di attirare l'attenzione dei cittadini europei su questo problema. Il più grande esempio che ricordiamo di questa opera di denuncia sociale è sicuramente il missionario Bartolomeo de Las Casas. Originariamente un \textit{encomandero}, dopo aver sperimentato la codizione alienante in cui gli Amerindi erano costretti a vivere, egli decide di rivolgersi al sovrano spagnolo Carlo V, che, nonostante abbia ascoltato le parole del frate, decise di ignorare il problema, poichè al tempo non era considerato un evento grave. La maggior parte degli intellettuali dell'epoca credevano che il trattamento riservato agli Indios fosse del tutto giusto, in quanto questi ultimi non venivano nemmeno considerati veri e propri esseri umani, ma esseri simili ad animali. Uno dei testimoni più significativi di questa mentalità estremamente chiusa è l'umanista Juan Gines de Sepulveda. Nel suo trattato riguardante la guerra contro gli Indios, egli giustifica l'operato dei \textit{conquistadores}, definendo i nativi americani degli "omuncoli".\cite{milazzo2014democrate} \\
	
	\begin{wrapfigure}{r}{2.5in}
		\includegraphics[width=2.5in]{"indios"}
		\caption{{\small Comunità indigena brasiliana ai tempi della colonizzazione}}
	\end{wrapfigure}
	
	Las Casas, prende come riferimento per le sue argomentazioni i valori cristiani di eguaglianza come principio fondante della fede. Las Casas constata che, dal punto di vista dottrinale, la religione cristiana può essere adottata da tutti, dopo di che afferma che tutte le nazioni sono destinate alla religione cristiana. Si ripete continuamente che gli indiani sono già dotati di caratteristiche cristiane e che non aspirano ad altro che a far riconoscere il loro cristianesimo selvaggio. Nelle descrizioni di Las Casa si esalta la docilità dei selvaggi e la loro predisposizione ad essere guidati. Egli rileva la mitezza e la tranquillità degli indiani, non solo nel loro stato psicologico ma anche nella loro configurazione culturale e sociale.\cite{de2015brevissima} Egli spiega i comportamenti docili e obbedienti degli indigeni col semplice fatto che questi si comportano come buoni cristiani, esaltando l'indifferenza degli indiani per i beni materiali, con la conseguente scarsa disposizione all'arricchimento e al lavoro, spiegandola esclusivamente con la morale cristiana.\\
	
	In conclusione, quando si tratta, in definitiva, di esprimere le differenze tra gli indiani e gli europei, Las Casas si affida a una terminologia, che oserei definire evoluzionista. Gli indigeni sono come gli europei erano tanto tempo prima, quando ancora non avevano conosciuto la parola del Cristo. Gli indigeni sono identici agli spagnoli e il tempo dimostrerà che il loro spirito inconsapevolmente cristiano li farà diventare migliori di quanto sia in quel momento.
	
	\bibliographystyle{naturemag}
	\bibliography{nuovo_mondo.bib}
	
\end{document}
