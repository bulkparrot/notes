\documentclass[11pt]{report}
\usepackage[margin=1in]{geometry}
\usepackage{amsfonts,amsmath,amssymb}
\usepackage[italian]{babel}
\usepackage{fancyhdr}
\usepackage{pgfplots}
\pgfplotsset{width=10cm,compat=1.9}

\title{La colonizzazione delle Americhe}
\author{Tommaso Severini}

\pagestyle{fancy}
\fancyhead{}
\fancyfoot{}
\fancyhead[L]{LA COLONIZZAZIONE}
\fancyhead[R]{Tommaso Severini}
\fancyfoot[C]{\thepage}

\parindent 0ex

\begin{document}
	
	
	\begin{titlepage}
		\begin{center}
			\Large\textbf{ III B Storia}
			
			\Large\textbf{Lavoro di approfondimento}
			\vfill
			\rule{400pt}{0.1pt}\\
			
			\huge\textbf{La colonizzazione delle Americhe}
			\rule{400pt}{0.1pt}
			\vfill
			{\small Tommaso Severini\\
			21 aprile 2021}
		\end{center}
	\end{titlepage}	
	
	\begin{abstract}
		Il processo di colonizzazione delle Americhe comportò la scoperta, l'esplorazione, la conquista e l'occupazione di svariati territori nel "Nuovo Mondo" da parte di varie nazioni europee, come l'Inghilterra, la Spagna, la Francia e il Portogallo. 
		
		Benché lo scopo fondamentale fosse quello di espandere i propri affari commerciali, civilizzando e propagando la fede cattolica in queste nuove terre, il processo di colonizzazione produsse una sistematica distruzione delle culture delle popolazioni locali.\\
		
		Nonostante questo fenomeno riguardi particolarmente i paesi occidentali, intendo cominciare a descrive il trascorrere degli eventi a partire dalla caduta della capitale dell'impero bizantino, Costantinopoli. Questo evento ebbe una grande influenza sul rinascimento,epoca storica durante la quale ebbe inizio il processo di colonizzazione.
	\end{abstract}
	
	\chapter{La caduta dell'impero romano d'Oriente}
	
	
	
\end{document}
