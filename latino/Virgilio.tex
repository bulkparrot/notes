\documentclass[10pt,a4paper]{article}
\usepackage[utf8]{inputenc}
\usepackage[T1]{fontenc}
\usepackage{amsmath}
\usepackage{amsfonts}
\usepackage{amssymb}
\usepackage{makeidx}
\usepackage{graphicx}
\usepackage[left=1.00in, right=1.00in, top=1.00in, bottom=1.00in]{geometry}
\author{Tommaso Severini}
\title{Letteratura latina - Virgilio}
\begin{document}
	\maketitle
	
	\section*{Vita}
	
	Virgilio nacque ad Andes, un piccolo villaggio sito nei pressi del territorio dell'antica città di Mantova, nella Gallia Cisalpina, divenuta parte integrante dell'Italia romana quasi una ventina d'anni prima della sua nascita, il \textbf{15 ottobre del 70 a.C.} da una benestante famiglia di coloni romani, figlio di Marone Figulo, un piccolo proprietario terriero, arricchitosi considerevolmente con \textbf{l'apicoltura, l'allevamento e l'artigianato} (attività che lui descriverà in dettaglie nelle sue "Georgiche").
	
	Virgilio frequenta la \textbf{scuola di grammatica} a Cremona, poi la scuola di filosofia a Napoli, dove si avvicina alla corrente \textbf{filosofica epicureista} grazie a Sirone e infine la scuola di retorica a Roma. Qui conobbe molti poeti e uomini di cultura e si dedicò alla composizione delle sue opere. Inoltre nella capitale portò a termine la propria formazione oratoria studiando eloquenza alla scuola di Epidio, un maestro importante di quell'epoca.
	
	\section*{Contestualizzazione storica} 
	
\end{document}