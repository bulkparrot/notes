\documentclass[10pt,a4paper]{article}
\usepackage[utf8]{inputenc}
\usepackage[T1]{fontenc}
\usepackage{amsmath}
\usepackage{amsfonts}
\usepackage{amssymb}
\usepackage{makeidx}
\usepackage{graphicx}
\usepackage[left=1.00in, right=1.00in, top=1.00in, bottom=1.00in]{geometry}
\author{Tommaso Severini}
\title{Letteratura latina - Virgilio}
\begin{document}
	\maketitle
	
	\subsection*{Vita}
	
	Virgilio nacque ad Andes, un piccolo villaggio sito nei pressi del territorio dell'antica città di Mantova, nella Gallia Cisalpina, divenuta parte integrante dell'Italia romana quasi una ventina d'anni prima della sua nascita, il \textbf{15 ottobre del 70 a.C.} da una benestante famiglia di coloni romani, figlio di Marone Figulo, un piccolo proprietario terriero, arricchitosi considerevolmente con \textbf{l'apicoltura, l'allevamento e l'artigianato} (attività che lui descriverà in dettaglie nelle sue "Georgiche").
	
	Virgilio frequenta la \textbf{scuola di grammatica} a Cremona, poi la scuola di filosofia a Napoli, dove si avvicina alla corrente \textbf{filosofica epicureista} grazie a Sirone e infine la scuola di retorica a Roma. Qui conobbe molti poeti e uomini di cultura e si dedicò alla composizione delle sue opere. Inoltre nella capitale portò a termine la propria formazione oratoria studiando eloquenza alla scuola di Epidio, un maestro importante di quell'epoca.
	
	\subsection*{Contestualizzazione storica}
	
	Gli anni in cui Virgilio si trova a vivere sono anni di grandi sconvolgimenti a causa delle guerre civili: prima lo scontro tra Cesare e Pompeo, culminato con la sconfitta di quest'ultimo a Farsalo (48 a.C.), poi l'uccisione di Cesare (44 a.C.) in una congiura, e lo scontro tra Ottaviano e Marco Antonio da una parte e i cesaricidi (Bruto e Cassio) dall'altra, culminato con la battaglia di Filippi (42 a.C.). Egli fu toccato direttamente da queste tragedie come testimoniano le sue opere: infatti la distribuzione delle terre ai veterani dopo la battaglia di Filippi mise in grave pericolo le sue proprietà nel Mantovano ma sembra che, grazie all'intercessione di personaggi influenti (Pollione, Varo, Gallo, Alfeno, Mecenate e dunque lo stesso Augusto), Virgilio sia riuscito (almeno in un primo tempo) ad evitare la confisca. Si spostò poi a Napoli con la famiglia e in seguito nel 38 si fece assegnare da Mecenate un podere in Campania come risarcimento per le proprietà perdute ad Andes. In Campania avrebbe terminato le Bucoliche e composto le Georgiche, dedicate all'amico Mecenate, che Virgilio frequentava.
	
	\subsubsection*{Ottaviano}
	
	Ottaviano inizia la sua ascesa al poter un anno dopo la morte del suo padre adottivo, nel 43 a.C., con la formazione del secondo triumvirato \textit{rei publicae costituendae}, come osservato da Ottaviano stesso nelle sue \textit{Res Gestae}. Questo accorda,  questa volta sancito secondo decreti ufficiali, aveva il compito di difendere la patria da ulteriori attacchi da parte dei cesaricidi e di riformare la patria. Infatti, nell'anno successivo, presso Filippi avvenne la battaglia che vide come vincitori i triumviri e come vinti i cesaricidi. 
	
	Successivamente a questa vittoria, però, Lepido, uno dei triumviri, fu esaurtorato dalla vita politica. Ciò rese Ottaviano un passo più vicino alla conquista del potere assoluto; il suo unico avversario rimasto era Marco Antonio.
	
	Sfortunatamente per quest'ultimo, la possibilità di iniziare una "guerra" contro di lui si realizzò in seguito alla lettura da parte di Ottaviano in senato del testamento di Antonio. Egli infatti desiderava lasciare alcune delle provincie romane sotto il suo controllo ai suoi figli avuti con il faraone Cleopatra. Il senato dichiarò Marco Antonio un nemico della repubblica e concesse a Ottaviano poteri straordinari per eliminarlo. La battaglia di Azio, nel 31 a.C., segno la sconfitta definitiva dell'ex luogotenente.\\
	
	Ottaviano era riuscito a ottenere il potere, ma sopratutto, aspetto più importante, il consenso della repubblica. Di ritorno dal campo di battaglia gli furono conferiti i titoli onorifici di \textbf{imperator} e \textbf{princeps senatus}, nel 29 a.C. .
	
	Questi eventi segnano l'inizio del dominio assolutistico di Ottaviano, che sarà successivamente concretizzato dopo la \textit{restitutio rei publicae} e l'attribuzione dell'onorifico \textbf{Augustus}, nel 27 a.C. .
	
	\section*{Opere}
	
	Le opere virgiliano possono essere tutte accomunate secondo un concetto chiave: la \textbf{guerra}. Infatti Virgilio, come si è detto, vive in un periodo di grandi sconvolgimenti dal punto di vista politico e sociale che lo hanno molto segnato. Nell'Eneide, \textit{opus magnus} del poeta, la guerra funge da motivo scatenante del viaggio di Enea, che lo porterà a vistare il Mediterraneo prima di giungere finalmente a Roma. Nelle Georgiche, invece, Virgilio espone il dramma che la guerra a portato in Italia, sperando che sotto il potere di Ottaviano la pace possa essere ristabilita. Infine, nelle sue Bucoliche, l'autore descrive i problemi che lo hanno direttamente colpito, come la perdita del suo terreno a seguito della battaglia di Filippi.
	
	\subsection*{Bucoliche}
	
	Le \textbf{\textit{Bucoliche}} sono una raccolta di 10 brevi carmi, detti \textit{ecloghe}, appartenenti principalmente alla poesia pastorale, o bucolica. Esse furono realizzate nel periodo di tempo che va dal 42 a.C. al 39 a.C. .\\
	
	\subsubsection*{Modelli}
	
	Le ecloghe sono carmi autonomi che, escludendo la IV e la VI, hanno come ambientazione un paesaggio bucolico abitato da pastori, i quali vivono secondo uno stile di vita sereno e dedicato all'\textbf{otium letterario}. Ciò ovviamente è un aspetto tipico di alcuni poeti classici che furono da modello per le opere virgiliane. In particolare, il poeta che ha avuto la maggiore influenza sull'opera è \textbf{Teocrito}, autore degli Idilli. Quest'ultimo utilizzava le attività pastorali come un pretesto, per dare spazio a temi come l'amore ed i suoi effetti (Ciò può essere interpretato anche come un elemento che vede la ripresa della filosofia epicureista: i pastori si allontanano dal chaos cittadino per vivere in uno stato di solitudine ed atarassia, seguendo il principio del \textbf{lathe biosas}).\\
	
	Altro elemento fondamentale da considerare è la totale assenza di fatica nel lavoro e di riferimenti agli eventi che hanno caratterizzato la fine della Repubblica. Ciò deve conferire all'ambiente un carattere quasi edenico e simile a quello dell'\textbf{Arcadia}. Questo ambiente, descritto per la prima volta da Polibio, assume il significato odierno grazie allo stesso Virgilio: un luogo popolato da pastori-poeti capaci di esprimere se stessi attraverso poesie e canti profondi.
	
	\subsubsection*{Struttura}
	
	L'elevata attenzione ai dettagli e la simmetria presente nella disposizione delle ecloghe ci permette non solo di comprendere meglio l'importanza del modello alessandrino in Virgilio, ma anche di porre maggiore enfasi sui temi delle ecloghe stesse. Le "coppie" di ecloghe su cui poniamo maggiore attenzione sono la I-IX, di argomento autobiografico e in cui Virgilio descrive indirettamente la perdita del suo podere, e la IV-VI, uniche ecloghe di argomento non bucoliche che sono usate, oltre a spiegare il tema della \textbf{palingenesi}, per lodare Asinio Pollione ed Ottaviano, che hanno aiutato Virgilio a recuperare le sue terre.
	
	\subsubsection*{Leitmotiv}
	
	Cercando di interpretare quest'opera da un punto di vista più umano e filosofico, emerge un leitmotiv onnipresente nelle opere virgiliane: un sentimento di ingiustizia e fragilità che caratterizza un mondo continuamente spezzato da guerre e anche la vita umana stessa. 
	
	Per questo motivo la poesia si eleva: da mero oggetto di intrattenimento ad arte in grado di consolare e far dimenticare le sofferenze che caratterizzano l'esistenza umana. 
	
	\subsubsection*{Testi}
	
	\textsc{Ecloga I}\\
	
	Motivi conduttori
	
	\begin{itemize}
		\item Tema dell'abbandono (dovuto alla guerra)
		\item Contatto tra uomo e natura (che si concilia con l'atarassia epicurea)
		\item Tema della devozione (sia nei confronti di Asinio Polliono, sia nei confronti del \textit{deus} Ottaviano)
	\end{itemize}
	
	Il primo carme dell'opera si sviluppa in un paesaggio arcadico, che riprende anche elementi della terra natia del poeta. Essa, quindi, risulta un ambiente idealizzato, ma nonostante ciò indirettamente legato ad un evento reale che ha segnato la vita del Virgilio: la ridistribuzione delle terre a seguito della battaglia di Filippi.
	
	Infatti, i due protagonisti dell'ecloga non sono altro che simboli che rappresentano lo stato d'animo dell'autore. Titiro è un pastore fortunato, destinato a rimanere nel suo podere dedito all'otium(successivo Virgilio); Melibeo, pastore sfortunato, è costretto a lasciare la sua terra e ad abbandonare ciò che fino a quel momento era stata la sua vita(precedente Virgilio). Questa impossibilità di poter vivere la sua esistenze come sempre permette di evidenziare il primo \textit{leit motiv} che accompagna tutto il canto: \textbf{l'abbandono}.
	
	In particolare, riferimenti a questo motivo conduttore sono i termini che hanno a che vedere con il campo semantico della fuga: si osservi il poliptoto "patriae"/"patriam" che mette in risalto l'idea dell'appartenenza ad un territorio o l'utilizzo di termini come "aeger ago" e "linquimus"/"fugimus", caratterizzati da un poliptoto, che riconducono al futuro che spetta al nostro Melibeo.\\
	
	Oltre a questo tema, in questa ecloga, è possibile porre enfasi sull'altro elemento ricorrente nel resto del canto: l'\textbf{otium letterario}. 
	
	Mentre, da un lato, osserviamo la sofferenza di Melibeo, dall'altro è necessario analizzare anche lo stile di vita del pastore Titiro, che viene descritto nell'atto di recitare un canto sotto l'ombra di un faggio. Egli, infatti, a seguito di un dubbio da parte di Melibeo, gli spiega che il motivo per cui egli può ancora vivere in questo "paradiso" è grazie all'intercessione di un \textit{deus} che ha tutelato il suo otium. Questa figura rappresenta, ovviamente, Ottaviano, che viene successivamente lodato nella maggior parte delle opere virgiliane, dando origine al tema di fondo \textbf{encomiastico riferito al futuro imperatore}.\\
	
	\textsc{Ecloga IV}\\
	
	Durante l'ottobre del 40 a.C., mentre Virgilio scriveva l'opera, l'atmosfera nell'Urbe era molto tesa, e la guerra civile era al suo culmine: nel 40 a.C. Ottaviano e Lucio Antonio (fratello di Marco) si scontrarono nella cruentissima battaglia di Perugia e, dopo di essa, alcuni mediatori (Nerva, Mecenate e lo stesso Pollione, amico di Virgilio e console in carica per quell'anno) riconciliarono i due triumviri, che stipularono quindi la pace di Brindisi; in base a questo trattato, ad Ottaviano fu assegnato l'Occidente, e ad Antonio l'Oriente; la penisola italica apparteneva ad entrambi. La tregua fu sancita con un matrimonio tra Ottavia, la sorella di Ottaviano, e Marco Antonio. Questo accordo fu salutato con grande speranza e gioia da parte dei veterani e degli abitanti di Roma, ed i due triumviri, tra il tripudio della folla, celebrarono l'ovazione.
	
	Anche Virgilio, di solito lontano dalla vita politica, dimostra grande entusiasmo per questo accordo: nella IV egloga, in particolare, con un registro stilistico notevolmente più alto rispetto alle altre, il poeta celebra l'imminenza del ritorno dei Saturnia Regna, in seguito alla nascita di un “bambino divino”, che avrebbe posto fine al tragico presente per inaugurare una nuova età dell'oro. Il poeta non fa il nome del puer, e il componimento assume così un tono profetico e misterioso. Secondo alcuni studiosi, questo bambino a cui, senza immaginare che sarebbe stata una femmina, Virgilio si riferisce, sarebbe il figlio derivante dall'unione tra Ottavia e Marco Antonio; secondo altre interpretazioni potrebbe essere Asinio Gallo, figlio di Pollione; infine, secondo una successiva interpretazione da parte degli amanuensi cristiani, la descrizione del puer non rappresenterebbe altro che una profezia sulla venuta di Gesù Cristo.\\
	
	
	Questa ecloga, insieme alla VI, si differenzia dal resto dell'opera per il suo argomento non agreste, bensì mitologico. Virgilio, infatti, ha voluto descrivere l'età dell'oro ideata da Esiodo e il suo avvento. 
	
	Secondo le leggende, durante l'età dell'oro gli esseri umani vivevano senza bisogno di leggi, né avevano la necessità di coltivare la terra poiché da essa cresceva spontaneamente ogni genere di pianta. Non esisteva la proprietà privata, non c'era odio tra gli individui e le guerre non flagellavano il mondo. Era sempre primavera e il caldo ed il freddo non tormentavano la gente, perciò non c'era bisogno di costruire case o di ripararsi in grotte. Con l'avvento di Giove finisce l'età dell'oro e ha inizio l'età dell'argento.
	
	\vfil
	
	\subsection*{Georgiche}
	
	Analizziamo la seconda delle opere virgiliane, che, a differenza delle bucoliche, non è di argomento agreste, ma un poema epico-didascalico. 
	
	\subsubsection*{Tematiche}
	
	Questa opera si prefigge l'obbiettivo di illustrare le tecniche utilizzate da allevatori e contadini nei lavori agricoli e al contempo di offrire insegnamenti relativi alla vita di ogni uomo.
	
	L'opera richiese un gran periodo di tempo per essere composta, dal \textbf{37 a.C. al 29 a.C.}, presso la sua dimora a \textbf{Napoli}. Ciò è dovuta al fatto che l'opera gli fu "commissionata" da Mecenate stesso per celebrare le politiche attuate da Augusto(basate sul \textit{mos maiorum} e il ritorno al contadino come esempio del romano laborioso), secondo azioni che Virgilio osò definire "haud mollia iussa"(comandi non leggeri). 
	
	\subsubsection*{Struttura}
	
	Il poema risulta diviso in 4 libri, ognuno dei quali tratta un aspetto differenta del lavoro agricolo: in ordine dal \textbf{lavoro nei campi, all'arboricoltura, all'allevamento del bestiame e all'apicoltura, per un totale di 2188 versi, precisamente esametri.}\\
	
	\textsc{Libro I}\\
	
	Dopo una dedica a Mecenate e una preghiera alle divinità della casa, Virglio inizia una digressione mirata a spiegare quali siano i momenti migliori e le modalità per eseguire la semina.
	
	A ciò segue la regola di Giove, che diventerà il comandamento che ogni uomo deve fare proprio: \textbf{il progresso va conquistato solo attraverso fatica e lavoro}, concezione secondo cui la laboriosità assume il ruolo di teodicea(giustizia divina).
	
	Sono successivamente illustrati i vari strumenti dei contadini con i rispettivi utilizzi ed altri consigli utili alla coltivazione. Il libro si chiude con il ricordo dei segni premonitori che portarono alla terza guerra civile.\\
	
	\textsc{Libro II}\\
	
	Nel secondo libro sono approfonditi tutte le caratteristiche e tecniche di coltivazione relative alle piante. In particolare Virgilio si sofferma sulla descrizione delle viti, degli ulivi e degli alberi da frutto. 
	
	Il libro si chiude con una lode alla vita del contadino.\\
	
	\textsc{Libro III}\\
	
	Iniziando con un invocazione alle divinità pastorali, il libro si dilunga sulla descrizione dell'allevamento delle bestie.
	
	L'autore pone particolare attenzione nel far capire al lettore quanto sia importante tenere sotto controllo il \textit{\textbf{furor}} amoroso in quanto può risultare nocivo (concetto ripreso anche nel libro successivo). 
	
	Il libro si conclude con un avvertimento nei confronti di serpenti e varie pestilenze che potrebbero affliggere gli animali.\\
	
	\textsc{Libro IV}\\
	
	Riguarda l'apicoltura. Questa \textit{ars} permette a Virgilio di esemplificare la figura di un cittadino romano modello. Egli infatti deve essere simile ad un'ape: laboriosa, instancabile, rispettosa della gerarchia ed in grado di saper proteggere il suo alveare e le suecompagne. Per questo motivo gli insetti sono detti \textbf{parvi Quirites}, ovvero piccoli Romani.
	
	Dopo aver offerto altri consigli, il libro si chiude con la parabola di \textbf{Aristeo e il mito di Orfeo ed Euridice}.
	
	\subsubsection*{Modelli}
	
	Anche lo stile è più ricco e ricercato rispetto alle Bucoliche, e coniuga i canoni dell'alessandrinismo e della poesia neoterica con il gusto spontaneo per il sublime e l'aspra versificazione scientifica del De rerum natura di Lucrezio, pubblicato nel 53 a.C., in un alternarsi ininterrotto di pungente malinconia e serena consapevolezza della caducità umana.
	
	\subsubsection*{Testi}
	
	\textsc{Georgiche I}
	
	
	
\end{document}