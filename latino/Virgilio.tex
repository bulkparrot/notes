\documentclass[10pt,a4paper]{article}
\usepackage[utf8]{inputenc}
\usepackage[T1]{fontenc}
\usepackage{amsmath}
\usepackage{amsfonts}
\usepackage{amssymb}
\usepackage{makeidx}
\usepackage{graphicx}
\usepackage[left=1.00in, right=1.00in, top=1.00in, bottom=1.00in]{geometry}
\author{Tommaso Severini}
\title{Letteratura latina - Virgilio}
\begin{document}
	\maketitle
	
	Virgilio nacque ad Andes, un piccolo villaggio sito nei pressi del territorio dell'antica città di Mantua (odierna Mantova), nella Gallia Cisalpina, divenuta parte integrante dell'Italia romana quasi una ventina d'anni prima della sua nascita, il 15 ottobre del 70 a.C. da una benestante famiglia di coloni romani, figlio di Marone Figulo, un piccolo proprietario terriero, arricchitosi considerevolmente con l'apicoltura, l'allevamento e l'artigianato (attività che lui descriverà in dettaglie nelle sue "Georgiche").
	
\end{document}