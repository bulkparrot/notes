\documentclass[10pt,a4paper]{article}
\usepackage[utf8]{inputenc}
\usepackage[T1]{fontenc}
\usepackage{amsmath}
\usepackage{amsfonts}
\usepackage{amssymb}
\usepackage{makeidx}
\usepackage{graphicx}

\usepackage{hyperref}

\usepackage[left=1.00in, right=1.00in, top=1.00in, bottom=1.00in]{geometry}
\author{Tommaso Severini}
\title{Letteratura latina - Cicerone}
\begin{document}
	\maketitle

 Marco Tullio Cicerone fu uno degli autori latini che ha saputo trasmettere meglio il significato di latinità. Loquace oratore ed impareggiabile avvocato, ebbe un ruolo fondamentale nel trasferimento della filosofai greca a Roma e incarna il tipico romano come figura di \textbf{azione e partecipe alla vita politica (negotium), ma anche come uomo appassionato dell'apprendimento (otium)}. Egli fu inoltre uno dei più importanti protagonisti della sua epoca, vivendo e difendendo a pieno la repubblica romana, destinata a crollare nelle mani di Ottaviano Augusto. 

\part*{Biografia}

Cicerone nasce ad Arpino, cittadina provinciale legata alla tradizione contadina e che diede alla luce Gaio Mario. Tullio non era un uomo nobile e fu il primo membro della sua famiglia ad intraprendere il \textit{cursus honorum}. Egli intraprende studi di tipo letterario, filosofico e legale, divenendo un grande oratore. Uno delle sue prime orazione si rivolse contro un \textbf{liberto del dittatore Silla, Crisogono}, persona sicuramente non di poco conto e che aveva fatto scrivere le famose \textbf{liste di proscrizione}. Per questo motivo, l'anno successivo andò in Grecia, ma, immediatamente dopo la morte del dittatore, tornò a Roma e divenne \textbf{questore in Sicilia}. Ciò gli permette di essere amato dal popolo e, cosa più importante, di aiutare il popolo a liberarsi del \textbf{governatore Verre}, che era accusato di aver rubato del denaro pubblico (le orazioni contro Verre prendono il nome di Verrine). 

\subsection*{Il consolato del 63 a.C.}

La sua schiacciante vittoria contro Verre diede la possibilità a Cicerone di proseguire nel suo cursus honorum e di cominciare a divenire una delle figure di spicco della vita politica romana. Nel 64 divenne pretore e nel 63 divenne addirittura console. Questo anno fu uno dei più decisivi della vita dell'oratore, durante il quale un senatore aveva organizzato una congiura per ottenere il potere, Catilina.\\

 Egli aveva provato a candidarsi diverse volte come console, ma fu ostacolato dall'operato del triumvirato e di Cicerone. Ciò aveva spinto il senatore ad organizzare un colpo di stato per ottenere il potere. Fortunatamente, la moglie di uno dei congiurati avvisò Cicerone di ciò che stava per accadere. Cicerone prende provvedimenti: il giorno successivo attende Catilina in senato e si accanisce contro di lui con una violenta arringa (le \textbf{Catilinariae}) e continua ad ostacolare ogni azione del congiurato finchè egli e la sua combriccola non sono dichiarati nemici dello stato. Catilina fu giustiziato senza processo per ordine di Cicerone e le milizie caatilinarie furono devastate. 
 
 \subsection*{Clodio: 58 a.C.}
 
 Nonostante Cicerone fosse soddisfatto del suo operato, ciò gli si ritorce contro qualche anno dopo, quando Clodio, tribuno della plebe, fa promulgare una legge con valore retroattivo che punisca tutti coloro che hanno condannato a morte un cittadino romano senza processo.

Cicerone si ritrova in esilio per un anno, ricordo che porterà con se per tutta la vita e che spesso citerà nelle sue opere.

\subsection*{La morte di Clodio: 52 a.C.}

Qualche anno dopo il suo esilio, Clodio viene trovato morto dopo uno scontro tra le bande armate comandate da Clodio e da Milone. Quest'ultimo sopravvive allo scontro ma viene accusato diomicidio da parte dei membri del partito cesariano. Cicerone deciderà di prendere le difese di Milone, scrivendo una delle sue orazioni più famose, la \textbf{Pro Milone}

\subsection*{La prima guerra civile: 49 a.C.}

Dopo il suo ritorno in patria, Cicerone dovette fare i conti con una situazione politica molto tesa, in cui i due protagonisti principali erano Cesare e Pompeo Magno. Da questa situazione, come sappiamo, ne scaturì la prima e sanguinosa guerra civile tra i due sopracitati. Nonostante Cicerone non fosse nè un cesariano nè un pompeiano, egli decise di fuggire da Roma e di recarsi in Grecia, vicino a Pompeo, in modo che, se egli avesse vinto, sarebbe stato più facile riorganizzare il governo con l'aiuto del senato. 

Purtroppo Pompeo muore per mano di Tolomeo XIII e Cicerone è costretto a chiedere in modo molto umile a Cesare di risparmiargli la vita attraverso tre orazioni a lode del dittatore (\textbf{Pro Marcello \url{https://it.wikipedia.org/wiki/Pro_Marcello}, Pro ligario e Pro rege Deiotaro}) in cui il senatore loda le capacità del dittatore e gli espone dei consigli per governare il modo migliore la repubblica di Roma . \\

\textit{cfr.} Epistola \textit{Ad familiares} VI,5

\subsection*{La lotta contro Marco Antonio: 44 a.C.}

Durante la tirannia cesariana, Cicerone ricoprì un ruolo marginale nella politica romana, ma tutto ciò cambiò con la fine della dittatura, con le Idi di marzo 44 a.C.

Dopo la morte di Cesare, il clima politico divenne molto teso: ai cesaricidi viene concessa l'amnestia, Ottaviano viene adottato per via testamentaria da Cesare e Marco Antonio cerca di ottenere il potere assoluto. Per questo motivo, Cicerone tentò fin da subito a dividere e dilaniare il partito cesariano, supportando il figlio di Cesare. Il rapporto tra i due divenne così stretto che Ottaviano cominciò a considerare Cicerone una figura paterna. Nel frattempo, Cicerone provò anche a infangare l'onore di Marco Antonio, scrivendo alcune delle orazione più crudeli dell'intera vita politica del senatore, le \textbf{Filippiche}.

\subsection*{Il secondo triumvirato: 43 a.C.}

Nonostante le sue buone intenzioni, il piano di Cicerone fallì dopo che il senato si rifiutò di concedere il consolato ad Ottaviano. Quest'ultimo decise di stringere di nuovo i suoi rapporti con Marco Antonio, così formando il secondo triumvirato insieme a Lepido. Una delle conseguenze di ciò furono le liste di proscrizione: i triumviri decisero di eliminare dal principio i loro nemici politici e tra essi Marco Antonio fece scrivere il nome di Cicerone, nonostante lo scontento di Ottaviano. Marco Tullio Cicerone verrà ucciso il 7 dicembre del 43 a.C.\\

La sua testa e le sue mani furono mozzate ed esposte nel foro, dove Cicerone soleva proclamare i suoi discorsi. 

\part*{Le opere}

\section*{Gli strumenti dell'oratore}

Due degli elementi più importanti di cui un oratore deve essere a conoscenza sono sicuramente \textbf{Retorica e oratoria}:

\begin{itemize}
	\item Retorica: essa è costituita dagli elementi che rendono un discorso persuasivo
	\item Oratoria: essa è costituita dalla capacità di saper usare la parola
\end{itemize} 

\section*{Le opere retoriche}

\subsection*{De oratore}

Per descrivere gli strumenti dell'oratore, egli decide di scrivere tre opere retoriche, che racchiudono gli elementi fondamentali dell'eloquenza ciceroniana. Il più importante di questi è sicuramente il \textbf{De oratore}.\\

Esso si sviluppa in 3 libri sotto forma di un dialogo platonico tra Crasso e Antonio, nonno si Marco Antonio, ambientato nel 91 a.C.

I due discutono le abilità che un buon oratore deve possedere. Antonio sostiene che l'oratore debba essere una figura in grado di saper trovare delle argomentazioni valide e fornita di abilità naturali (\textbf{ingenium}), mentre Antonio riconosce anche che esso debba anche possedere un'enorme cultura e perizia tecnica (\textbf{ars}). Attraverso questa descrizione, troviamo che la descrizione del buon oratore da parte di Cicerone corrisponde a quella data da Catone. Essa può essere riassunta dalla famosa frase \textbf{"vir bonus dicendi peritus"}.\\

In questa opera, Cicerone descrive altri aspetti dell'oratoria, ma anche elementi che ci permettono di individuare il pensiero filosofico di Cicerone. Infatti, la filosofia principale che emerge dall'opera è sicuramente quella eclettica, una combinazione del pensiero \textbf{platonico, aristotelico e stoico}. 
\begin{itemize}
	\item Stoicismo: riprende il concetto di \textit{munus}: dovere
	\item Platonismo: riprende il concetto di stato ideale, cercando di spiegare perchè la repubblica romana sia uno stato perfetto 
	\item Aristotelismo: riprende la concezione dell'universo 
\end{itemize} 

\subsection*{Brutus}

Questa opera reppresenta semplicemente un dialogo in cui Cicerone, nella sua villa a Tisculum, ripercorre la storia dell'oratoria e si concentra sull'importanza del \textbf{movere}, una dei 3 scopi dell'oratore.

\subsection*{Orator}

In quest'opera, Cicerone riassume tutti gli elementi più importanti trattati nelle opere precedenti. Formalizza le 5 \textbf{partitiones oratorie} e i 3 \textbf{scopi} dell'oratore. \\

\textbf{Partitiones oratorie} (IDEMA)

\begin{itemize}
	\item Inventio: trovare le giuste argomentazioni
	\item Dispositio: ottenere una buona struttura del discorso
	\item Elocutio: lo stile in cui il discorso è scritto e pronunciato
	\item Memoria: utilizzo della mnemotecnica
	\item Actio: l'apparizione in pubblico
\end{itemize} 

\textbf{Scopi dell'oratore} (PDM)

\begin{itemize} 
	\item Probare: esprimere la propria tesi
	\item Delectare: intrattenere e suscitare emozioni nello spettatore
	\item Movere: convincere lo spettatore, coinvolgendolo
\end{itemize}

\textbf{Struttura dell'orazione} (ENDPAP)

\begin{itemize}
	\item Exordium: parte iniziale
	\item Narratio: esposizione dei fatti
	\item Divisio: esposizione della struttura dell'orazione
	\item Propositio: si approfondiscono maggiormente i punti esposti nella Divisio
	\item Argomentatio: argomentazioni a favore della tesi
	\item Peroratio: riferimento al pubblico, in modo tale da coinvolgerlo 
\end{itemize}

\section*{Opere giudiziarie}

\subsection*{In Verrem}

Il primo processo giudiziario che lanciò Cicerone nella scena politica romana fu il processo contro Verre, governatore corrotto e disonesto della Sicilia accusato di concussione. \\

Cicerone si fece conoscere dai siciliani qualche anno prima, nel 75 a.C., quando fu questore sull'isola e fu apprezzato soprattutto per la sua onestà. I discorsi delle Verrine furono scritte nel 70 a.C. in occasione del processo, frutto di intense investigazioni spesso ostacolate dagli alleati di Verre.\\

Le orazioni \textit{In Verrem} si dividono in 3 parti, 1 preliminare (divinatio) e due libri (actio I e II), divisi rispettivamente in una e cinque orazioni. 
La \textit{Divinatio in Caecilium} è un'orazione scritta e recitata prima dell'inizio del processo per ottenere il permesso di condurre l'accusa contro Verre, al posto di Cecilio Nigro, alleato di Verre.

La \textit{actio I} è un'orazione utilizzata da Cicerone per raccontare il corso delle indagini contro Verre e i relativi ostacoli incontrati. L'accusa si presenta in  realtà molto breve, spiazzando completamente la difesa e passando subito all'interrogazione dei testimoni. Verre resterà talmente basito dal discorso e dalle azioni del suo accusatoreche deciderà di fuggire dall'Italia prima della fine del processo.

Le orazioni della \textit{actio II} non furono mai pronunciate a causa della decisione di Verre, ma Cicerone decide di divulgare le sue orazioni. \\

Questa opera rappresenta il primo esempio delle enormi capacità oratorie di Cicerone, in grado di dipingere Verre come un essere spregevole e di farlo fuggire con un solo discorso. 

\subsection*{In Catilinam}

Tra il 67 e 62 a.C., mentre Pompeo era impegnato in Oriente, a Roma la crisi dello Stato non era ancora stata risolta e la politica romana era minacciata da Lucio Sergio Catilina, rappresentante della vecchia nobilitas arricchitasi durante la dittatura sillana e poi via via indebolitasi. In un primo momento, Catilina aspirava a raggiungere il consolato per mettere in atto una riforma costituzionale ed economica e successivamente una dittatura, indispensabile per lui. La riforma consisteva nel sottrarre le magistrature e sacerdozi agli oligarchici, ridistribuire le ricchezze e modificare i sistemi giudiziari. A determinare il cambiamento del suo atteggiamento politico furono elementi egoistici, l'ambizione di molti ottimati che aspiravano al consolato e il timore di una dittatura di Pompeo.\\

Cicerone, attraverso una delle mogli dei congiurati, scopre il piano di Catilina e comincia a scrivere \textbf{4 discorsi} che inizierà a pronunciare l'\textbf{8 ottobre 63 a.C.} nella seduta del senato nel tempio di Giove Statore. Questi discorsi sono giunti a noi intatti e prendo il nome di Catilinarie.

In questi discorsi, Cicerone critica e denigra il suo avversario, di cui non viene più di tanto criticata la \textit{coniuratione}, ma l'\textbf{audacia} ed il \textbf{furor}.

Oltre a ciò, Cicerone inizia il secondo paragrafo della sua orazione con un'esclamazione divenuta ormai proverbiale \textit{O tempora, o more!}. Cicerone utilizza la terza persona per distaccarsi ancora di più da Catilina, isolato nel senato. Egli conclude il suo discorso affermando come Catilina meriti la morte più di altri uomini morti per volere di Roma,come i Gracchi, e con un avvertimento nei confronti del senatore, ricordandogli che il \textit{senatus consultum ultimum} era stato approvato.

\subsection*{Pro Archia}

Originario di Antiochia, Archia giunse a Roma, dove intraprese l’attività di poeta sotto il patronato di Lucullo.
Questi lo aiutò ad assumere lo status di cittadino del municipio di Eraclea. Quindi, Archia assunse residenza
permanente a Roma, aspettando di acquisire la piena cittadinanza romana. Fu proprio a Roma che Archia
divenne il mentore e insegnante di Cicerone nella sua educazione retorica. Nel 65 a.C. la Lex Papia de Peregrinis
rigettò false richieste di cittadinanza ed espulse gli stranieri da Roma. \textbf{È probabilmente sotto questa legge che
Archia fu perseguito.} Cicerone entrò nel processo in difesa del suo insegnante nel 62 a.C., sperando che Archia
componesse un poema celebrativo su di lui perché aveva sventato la congiura di Catilina quando era console nel
63.
La difesa fu semplice per \textbf{tre motivi: Archia aveva come patronus Lucullo, cittadino molto in vista; nella giuria
era presente Quinto, il fratello di Cicerone; non esistevano più i registri della città di Eraclea con i nomi dei
cittadini romani, quindi era impossibile dimostrare che Archia non fosse presente negli elenchi.} Dopo aver
concluso le argomentazioni giudiziarie, Cicerone si dedica ad un vero e proprio panegirico (discorso di lode) sul
ruolo della poesia in una società civile.\\

 Dal momento che l’aspetto giudiziario dell’orazione è molto limitato,
mentre l’elogio della letteratura è estremamente ampio, possiamo considerare la “Pro Archia” un’orazione che si
avvicina al sottogenere \textbf{epidittico (celebrativo)}. In questa orazione, Cicerone tende a lodare il poeta per la sua \textit{humanitas}, intesa come cultura e arricchimento spirituale. Per questo motivo, Cicerone presenta molte citazioni di altri autori, come Ennio, e filosofi, come Platone. In quest'opera, infatti, è possibile ritrovare l'immagine dell'oratore descritta nel \textit{De oratore}, ovvero di \textbf{vir bouns dicendi peritus}.

La poesia assume sia una connotazione individuale che comunitaria: oltre ad arricchire il singolo individuo, permette di celebrare le gesta degli eroi e renderle immortali. Quest'ultimo tema della \textbf{lotta contro l'oblio} è uno dei più forti in questa orazione, tanto che Cicerone si aspetta che Archia scriverà un poema in cui sono descritte le gesta durante il suo consolato. Purtroppo, questo desiderio sarà disatteso.

\textbf{Argomentazioni}

\begin{itemize}
	\item I argomentazione: la poesia offre \textbf{insegnamenti morali}
	\item II argomentazione: il poeta possiede un \textbf{dono proveniente dagli dei}. Per questo motivo bisogna concedere la cittadinanza ad Archia, come molte città fecero con Omero.
	\item III argomentazione: la poesia non arricchisce solo a livello personale, ma celebra anche le opere collettive
	\item IV argomentazione: La letteratura greca è più diffusa rispetto a quella latina, quindi Archia darebbe la possibilità ai Romani di far conoscere le sue gesta ovunque
\end{itemize}

\subsection*{Philippicae}

L'ultima battaglia retorica di Cicerone si svolge nei confronti di Marco Antonio, successivamente all'assassinio di Cesare,ne a.C.. Infatti, Antonio si circondò dei membri più eminenti della fazione cesariana, con l'obbiettivo di diventare il "nuovo Cesare". Il titolo, posto da Cicerone stesso, è un riferimento ai discorsi critici pronunciati da Demostene nei confronti di Filippo II di Macedonia, un tiranno. Cicerone, infatti, condanna aspramente Antonio, ricordando tutti gli sforzi compiuti dalla cittadinanza romana per eliminare la monarchia. 

In particolare, nella \textit{Philippica II} ricorda l'episodio in cui Antonio, durante dei festeggiamenti pubblici (\textbf{I Lupercali}), tenta di porre un diadema sulla testa di Cesare offrendogli la possibilità di diventare re. Cicerone si sfoga con tutta la sua rabbia nei confronti di Antonio, ponendo enfasi sulla premeditazione degli atti compiuti. 

Saranno proprio queste parole che, dopo la formazione del secondo triumvirato, porteranno alla morte di Cicerone stesso.

\subsection*{Pro Milone}

La \textit{Pro Milone} è un'orazione che avrebbe dovuto pronunciare Cicerone nel 52 a.C., durante il processo contro Milone. Egli infatti, esponente del partito degli \textit{optimates}, fu accusato di aver ucciso Clodio, tribuno della plebe dei \textit{populares}, durante uno scontro tra bande armate della città. 

Dato il suo odio nei confronti di Clodio, che lo aveva costretto all'esilio, e la simpatia nei confronti di Milone, decide di prendere le sue difese durante il suo processo. Nonostante ciò, appena salito sui rostri, una folla inferocita, e amante del demagogico Clodio, impedisce a Cicerone di farsi sentire, rendendo la sua difesa totalmente inefficace. Milone fu trovato colpevole.

Nonostante l'orazione originale non sia pervenuta fino ai giorni nostri, ci troviamo in possesso di una copia scritta successivamente al processo dallo stesso Cicerone; questa risulta divisa in 7 momenti.\\

Nel brano letto è possibile analizzare la parte centrale del discorso, in cui Cicerone, dopo aver narrato i fatti, spiega come Milone abbia agito secondo la legittima difesa, mostrando che Milone non avrebbe mai compiuto un gesto talmente ignobile. Egli motiva le sue argomentazioni mostrando il fatto che la morte di Milone avrebbe portato più vantaggi a Clodio, e non viceversa. Oltre a ciò, Milone non ha mai provato ad attaccare clodio dal punto di vista fisico, ma sempre ricorrendo alla legge. 

\part*{Pensiero politico}

\subsection*{Pro Sestio}

La Pro Sestio è un'orazione di Cicerone composta nel 56 a.C., nella quale il famoso oratore difende il tribuno della plebe Publio Sestio. Questo era accusato de vi, per aver organizzato delle bande armate da opporre a quelle di Clodio per favorire il rientro in patria di Cicerone.

Grazie alle sue abilità oratorie, Cicerone salva Sestio, sostenendo la tesi (apparentemente contraddittoria rispetto alle sue posizioni legalitarie) che il ricorso a mezzi illegali si è reso necessario proprio per la difesa delle istituzioni, gravemente minacciate dai programmi eversivi dei Populares. Lancia inoltre un appello per il \textbf{consensus omnium bonorum}, ossia per un'alleanza di tutti i cittadini moderati che miri alla salvaguardia degli interessi comuni. Questo \textit{consensus} ricopre un ruolo fondamentale perchè si differenzia dal fallimentare \textit{conndordium ordinum}, politica ciceroniana con cui si era tentato si unire classe equestre e ceto senatorio.

\subsection*{De re publica}

\end{document}