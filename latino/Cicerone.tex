\documentclass[10pt,a4paper]{article}
\usepackage[utf8]{inputenc}
\usepackage[T1]{fontenc}
\usepackage{amsmath}
\usepackage{amsfonts}
\usepackage{amssymb}
\usepackage{makeidx}
\usepackage{graphicx}

\usepackage{hyperref}

\usepackage[left=1.00in, right=1.00in, top=1.00in, bottom=1.00in]{geometry}
\author{Tommaso Severini}
\title{Letteratura latina - Cicerone}
\begin{document}
	\maketitle

 Marco Tullio Cicerone fu uno degli autori latini che ha saputo trasmettere meglio il significato di latinità. Loquace oratore ed impareggiabile avvocato, ebbe un ruolo fondamentale nel trasferimento della filosofai greca a Roma e incarna il tipico romano come figura di \textbf{azione e partecipe alla vita politica (negotium), ma anche come uomo appassionato dell'apprendimento (otium)}. Egli fu inoltre uno dei più importanti protagonisti della sua epoca, vivendo e difendendo a pieno la repubblica romana, destinata a crollare nelle mani di Ottaviano Augusto. 

\part*{Biografia}

Cicerone nasce ad Arpino, cittadina provinciale legata alla tradizione contadina e che diede alla luce Gaio Mario. Tullio non era un uomo nobile e fu il primo membro della sua famiglia ad intraprendere il \textit{cursus honorum}. Egli intraprende studi di tipo letterario, filosofico e legale, divenendo un grande oratore. Uno delle sue prime orazione si rivolse contro un \textbf{liberto del dittatore Silla, Crisogono}, persona sicuramente non di poco conto e che aveva fatto scrivere le famose \textbf{liste di proscrizione}. Per questo motivo, l'anno successivo andò in Grecia, ma, immediatamente dopo la morte del dittatore, tornò a Roma e divenne \textbf{questore in Sicilia}. Ciò gli permette di essere amato dal popolo e, cosa più importante, di aiutare il popolo a liberarsi del \textbf{governatore Verre}, che era accusato di aver rubato del denaro pubblico (le orazioni contro Verre prendono il nome di Verrine). 

\subsection*{Il consolato del 63 a.C.}

La sua schiacciante vittoria contro Verre diede la possibilità a Cicerone di proseguire nel suo cursus honorum e di cominciare a divenire una delle figure di spicco della vita politica romana. Nel 64 divenne pretore e nel 63 divenne addirittura console. Questo anno fu uno dei più decisivi della vita dell'oratore, durante il quale un senatore aveva organizzato una congiura per ottenere il potere, Catilina.\\

 Egli aveva provato a candidarsi diverse volte come console, ma fu ostacolato dall'operato del triumvirato e di Cicerone. Ciò aveva spinto il senatore ad organizzare un colpo di stato per ottenere il potere. Fortunatamente, la moglie di uno dei congiurati avvisò Cicerone di ciò che stava per accadere. Cicerone prende provvedimenti: il giorno successivo attende Catilina in senato e si accanisce contro di lui con una violenta arringa (le \textbf{Catilinariae}) e continua ad ostacolare ogni azione del congiurato finchè egli e la sua combriccola non sono dichiarati nemici dello stato. Catilina fu giustiziato senza processo per ordine di Cicerone e le milizie caatilinarie furono devastate. 
 
 \subsection*{Clodio: 58 a.C.}
 
 Nonostante Cicerone fosse soddisfatto del suo operato, ciò gli si ritorce contro qualche anno dopo, quando Clodio, tribuno della plebe, fa promulgare una legge con valore retroattivo che punisca tutti coloro che hanno condannato a morte un cittadino romano senza processo.

Cicerone si ritrova in esilio per un anno, ricordo che porterà con se per tutta la vita e che spesso citerà nelle sue opere.

\subsection*{La morte di Clodio: 52 a.C.}

Qualche anno dopo il suo esilio, Clodio viene trovato morto dopo uno scontro tra le bande armate comandate da Clodio e da Milone. Quest'ultimo sopravvive allo scontro ma viene accusato diomicidio da parte dei membri del partito cesariano. Cicerone deciderà di prendere le difese di Milone, scrivendo una delle sue orazioni più famose, la \textbf{Pro Milone}

\subsection*{La prima guerra civile: 49 a.C.}

Dopo il suo ritorno in patria, Cicerone dovette fare i conti con una situazione politica molto tesa, in cui i due protagonisti principali erano Cesare e Pompeo Magno. Da questa situazione, come sappiamo, ne scaturì la prima e sanguinosa guerra civile tra i due sopracitati. Nonostante Cicerone non fosse nè un cesariano nè un pompeiano, egli decise di fuggire da Roma e di recarsi in Grecia, vicino a Pompeo, in modo che, se egli avesse vinto, sarebbe stato più facile riorganizzare il governo con l'aiuto del senato. 

Purtroppo Pompeo muore per mano di Tolomeo XIII e Cicerone è costretto a chiedere in modo molto umile a Cesare di risparmiargli la vita attraverso tre orazioni a lode del dittatore (\textbf{Pro Marcello \url{https://it.wikipedia.org/wiki/Pro_Marcello}, Pro ligario e Pro rege Deiotaro}) in cui il senatore loda le capacità del dittatore e gli espone dei consigli per governare il modo migliore la repubblica di Roma . \\

\textit{cfr.} Epistola \textit{Ad familiares} VI,5

\subsection*{La lotta contro Marco Antonio: 44 a.C.}

Durante la tirannia cesariana, Cicerone ricoprì un ruolo marginale nella politica romana, ma tutto ciò cambiò con la fine della dittatura, con le Idi di marzo 44 a.C.

Dopo la morte di Cesare, il clima politico divenne molto teso: ai cesaricidi viene concessa l'amnestia, Ottaviano viene adottato per via testamentaria da Cesare e Marco Antonio cerca di ottenere il potere assoluto. Per questo motivo, Cicerone tentò fin da subito a dividere e dilaniare il partito cesariano, supportando il figlio di Cesare. Il rapporto tra i due divenne così stretto che Ottaviano cominciò a considerare Cicerone una figura paterna. Nel frattempo, Cicerone provò anche a infangare l'onore di Marco Antonio, scrivendo alcune delle orazione più crudeli dell'intera vita politica del senatore, le \textbf{Filippiche}.

\subsection*{Il secondo triumvirato: 43 a.C.}

Nonostante le sue buone intenzioni, il piano di Cicerone fallì dopo che il senato si rifiutò di concedere il consolato ad Ottaviano. Quest'ultimo decise di stringere di nuovo i suoi rapporti con Marco Antonio, così formando il secondo triumvirato insieme a Lepido. Una delle conseguenze di ciò furono le liste di proscrizione: i triumviri decisero di eliminare dal principio i loro nemici politici e tra essi Marco Antonio fece scrivere il nome di Cicerone, nonostante lo scontento di Ottaviano. Marco Tullio Cicerone verrà ucciso il 7 dicembre del 43 a.C.\\

La sua testa e le sue mani furono mozzate ed esposte nel foro, dove Cicerone soleva proclamare i suoi discorsi. 

\part*{Le opere}

\section*{Gli strumenti dell'oratore}



\section*{Le opere retoriche}



\end{document}
