\documentclass[10pt,a4paper]{article}
\usepackage[utf8]{inputenc}
\usepackage[T1]{fontenc}
\usepackage{amsmath}
\usepackage{amsfonts}
\usepackage{amssymb}
\usepackage{makeidx}
\usepackage{graphicx}
\usepackage[left=1.00in, right=1.00in, top=1.00in, bottom=1.00in]{geometry}
\author{Tommaso Severini}
\title{Letteratura latina - Cicerone}
\begin{document}
	\maketitle

Cicerone nasce ad Arpino, cittadina provinciale legata alla tradizione contadina e che diede alla luce Gaio Mario. Tullio non era un uomo nobile e fu il primo membro della sua famiglia ad intraprendere il \textit{cursus honorum}. Egli intraprende studi di tipo letterario, filosofico e legale, divenendo un grande oratore. Uno delle sue prime orazione si rivolse contro un liberto del dittatore Silla, persona sicuramente non di poco conto e che aveva fatto scrivere le famose \textit{liste di proscrizione}. Per questo motivo, l'anno successivo andò in Grecia, ma, immediatamente dopo la morte del dittatore, tornò a Roma e divenne questore. Ciò gli permette di essere amato dal popolo e, cosa più importante, di aiutare il popolo a liberarsi del governatore Verre, che era accusato di aver rubato del denaro pubblico (le orazioni contro Verre prendono il nome di Verrine).  
\end{document}
