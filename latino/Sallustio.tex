\documentclass[10pt,a4paper]{article}
\usepackage[utf8]{inputenc}
\usepackage[T1]{fontenc}
\usepackage{amsmath}
\usepackage{amsfonts}
\usepackage{amssymb}
\usepackage{makeidx}
\usepackage{graphicx}
\usepackage[left=1.00in, right=1.00in, top=1.00in, bottom=1.00in]{geometry}
\author{Tommaso Severini}
\title{Letteratura latina - Sallustio}
\parindent0ex

\begin{document}
	\maketitle
	
	\section*{Biografia}
	
	Gaio Sallustio Crispo fu un senatore e storico romano vissuto nel I secolo a.C. appartenente al partito dei populares e convinto \textbf{filocesariano}.
	
	Intraprende il \textit{cursus honorum} fino a divenire senatore e partecipante nella prima guerra civile al fianco di Cesare, che lo supporterà in seguito.
	
	Nel 52 a.C. partecipò come accusatore al processo contro Milone, difeso da Cicerone.
	
	Sallustio, nel 50 a.C., fu accusato di corruzione e fu espulso dal senato \textit{probri causa}, ma fu subito riammesso grazie ai rapporti col futuro dittatore. Egli assunse anche il ruolo di propretore in Numidia, dove riuscì ad arricchirsi molto. Dopo la vittoria di Cesare nella guerra civile contro Pompeo, aiutò le truppe cesariane in Numidia ed ottenne il controllo di quella provincia, al tempo conosciuta come \textit{Africa Nova}.
	
	Nel 44 a.C., dopo la morte di Giulio Cesare, decide di ritirarsi dalla carriera politica e di dedicarsi all'\textit{otium letterarium}. Morì indisturbato nel 35 a.C. 
	
	\section*{Storiografia}
	
	
	
\end{document}