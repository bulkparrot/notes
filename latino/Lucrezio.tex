\documentclass[10pt,a4paper]{article}
\usepackage[utf8]{inputenc}
\usepackage[T1]{fontenc}
\usepackage{amsmath}
\usepackage{amsfonts}
\usepackage{amssymb}
\usepackage{makeidx}
\usepackage{graphicx}

\usepackage{xcolor}     % for colour
\usepackage{mdframed}   % for framing

\usepackage[left=1.00in, right=1.00in, top=1.00in, bottom=1.00in]{geometry}
\author{Tommaso Severini}
\title{Letteratura latina -  Lucrezio}
\begin{document}
	\maketitle
	
	\newmdtheoremenv[%
	linecolor=gray,leftmargin=60,%
	rightmargin=40,
	backgroundcolor=gray!40,%
	innertopmargin=5pt,%
	font=\ttfamily]{estratto}{Extract}[section]
	
	Lucrezio vive nel I secolo a.C. e, come altri sui predecessori come Catullo, non parla di politica nelle sue opere, ma si dedica alla filosofia.
	
	Tetrafarmacon:
	\begin{itemize}
		\item contro il timore degli dei (esistono ma vivono negli \textit{intermundia})
		\item contro il timore della morte (quando c'è la morte, noi non ci siamo; quando noi ci siamo la morte non c'è)
		\item contro il dolore
		\item spiega come si debba perseguire l'amore
	\end{itemize}

	Clinamen: inclinazione casuale che gli atomi acquistano quando si urtano.
	I corpi si ottengono dalle collisioni casuali di atomi. \\
	
	L'amore è considerato un aspetto negativo della vita. \\
	
	Catastematico: piacere che si presenta con l'assenza di dolore e turbamento. \\
	
	\section{Biografia}
	
	Si sa poco, egli non fu molto apprezzato poichè non partecipava alla vita politica (late biosas), era un sostenitore del materialismo. Nell'epoca successiva, San Girolamo scrisse della sua vita e ci racconta che morì intorno ai 30 anni e che morì pazzo. Ciò però non è attendibile, ma il fatto che sia morto da giovane è probabile. L'unico che apprezza Lucrezio fu Cicerone, che ammira il lavoro filosofico scritto in versi. Nei 6 libri del suo \textit{De rerum natura}, Lucrezio riassume tutta la filosofia epicurea, attuando una grande opera di \textit{brevitas}. Lucrezio crede che scrivere un manuale in versi sia utile per avvicinare i lettori alla filosofia.
	
	\section{De rerum natura}
	
	La sua struttura è:
	
	\begin{itemize}
		\item La fisica, i primi due libri parlano della teoria atomistica da lui ripresa, con la spiegazione di concetti fondamentali come il clinamen. Si apre con l'inno a Venere.
		\item L'antropologia, i libri III e IV sono usati per narrare della teoria evolutiva di epicuro, che anticipa di diversi secoli quella di Darwin. Anche la trasformazione dell'uomo ha un ruolo fondamentale. 
		\item La cosmologia, gli ultimi due libri narrano della mortalità del mondo e dell'uomo e della peste di Atene
	\end{itemize}

	\subsection{Inno a venere}
	
	Il primo libro del "De rerum natura", appartenente alla diade della "Fisica", si apre con una loda alla dea Venere. Nonostante ciò possa sembrare contraddittorio, poichè Lucrezio va contro la religione tradizionale, non lo è. Infatti, la Venere descritta dal poeta non è quella lodata dai Romani, ma un'allegoria della forza generatrice della natura e dell'energia che rinnova ogni essere. La sua forza è talmente possente che ella riesce a far soccombere Marte, dio che rappresenta la forza distruttrice della guerra. 
	
		\begin{estratto}
\texttt{	Aeneadum genetrix, hominum divomque voluptas, 1\\
			alma Venus, caeli subter labentia signa \\
			quae mare navigerum, quae terras frugiferentis \\
			concelebras, per te quoniam genus omne animantum \\
			concipitur visitque exortum lumina solis: 5}
		\end{estratto}
	
	In questi primi versi dell'opera, va posto in evidenza in verbo "concelebras" (riempi di vita), posto in posizione di rilievo ad inizio di verso per enfatizzare la funzione fondamentale di Venere. Va anche evidenziato l'aggettivo "alma", spesso connesso al verbo "alo" (nutrire, alimentare), riferito a Venere in quanto principio di vita. Infine, all'ultimo verso, va posta attenzione sui lessemi "concipitur", "visitque" e "exortum", che descrivono i tre momenti della vita di ogni uomo (concepimento, nascita e visione della luce).
	
	\begin{estratto}
		te, dea, te fugiunt venti, te nubila caeli 6\\
		adventumque tuum, tibi suavis daedala tellus \\
		summittit flores, tibi rident aequora ponti \\
		placatumque nitet diffuso lumine caelum. 9
	\end{estratto}

	In questi versi è possibile osservare gli effetti che la dea ha sulla natura. L'utilizzo di molte forme vocative, epiteti e l'insistenza sul pronome personale "tu", ricorda molto le caratteristiche delle tradizionali lodi alle divinità.
	
	\begin{estratto}
		denique per maria ac montis fluviosque rapacis 17  \\
		frondiferasque domos avium camposque virentis \\
		omnibus incutiens blandum per pectora amorem \\
		efficis ut cupide generatim saecla propagent. \\           
		quae quoniam rerum naturam sola gubernas \\
		nec sine te quicquam dias in luminis oras \\
		exoritur neque fit laetum neque amabile quicquam, \\
		te sociam studeo scribendis versibus esse, \\
		quos ego de rerum natura pangere conor \\        
		Memmiadae nostro, quem tu, dea, tempore in omni \\
		omnibus ornatum voluisti excellere rebus.	27
	\end{estratto}

	In questi versi è possibile osservare altri effetti di Venere sulla natura, come, ad esempio, la brama d'amore evidenziata dall'avverbio "cupide". Va inoltre posta attenzione sull'avverbio "generatim", che fa riferimento alla filosofia epicurea, descrivendo come le specie viventi si riproducano mantenendo le proprie caratteristiche. Infine va posta evidenza sulle parole "de rerum natura", titolo dell'opera, che ne descrive l'argomento.
	
		\begin{estratto}
			effice ut interea fera moenera militiaim\\
			per maria ac terras omnis sopita quiescant;               30\\
			nam tu sola potes tranquilla pace iuvare\\
			mortalis, quoniam belli fera moenera Mavors\\
			armipotens regit, in gremium qui saepe tuum se\\
			reiicit aeterno devictus vulnere amoris,\\
			atque ita suspiciens tereti cervice reposta               35\\
			pascit amore avidos inhians in te, dea, visus\\
			eque tuo pendet resupini spiritus ore.\\
			hunc tu, diva, tuo recubantem corpore sancto\\
			circum fusa super, suavis ex ore loquellas\\
			funde petens placidam Romanis, incluta, pacem;               40\\
		\end{estratto}
	
	In questi versi, bisogna, innanzitutto, porre in evidenza l'aggettivo sola al v. 31 che indica il ruolo insostituibile di Venere. Bisogna inoltre tenere in considerazione la continua contrapposizione tra Marte e Venere, elemento tipico della letteratura.
	
	\section{Domande}
	
	\subsection{Domandona}
	
	Il \textit{De rerum natura} è l'unica opera scritta da Tito Lucrezio Caro. Quest'ultimo, insieme a Catullo, è il maggiore esponente della poetica del I secolo a.C., periodo in cui è più largamente utilizzata la prosa da autori come Cicerone, Cesare e Sallustio. Oltre alla forma metrica scelta, il fattore che colpisce i lettori è il suo argomento: la filosofia epicurea, guardata con molto sospetto a Roma in quel periodo. Queste scelte molto singolari per l'epoca sono dovute al fatto che Lucrezio aveva intenzione di tradurre, adattare e riassumere in latino la \textit{opus magna} di Epicuro: la sua "Sulla natura"(da cui viene preso anche il nome latino dell'opera), dove il filosofo greco ha trascritto il suo pensiero in 37 libri. Infatti, Lucrezio crede che l'utilizzo dei versi per spiegare un argomento talmente complesso come la filosofia possa avvicinare più facilmente i lettori a questa oscura materia. Come analogia, nel libro I, Lucrezio paragona ciò al porre del miele sul bordo di un calice per far bere ai fanciulli una medicina amara.
	
	L'epicureismo si proponeva di liberare gli uomini dalle paure e dal dolore per poter raggiungere una felicità catastematico ,piacere statico, consistente nella pura assenza di sofferenza e turbamento. 
	Il principale strumento di questa filosofia era sicuramente il \textit{tetrafarmacon}, insieme di quattro proposizioni che rappresentano i cardini del pensiero di Epicuro. Essi sono l'avversità al timore negli dei, al dolore, alla paura della morte e la ricerca della felicità.
	Oltre a ciò, un altro valore epicureo che Lucrezio sosteneva e predicava è quello del \textit{lathe biosas}, letteralmente "vivi nascosto", che consisteva nel distaccarsi completamente dai problemi della società per poterla osservare dall'esterno (per questo motivo molto criticato dai romani).
	
	L'intera opera è suddivisa in 6 libri, a loro volta classificati in 3 diadi: la "fisica", l'"antropologia" e la "cosmologia". La prima coppia di libri tratta della teoria atomistica di Epicuro e descrive come l'uomo debba comportarsi per raggiungere l'\textbf{atarassia}, la seconda espone la definizione di anima, il fatto che essa sia mortale e la teoria dei \textit{simulacra} e la terza espone la storia del mondo e dell'uomo, l'evoluzione di quest'ultimo, anticipando la teoria evolutiva di Darwin, e le teorie riguardanti le divinità.
	
	Il proemio del grande poema didascalico di Lucrezio, il De rerum natura, è fondamentale per introdurre i temi fondamentali alla base dell’operazione letteraria, filosofica e culturale che il poeta si prefigge con la propria opera.L’apertura del testo è così dedicata all’invocazione a Venere (“Aeneadum genetrix”), dea dell’amore e principio vitale della continuità dell’esistenza, a cui Lucrezio chiede di poterlo aiutare a trasmettere e spiegare la filosofia epicurea, di poter garantire a Roma un periodo di pace in un periodo storico in cui la realtà romana è frastagliata da guerre civili. Infine conclude il proemio con un elogio al destinatario ideale dell'opera, nonchè suo protettore, Gaio Memmio. Tutti questi elementi prendono ispirazione dai modelli classici e, in particolare, dagli inni cletici.
	
	Può stupire che Lucrezio, forte oppositore della \textit{religio}, decida di dedicare la prima parte della sua opera alla lode di una divinità, ma ciò non è corretto. Infatti, la dea descritta da Lucrezio non è quella lodata dai Romani, ma deve essere solo il simbolo della forza creatrice e pacificatrice del mondo, che dona vita e che, con la sua forza, riesce a sottomettere anche il dio Marte, simbolo di guerra, morte e disgregazione.
	
	\subsection{Domanda 1}
	
	Uno delle tematiche più importanti che Lucrezio affronta nella sua opera è sicuramente la lotta contro la \textit{religio}, che egli vedeva solo come pratiche superstiziose che allontano gli uomini dalla ricerca della verità. Come esempio per dimostrare l'assurdità della religione romana è rappresentato dal racconto del sacrificio di Ifigenia, figlia di Agamennone. 
	Oltre a scegliere di utilizzare un \textit{exemplum} epico per supportare le sue argomentazioni, egli decide di narrare la versione più drammatica del mito, quella in cui Ifigenia viene sacrificata invano, sminuendo così la figura dell'indovino. Per rendere la scena ancora più suggestiva e crudele, Lucrezio descrive l'ambiente e le azioni in modo molto dettagliato. La vicenda si conclude con un commento molto amaro da parte del poeta.Questo episodio fa trasparire esattamente il messaggio che Lucrezio voleva lasciare al lettore: la \textit{religio} rende ciechi, terrorizza e minaccia gli uomini. Nonostante ciò, Lucrezio loda un uomo, che egli considera un eroe, per essere il primo ad aver lottato per riottenere la propria libertà utilizzando la ragione, Epicuro. Lucrezio ammira talmente tanto il suo eroe da dedicargli quattro elogi nella sua opera, di cui il primo prima della vicenda del sacrificio di Ifigenia. Questa lode si apre con la terrificante immagine della vita letteralmente calpestata dalla \textit{religio}, mostro gigantesco che grava sugli uomini. Tra questi, però, si erge una figura coraggiosa che per primo lotta per liberare l'umanità(ribadito diverse volte da vocaboli come \textit{primum, primus, primus...}). Epicuro, quindi, diviene non solo un eroe, ma anche il salvatore del genere umano e un titano, che riesce ad aprire le porte della conoscenza fino a quel momento serrate.
	
	\subsection{Domanda 2}
	
	\subsubsection{Teoria atomistica}
	
	Con la fisica ha quindi inizio l'esposizione della dottrina epicurea. Lucrezio fissa anzitutto il principio che nulla nasce dal nulla e ogni essere è costituito da una particolare aggregazione di elementi fini, semplici e si forma secondo specifiche modalità di tempi e di ambienti, con esclusione di ogni intervento divino. Niente può essere generato dal niente, altrimenti tutti gli esseri nascerebbero a caso e ciò che è generato sarebbe indipendente dal generante. Allo stesso modo nulla si riduce al nulla, dal momento che l'annullamento delle cose sarebbe spontaneo e non ci sarebbe bisogno di cause disgregatrici, mentre la distruzione ha bisogno di forze disgreganti, proporzionate alla dissolubilità delle cose. Se avvenisse l'annullamento, i corpi non potrebbero più rinnovarsi; nascita e morte delle cose è invece aggregamento e disgregamento di parti, perciò esiste una materia fondamentale ed eterna, la quale è costituita da corpi minimi e invisibili. Gli atomi sono semplici e indissolubili, perché non contengono vuoto e formano un complesso omogeneo e indivisibile di minimi indifferenziati oltre i quali è il nulla. I minimi per se stessi non potrebbero esistere: si trovano nell'atomo che è un insieme omogeneo e coerente di queste particelle e dal numero e dalla posizione dei minimi derivano le varie forme degli atomi. Gli elementi primari sono dunque solidi e semplici, tutti assai piccoli uniti da forti legami: non sono miscugli di cose, ma unità definite elementari a cui nulla si può aggiungere o togliere, compatti e immortali. Quando un corpo finisce il proprio ciclo vitale, essi ritornano a essere liberi e si scambiano con altri.
	
	\subsubsection{\textit{Clinamen}}
	
	Gli atomi si muovono appunto in modo travagliato e incessante con la stessa velocità, secondo la teoria del clinamen; infatti nel cadere verticalmente, trascinati dal proprio peso, nel vuoto, deviano leggermente e anziché precipitare in basso, danno luogo a scontri permettendo alla natura di creare le cose. 
	
	\subsubsection{Sensazioni e \textit{simulacra}}
	
	I corpuscoli primordiali delle cose hanno forme e figure molto diverse, poiché ognuno di essi ricerca quella che gli è “propria e ben nota” e grazie a questa caratteristica generano le sensazioni. Le particelle moleste e aspre come quelle “dell'assenzio e dell'acre centaurea”, sono strutturate dalla ruvidezza della materia, per questo, penetrando, lacerano il corpo e creano una sensazione sgradevole. Invece quelle che accarezzano i sensi, cioè quelle rotonde e buone al tatto sono formate dalla levigatezza della materia. Esistono poi dei corpuscoli che non sono né pungenti, né piacevolmente levigati, e che provocano solo “solletico” ai sensi. Paragonandoli a “cortecce” o “pellicole” (membranae vel cortex) staccate via dalla superficie dei corpi che volano in giro per l'aria e terrorizzano la nostra mente apparendoci nel sonno e nella veglia, i simulacri, atomi sottilissimi, si distaccano dalle cose o dai corpi e vanno a colpire i nostri sensi.
	
	\subsubsection{Le divinità}
	
	 Lucrezio nega il concetto di provvidenza e considera insensato qualunque timore verso gli dèi, i quali, inconsapevoli dell'esistenza dell'uomo, non si interessano delle sue azioni; infatti godono di un'eterna felicità e vivono fuori del mondo, negli intermundia.
	
	\subsubsection{Teoria evolutiva}
	
	 si parla del processo evolutivo dell'uomo, dalla sua prima comparsa sino alla civilizzazione. Gli uomini primitivi si sono evoluti sia per motivi convenzionali, come il linguaggio, sia per la semplice osservazione del mondo circostante, ad esempio la scoperta del fuoco e l'utilità dell'agricoltura. Grande importanza viene attribuita ai metalli che hanno garantito uno sviluppo delle capacità tecniche e conoscitive della specie; grazie all'insegnamento della natura l'uomo è riuscito a scoprire e a lavorare i metalli sia come utensili sia come armi. Per l'autore l'oro è simbolo di corruzione morale e decadenza, per questo considera l'età dell'oro esiodea peggiore di quella primitiva, anteponendo i beni naturali e necessari a quelli materiali. A fianco all'uomo, anche gli altri esseri viventi sono stati fin dall'inizio soggetti a una “selezione naturale”: tutte le specie che occupavano una posizione eminente hanno perpetuato la loro stirpe, mentre quelle incapaci di sopravvivere si sono estinte. Da ultimo viene trattato il confronto tra la civiltà primitiva e quella odierna; ne emerge che, col tempo, l'uomo ha preferito soddisfare il proprio benessere personale oltre ai bisogni primari. Brama di potere, avidità di ricchezze, guerre hanno causato paure d'ogni tipo, fino a una degenerazione della società.
	 
	 \subsubsection{Etica}
	 
	 Se, dice sostanzialmente Lucrezio, i movimenti (e gli avvenimenti) si succedessero in un rigoroso ordine deterministico non si potrebbe spiegare la libera voluntas dell'uomo.
	 È solo grazie al fenomeno del clinamen dunque che la nostra volontà è libera, avolsa fatis, sottratta ai fati.
	
	\subsection{Domanda 3}
	
	La filosofia di Epicuro si proponeva di rimuovere tutti gli ostacoli che impediscono all'uomo il conseguimento della felicità, tra cui la paura della morte. Lucrezio il modo di eliminare questo timore attraverso una corretta conoscenza della costituzione fisica del mondo e varie argomentazioni logiche.
	
	Argomentazioni contro la morte:
	\begin{enumerate}
		\item Non si percepisce nè piacere nè dolore durante la morte.
		\item Quando l'anima si distacca dal corpo, non è più essere e non può provare sensazioni (l'essere cosciente è formato da anima e corpo).
		\item Anche se l'anima ritornasse a far parte di un corpo, non conserverebbe sensazioni e ricordi della sua vita precedente.
	\end{enumerate} 
	Tutto ciò viene sintetizzato da Epicuro:"La morte non è nulla per noi, perché quando ci siamo noi non c'è lei, e quando c'è lei non ci siamo più noi."
	Oltre a eliminare la paura della morte, Lucrezio vuole anche insegnare ai suoi lettori di apprezzare la vita e viverne ogni momento, riprendendo la filosofia del \textit{Carpe diem}. Ciò viene realizzato attraverso una serie di argomentazioni logiche:
	\begin{enumerate}
		\item Prosopopea della natura, che interroga l'uomo
		\item Se si è felici della propria vita, non c'è bisogno di lamentarsi di essa
		\item Se la vita è dolorosa, bisogna essere grati che si diventerà liberi dal male attraverso la morte
		\item Se si è ossessionati dall'idea e dall'avvicianrsi della morte, si sta sprecando la propria esistenza.
		\item Il ciclo della vita, che inizia donando la vita e si conclude con la venuta della morte, è un processo naturale, quindi non deve spaventare.
	\end{enumerate}
	Dopo di che, Lucrezio espone il suo scetticismo nei confronti dell'inferno. Egli, infatti, crede che esso non esista e le atrocità che vivono i dannati non sono altro che allegorie dei peggiori peccati dell'uomo. Secondo Lucrezio, l'unico inferno esistente è rappresentato dalle sofferenze terrene. Per le correlazioni tra peccati umani e peccatori mitologici vedere pagina 301.
	
\end{document}