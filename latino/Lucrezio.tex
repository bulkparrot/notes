\documentclass[10pt,a4paper]{article}
\usepackage[utf8]{inputenc}
\usepackage[T1]{fontenc}
\usepackage{amsmath}
\usepackage{amsfonts}
\usepackage{amssymb}
\usepackage{makeidx}
\usepackage{graphicx}
\usepackage[left=1.00in, right=1.00in, top=1.00in, bottom=1.00in]{geometry}
\author{Tommaso Severini}
\title{Letteratura latina -  Lucrezio}
\begin{document}
	\maketitle
	
	Lucrezio vive nel I secolo a.C. e, come altri sui predecessori come Catullo, non parla di politica nelle sue opere, ma si dedica alla filosofia
	
	Tetrafarmacon:
	\begin{itemize}
		\item contro il timore degli dei (esistono ma vivono negli \textit{intermundia})
		\item contro il timore della morte (quando c'è la morte, noi non ci siamo; quando noi ci siamo la morte non c'è)
		\item contro il dolore
		\item spiega come si debba perseguire l'amore
	\end{itemize}

	clinamen: inclinazione casuale che gli atomi acquistano quando si urtano
	I corpi si ottengono dalle collisioni casuali di atomi
	
	L'amore asapetto negativo della vita
	
	Catastematico, piacere che si presenta con l'assenza di dolore e turbamento
	
	\section{Biografia}
	
	Si sa poco, egli non fu molto apprezzato poichè non partecipava alla vita politica (late biosas), era un sostenitore del materialismo. Nell'epoca successiva, San Girolamo scrisse della sua vita e ci racconta che morì intorno ai 30 anni e che morì pazzo. Ciò però non è attendibile, ma il fatto che sia morto da giovane è probabile. L'unico che apprezza Lucrezio fu Cicerone, che ammira il lavoro filosofico scritto in versi. Nei 6 libri del suo \textit{De rerum natura}, Lucrezio riassume tutta la filosofia epicurea, attuando una grande opera di \textit{brevitas}. Lucrezio crede che scrivere un manuale in versi sia utile per avvicinare i lettori alla filosofia.
	
	\section{De rerum natura}
	
	La sua struttura è:
	
	\begin{itemize}
		\item La fisica, i primi due libri parlano della teoria atomistica da lui ripresa, con la spiegazione di concetti fondamentali come il clinamen. Si apre con l'inno a Venere.
		\item L'antropologia, i libri III e IV sono usati per narrare della teoria evolutiva di epicuro, che anticipa di diversi secoli quella di Darwin. Anche la trasformazione dell'uomo ha un ruolo fondamentale. 
		\item La cosmologia, gli ultimi due libri narrano della mortalità del mondo e dell'uomo e della peste di Atene
	\end{itemize}
	
\end{document}