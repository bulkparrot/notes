\documentclass[10pt,a4paper]{article}
\usepackage[utf8]{inputenc}
\usepackage[T1]{fontenc}
\usepackage{amsmath}
\usepackage{amsfonts}
\usepackage{amssymb}
\usepackage{makeidx}
\usepackage{graphicx}

\usepackage{xcolor}     % for colour
\usepackage{mdframed}   % for framing

\usepackage[left=1.00in, right=1.00in, top=1.00in, bottom=1.00in]{geometry}
\author{Tommaso Severini}
\title{Letteratura latina -  Lucrezio}
\begin{document}
	\maketitle
	
	\newmdtheoremenv[%
	linecolor=gray,leftmargin=60,%
	rightmargin=40,
	backgroundcolor=gray!40,%
	innertopmargin=5pt,%
	font=\ttfamily]{estratto}{Extract}[section]
	
	Lucrezio vive nel I secolo a.C. e, come altri sui predecessori come Catullo, non parla di politica nelle sue opere, ma si dedica alla filosofia.
	
	Tetrafarmacon:
	\begin{itemize}
		\item contro il timore degli dei (esistono ma vivono negli \textit{intermundia})
		\item contro il timore della morte (quando c'è la morte, noi non ci siamo; quando noi ci siamo la morte non c'è)
		\item contro il dolore
		\item spiega come si debba perseguire l'amore
	\end{itemize}

	Clinamen: inclinazione casuale che gli atomi acquistano quando si urtano.
	I corpi si ottengono dalle collisioni casuali di atomi. \\
	
	L'amore è considerato un aspetto negativo della vita. \\
	
	Catastematico: piacere che si presenta con l'assenza di dolore e turbamento. \\
	
	\section{Biografia}
	
	Si sa poco, egli non fu molto apprezzato poichè non partecipava alla vita politica (late biosas), era un sostenitore del materialismo. Nell'epoca successiva, San Girolamo scrisse della sua vita e ci racconta che morì intorno ai 30 anni e che morì pazzo. Ciò però non è attendibile, ma il fatto che sia morto da giovane è probabile. L'unico che apprezza Lucrezio fu Cicerone, che ammira il lavoro filosofico scritto in versi. Nei 6 libri del suo \textit{De rerum natura}, Lucrezio riassume tutta la filosofia epicurea, attuando una grande opera di \textit{brevitas}. Lucrezio crede che scrivere un manuale in versi sia utile per avvicinare i lettori alla filosofia.
	
	\section{De rerum natura}
	
	La sua struttura è:
	
	\begin{itemize}
		\item La fisica, i primi due libri parlano della teoria atomistica da lui ripresa, con la spiegazione di concetti fondamentali come il clinamen. Si apre con l'inno a Venere.
		\item L'antropologia, i libri III e IV sono usati per narrare della teoria evolutiva di epicuro, che anticipa di diversi secoli quella di Darwin. Anche la trasformazione dell'uomo ha un ruolo fondamentale. 
		\item La cosmologia, gli ultimi due libri narrano della mortalità del mondo e dell'uomo e della peste di Atene
	\end{itemize}

	\subsection{Inno a venere}
	
	Il primo libro del "De rerum natura", appartenente alla diade della "Fisica", si apre con una loda alla dea Venere. Nonostante ciò possa sembrare contraddittorio, poichè Lucrezio va contro la religione tradizionale, non lo è. Infatti, la Venere descritta dal poeta non è quella lodata dai Romani, ma un'allegoria della forza generatrice della natura e dell'energia che rinnova ogni essere. La sua forza è talmente possente che ella riesce a far soccombere Marte, dio che rappresenta la forza distruttrice della guerra. 
	
		\begin{estratto}
\texttt{	Aeneadum genetrix, hominum divomque voluptas, 1\\
			alma Venus, caeli subter labentia signa \\
			quae mare navigerum, quae terras frugiferentis \\
			concelebras, per te quoniam genus omne animantum \\
			concipitur visitque exortum lumina solis: 5}
		\end{estratto}
	
	In questi primi versi dell'opera, va posto in evidenza in verbo "concelebras" (riempi di vita), posto in posizione di rilievo ad inizio di verso per enfatizzare la funzione fondamentale di Venere. Va anche evidenziato l'aggettivo "alma", spesso connesso al verbo "alo" (nutrire, alimentare), riferito a Venere in quanto principio di vita. Infine, all'ultimo verso, va posta attenzione sui lessemi "concipitur", "visitque" e "exortum", che descrivono i tre momenti della vita di ogni uomo (concepimento, nascita e visione della luce).
	
	\begin{estratto}
		te, dea, te fugiunt venti, te nubila caeli 6\\
		adventumque tuum, tibi suavis daedala tellus \\
		summittit flores, tibi rident aequora ponti \\
		placatumque nitet diffuso lumine caelum. 9
	\end{estratto}

	In questi versi è possibile osservare gli effetti che la dea ha sulla natura. L'utilizzo di molte forme vocative, epiteti e l'insistenza sul pronome personale "tu", ricorda molto le caratteristiche delle tradizionali lodi alle divinità.
	
	\begin{estratto}
		denique per maria ac montis fluviosque rapacis 17  \\
		frondiferasque domos avium camposque virentis \\
		omnibus incutiens blandum per pectora amorem \\
		efficis ut cupide generatim saecla propagent. \\           
		quae quoniam rerum naturam sola gubernas \\
		nec sine te quicquam dias in luminis oras \\
		exoritur neque fit laetum neque amabile quicquam, \\
		te sociam studeo scribendis versibus esse, \\
		quos ego de rerum natura pangere conor \\        
		Memmiadae nostro, quem tu, dea, tempore in omni \\
		omnibus ornatum voluisti excellere rebus.	27
	\end{estratto}

	In questi versi è possibile osservare altri effetti di Venere sulla natura, come, ad esempio, la brama d'amore evidenziata dall'avverbio "cupide". Va inoltre posta attenzione sull'avverbio "generatim", che fa riferimento alla filosofia epicurea, descrivendo come le specie viventi si riproducano mantenendo le proprie caratteristiche. Infine va posta evidenza sulle parole "de rerum natura", titolo dell'opera, che ne descrive l'argomento.
	
\end{document}