\documentclass[10pt,a4paper]{article}
\usepackage[utf8]{inputenc}
\usepackage[T1]{fontenc}
\usepackage{amsmath}
\usepackage{amsfonts}
\usepackage{amssymb}
\usepackage{makeidx}
\usepackage{graphicx}

\usepackage{xcolor}     % for colour
\usepackage{mdframed}   % for framing


\usepackage[left=1.00in, right=1.00in, top=1.00in, bottom=1.00in]{geometry}
\author{Tommaso Severini}
\title{Letteratura latina - Cesare}
\begin{document}
	\maketitle
	
	\newmdtheoremenv[%
	linecolor=gray,leftmargin=60,%
	rightmargin=40,
	backgroundcolor=gray!40,%
	innertopmargin=5pt,%
	font=\ttfamily]{estratto}{Extract}[section]
	
	\section{De bello gallico, Liber I}
		
	\begin{quote}
		Gallia est omnis divisa in partes tres, quarum unam incolunt Belgae, aliam Aquitani, tertiam qui ipsorum lingua Celtae, nostra Galli appellantur. 2 Hi omnes lingua, institutis, legibus inter se differunt. Gallos ab Aquitanis Garumna flumen, a Belgis Matrona et Sequana dividit. 3 Horum omnium fortissimi sunt Belgae, propterea quod a cultu atque humanitate provinciae longissime absunt, minimeque ad eos mercatores saepe commeant atque ea quae ad effeminandos animos pertinent important, 4 proximique sunt Germanis, qui trans Rhenum incolunt, quibuscum continenter bellum gerunt. Qua de causa Helvetii quoque reliquos Gallos virtute praecedunt, quod fere cotidianis proeliis cum Germanis contendunt, cum aut suis finibus eos prohibent aut ipsi in eorum finibus bellum gerunt. 5 Eorum una pars, quam Gallos obtinere dictum est, initium capit a flumine Rhodano, continetur Garumna flumine, Oceano, finibus Belgarum, attingit etiam ab Sequanis et Helvetiis flumen Rhenum, vergit ad septentriones. 6 Belgae ab extremis Galliae finibus oriuntur, pertinent ad inferiorem partem fluminis Rheni, spectant in septentrionem et orientem solem. 7 Aquitania a Garumna flumine ad Pyrenaeos montes et eam partem Oceani quae est ad Hispaniam pertinet; spectat inter occasum solis et septentriones.
	\end{quote}

	\subsection{Analisi del periodo}
	
	Gallia est omnis divisa in partes tres = principale rggente \\
	quarum unam incolunt Belgae, aliam Aquitani, tertiam qui = sub. relativa pr.\\
	qui ipsorum lingua Celtae, nostra Galli appellantur = sub. relativa propria\\
	Hi omnes lingua, institutis, legibus inter se differunt = principale.\\
	Gallos ab Aquitanis Garumna flumen, a Belgis Matrona et Sequana dividit = princ. \\
	Horum omnium fortissimi sunt Belgae = principale reggente\\
	propterea quod a cultu atque humanitate provinciae longe absunt = sub. causale\\
	minimeque ad eos mercatore saepe commeant = coord. alla principale\\
	nec ea important = coord. alla precedente\\
	quae ad mollitiam animorum pertinent = sub. relativa propria\\
	Proximique sunt Germani = prncipale reggente\\
	qui trans Rhenum incolunt = sub. relativa\\
	hac de causa Helvetii quoque reliquos Gallos virtute praecedunt = princ. reggente\\
	quod fere cotidianis proeliis cum Germanis contendunt = sub. causale\\
	cum aut a suis finibus eos prohibent = sub. temporale\\
	aut ipsi in eorum finibus bellum gerunt = coord. alla temporale\\
	Eorum una pars initium capit a flumine Rhodano = principale reggente\\
	quam Gallos obtinere dictum est = sub.relativa propria\\
	continetur Garumna flumine, Oceano, finibus Belgarum = coord. alla reggente\\
	sed attingit flumen Rhenum = coordinata alla precedente\\
	et vergit ad septemtriones = coordinata alla precedente\\
	Belgae ab extremis Galliae finibus oriuntur = principale\\
	pertinent ad inferiorem partem fluminis Rheni = coordinata per asindeto alla princ.\\
	spectant in septentrionem et orientem solem = coordinata per asindeto alla princ.\\

\end{document}