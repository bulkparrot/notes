\documentclass[10pt,a4paper]{article}
\usepackage[utf8]{inputenc}
\usepackage[T1]{fontenc}
\usepackage{amsmath}
\usepackage{amsfonts}
\usepackage{amssymb}
\usepackage{makeidx}
\usepackage{graphicx}

\usepackage{xcolor}     % for colour
\usepackage{mdframed}   % for framing


\usepackage[left=1.00in, right=1.00in, top=1.00in, bottom=1.00in]{geometry}
\author{Tommaso Severini}
\title{Letteratura latina - Cesare}
\begin{document}
	\maketitle

	\newmdtheoremenv[%
	linecolor=gray,leftmargin=60,%
	rightmargin=40,
	backgroundcolor=gray!40,%
	innertopmargin=5pt,%
	font=\ttfamily]{estratto}{Extract}[section]

	\section{Contestualizzazione storica}

	\subsection{Personaggi chiave}

	\begin{itemize}
		\item Marco Tullio Cicerone, importante oratore della fazione degli "optimates"
		\item Gaio Giulio Cesare, uomo di nobile famiglia e stratega militare legato alla fazione dei "populares"
		\item Pompeo, esponente degli "optimates", combattè contro Mitridate
	\end{itemize}

	\subsection{Storia di Caius Iulius Caesar}

	\subsubsection{100 a.C.}

	Nasce Giulio Cesare dalla \textit{gens Iulia}, che faceva risalire le proprie origini ad Enea.

	\subsubsection{78 a.C.}

	Durante il regime di Silla è iscritto nelle liste di proscrizione e si rifugia in Asia; potrà ritornare solo dopo la morte di Silla.

	\subsubsection{77 a.C.}

	Diviene un oratore e nel 69 a.C. comincia il suo \textit{cursus honorum} diventando questore in Spagna.

	\subsubsection{60 a.C.}

	Forma il primo triumvirato, insieme a Pompeo e Crasso. L'anno successivo diviene console e nel 58 ottiene il proconsolato in Gallia, supervisionando i territori di \textbf{
	Gallia cisalpina e narbonense}.

	\subsubsection{Accordi di Lucca}

	I triumviri rinnovano il loro accordo e il proconsolato di Cesare viene prolungato di altri 5 anni.

	\subsubsection{58-51 a.C.}

	Cesare ottiene il controllo di tutta la Gallia attraverso una serie di guerre descritte nel suo commentario: il \textit{De bello gallico}.

	\subsubsection{49 a.C.}

	La situazione a Roma era molto delicata:

	Crasso, ottenuto il controllo della Siria, fu ucciso nel 53 a.C. in una battaglia contro i Parti.

	Pompeo, invece di essere rimasto in Spagna, si fece sostituire da dei legati e rimase a Roma per mitigare il clima politico teso della Roma del tempo. Per fare ciò, il senato elesse Pompeo come \textit{consul sine collega}, riavvicinandosi così al partito degli "optimates" e cercando di levare il prima possibile il potere a Cesare, in modo da farlo rientrare a Roma da privato cittadino.

	Cesare accetta di tornare a Roma da privato cittadino, ma solo nel caso in cui anche Pompeo avesse rinunciato al suo esercito. Pompeo, spaventato dal potere di Cesare, decide di non accettare le condizioni imposte da Cesare e quest'ultimo si vede costretto ad attraversare il fiume Rubicone, che costituiva il \textit{pomerium}, e a iniziare la guerra civile contro Pompeo, che nel frattempo era scappato in Puglia.

	\subsection{48 a.C.}

	Cesare sconfigge definitivamente l'esercito pompeiano a Farsàlo in Tessaglia e Pompeo decide di chiedere asilo al faraone Tolomeo XIII, che lo uccide tagliandogli la testa per cercare di ottenere l'appoggio di Cesare nella guerra civile tolemaica, ma invano.

	\subsubsection{46 a.C.}

	Cesare sconfigge le armate pompeiane rimanenti nelle battaglie di Taspo, in Africa, e di Munda, in Spagna. Tornato a Roma si fa nominare dal senato \textit{dictator perpetuus} a metà del 46 a.C.

	\subsubsection{44 a.C.}

	Il governo di Cesare, però, non durerà a lungo. Infatti, dopo una serie di riforme demagogiche che minavano il potere della classe nobiliare, durante la seduta del senato delle Idi di marzo del 44 a.C. viene pugnalato dai senatori presenti, tra cui suo (non tanto) figlio adottivo Marco Decimo Bruto.

	\subsection{Riforme politiche}

	\begin{itemize}
	\item Numero di senatori da 600 a 900

	\item Allargamento del ceto dirigente, permettendo a più persone di entrare a far parte dell'ordine equestre

	\item Aumento del numero dei magistrati, migliorando l'amministrazione

	\item Concede la cittadinanza romana agli abitanti della Gallia.
\end{itemize}

\subsection{Riforme sociali}

Durante il suo governo, Cesare attuò molte politiche a favori delle classi sociali meno abienti, per i debitori, i morosi e i soldati. Facendo ciò riuscì a guadagnare il consenso di una grande parte della popolazione, ma attirando così la preoccupazione dei nobili.

\section{Produzione letteraria}

Cesare fu un prolifico scrittore, oltre che grande uomo politico, e trattò diversi generi nelle sue opere anche se le uniche sue opere giunte a noi sono di carattere politico (i commentarii). Da punto di vista grammatico, egli era un forte sostenitore dell'\textbf{analogia} e dell'\textbf{atticismo}.

\subsection{Atticismo e asianesimo}

L'arte oratoria era divisa secondo due differenti stili: l'atticismo, che consisteva nella scrittura di frasi semplici e coincise, e l'asianesimo, che pridiligeva l'uso di molte figure retoriche e del \textit{pathos}. Cesare tendeva ad utilizzare maggiormente la prima, mentre altri grandi oratori come Cicerone utilizzarono uno stile medio, conosciuto come \textit{rodio}.

\subsection{Analogia e anomalia}

L'analogia e l'anomalia sono due tendenze stilistiche contrapposte: la prima crede che la lingua debba essere regolata da norme precise e il più razionale e pura possibile. L'anomalia, al contrario, sostiene che il linguaggio non debba essere regolato dal raziocinio e che debba seguire la sua vena "naturale"

\subsection{Corpus caesarianum}

\begin{itemize}
	\item De bello gallico
	\item De bello civili
	\item Belum Africum*
	\item Bellum Hispaniense*
	\item Bellum Alexandrinum*
\end{itemize}

\section{I commentarii}

I \textit{commentarii} cesariani appartengono ad un genere non esattamente definibile, ma che riprende diversi apetti da diversi generi letterari. In particolare, queste opere rappresentano un punto di incontro tra l'opera storiografica, che si propone di rappresentare la realtà in modo oggettivo e con le dovute premesse storiche, e le \textit{hypomnema}, appunti presi dai generali durante le battaglie. Inoltre, Cesare decide di dare ai suoi commentarii una valenza \textbf{annalistica}, che racconta gli eventi anno per anno, e una di tipo \textbf{monografico}, ovvero incentrato su un singolo argomento (nel primo caso della guerra in Gallia, nel secondo della guerra civile contro Pompeo Magno).

Per quanto riguarda lo \textbf{stile}, come scrive Cicerone nel suo \textit{Brutus}, è semplice e limpido. Cesare, infatti, cerca sempre di narrare gli avvenimenti in modo (solo apparentemente) soggettivo e distaccato emotivamente, utilizzando una struttura sintattica molto semplice e ricca di forme ricorrenti (come il \textit{quod}). La lingua è molto semplice e priva di arcaismi, neologismi o figure retoriche.

\section{De bello gallico}

\subsubsection{Struttura}

Il \textit{De bello gallico} è suddiviso in 7 libri e narra delle battaglie compiute da Cesare contro i Galli tra il 58 e il 52 a.C. Questa opera tien conto sia del carattere tecnico delle battaglie ma tende a delineare anche gli aspetti etnografici delle popolazioni incontrate (\textit{excursus} etnografici). In particolare, ogni libro tratta un argomento specifico:
\begin{itemize}
	\item libro I: Descrizione della Gallia, conflitto con gli Elvezi
	\item libro II: Campagna contro i Belgi
	\item libro III: Campagna in Bretannia
	\item libro IV: Campagna contro le popolazioni geramniche
	\item libro V: Campagna in gran Bretannia
	\item libro VI: Escursione di Cesare in Germania
	\item libro VII: Campagna finale contro Vercingetorìge
\end{itemize}

	\subsection{De bello gallico, Liber I}

	\begin{quote}
		Gallia est omnis divisa in partes tres, quarum unam incolunt Belgae, aliam Aquitani, tertiam qui ipsorum lingua Celtae, nostra Galli appellantur. 2 Hi omnes lingua, institutis, legibus inter se differunt. Gallos ab Aquitanis Garumna flumen, a Belgis Matrona et Sequana dividit. 3 Horum omnium fortissimi sunt Belgae, propterea quod a cultu atque humanitate provinciae longissime absunt, minimeque ad eos mercatores saepe commeant atque ea quae ad effeminandos animos pertinent important, 4 proximique sunt Germanis, qui trans Rhenum incolunt, quibuscum continenter bellum gerunt. Qua de causa Helvetii quoque reliquos Gallos virtute praecedunt, quod fere cotidianis proeliis cum Germanis contendunt, cum aut suis finibus eos prohibent aut ipsi in eorum finibus bellum gerunt. 5 Eorum una pars, quam Gallos obtinere dictum est, initium capit a flumine Rhodano, continetur Garumna flumine, Oceano, finibus Belgarum, attingit etiam ab Sequanis et Helvetiis flumen Rhenum, vergit ad septentriones. 6 Belgae ab extremis Galliae finibus oriuntur, pertinent ad inferiorem partem fluminis Rheni, spectant in septentrionem et orientem solem. 7 Aquitania a Garumna flumine ad Pyrenaeos montes et eam partem Oceani quae est ad Hispaniam pertinet; spectat inter occasum solis et septentriones.
	\end{quote}

	\subsubsection{Analisi del periodo}

	Gallia est omnis divisa in partes tres = principale rggente \\
	quarum unam incolunt Belgae, aliam Aquitani, tertiam qui = sub. relativa pr.\\
	qui ipsorum lingua Celtae, nostra Galli appellantur = sub. relativa propria\\
	Hi omnes lingua, institutis, legibus inter se differunt = principale.\\
	Gallos ab Aquitanis Garumna flumen, a Belgis Matrona et Sequana dividit = princ. \\
	Horum omnium fortissimi sunt Belgae = principale reggente\\
	propterea quod a cultu atque humanitate provinciae longe absunt = sub. causale\\
	minimeque ad eos mercatore saepe commeant = coord. alla principale\\
	nec ea important = coord. alla precedente\\
	quae ad mollitiam animorum pertinent = sub. relativa propria\\
	Proximique sunt Germani = prncipale reggente\\
	qui trans Rhenum incolunt = sub. relativa\\
	hac de causa Helvetii quoque reliquos Gallos virtute praecedunt = princ. reggente\\
	quod fere cotidianis proeliis cum Germanis contendunt = sub. causale\\
	cum aut a suis finibus eos prohibent = sub. temporale\\
	aut ipsi in eorum finibus bellum gerunt = coord. alla temporale\\
	Eorum una pars initium capit a flumine Rhodano = principale reggente\\
	quam Gallos obtinere dictum est = sub.relativa propria\\
	continetur Garumna flumine, Oceano, finibus Belgarum = coord. alla reggente\\
	sed attingit flumen Rhenum = coordinata alla precedente\\
	et vergit ad septemtriones = coordinata alla precedente\\
	Belgae ab extremis Galliae finibus oriuntur = principale\\
	pertinent ad inferiorem partem fluminis Rheni = coordinata per asindeto alla princ.\\
	spectant in septentrionem et orientem solem = coordinata per asindeto alla princ.\\

	\subsubsection{Analisi del testo}

	Come il resto dell'opera, questo testo presenta una struttura lineare e molto semplice: nella prima sezione Cesare fornisce una breve descrizione delle popolazioni che abitano la Gallia, nella parte centrale sono descritte le differenze culturali e sociali fra le diverse popolazioni e, infine, a conclusione del testo viene descritta la suddivisione del territorio delle varie popolazioni.

	Tra ciò che Cesare racconta, possiamo ben capire quali differenze fossero presenti tra i vari popoli dal punto di vista della "romanità", ovvero quanto la cultura romana influisse su ogni area della Gallia. In particolare, Cesare afferma che le popolazioni più lontane e distaccate dalla "romanizzazione" sono quelle più pericolose e ciò fornisce un motivo a Cesare per attaccare i Belgi.

	\subparagraph{Lo scontro con gli Elvezi}

	In una sezione successiva del primo libro del \textit{De bello gallico}, Cesare inizia il racconto delle guerre galliche che sono scaturite dopo lo scontro con gli Elvezi, popolazione celtica che avevano osato entrare in territorio romano in cerca di un nuovo territorio dove stabilirsi. Infatti, gli Elvezi, dopo il fallimentare tentativo di Orgetorige di riunificare la Gallia, decisero di abbandoanre i loro territori per stabilirsi altrove. Dopo aver bruciato le loro vecchie abitazioni e campi coltivati, segno decisivo della partenza, gli Elvezi tentarono di spostarsi verso sud, in direzione della Gallia narbonense. Cesare, che per la prima volta interviene militarmente nella narrazione, sentendosi spaventato dall'arrivo degli Elvezi decide di fortificare i confini della provincia romana e di respingere ogni eventuale attacco da parte di questa popolazione, riuscendo brillantemente. A questo punto, gli Elvezi, non scoraggiati dal comportamento di Cesare decidono di muoversi verso nord e di attraversare il territorio degli Edui, che, terrorizzati da ciò e alleati di Roma, chiedo aiuto militarmente a Cesare, creando così il perfetto \textit{casus belli}. L'esercito romano si addentrò nei territori gallici e cominciò a viaggiare verso nord fino a raggiungere gli Elvezi sul fiume Arar, dove gli Elvezi stavano trasportando le loro truppe al di là del fiume. All'arrivo dei romani circa tre quarti degli Elvezi avevano raggiunto l'altrasponda e Cesare sfruttò l'occasione la parte delle armate che non avevano ancoram effettuato la traversata. Anche se riuscirono a mietere un grande numero di vittime, a causa della bassa visibilità e confusione generale, una parte degli Elvezi riuscì a rifugiarsi nelle selve circostanti.

	\subsection{De bello gallico, Liber VI}

	Cesare, durante le sue campagne, tende a redigere degli \textit{excursus} etnografici delle popoalzioni che incotra per mostrare le loro condizioni di vita in modo \textbf{autoptico} al popolo romano e giustificare l'inizio del processo di romanizzazione.



\end{document}
