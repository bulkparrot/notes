\documentclass[10pt,a4paper]{article}
\usepackage[utf8]{inputenc}
\usepackage[T1]{fontenc}
\usepackage{amsmath}
\usepackage{amsfonts}
\usepackage{amssymb}
\usepackage{makeidx}
\usepackage{graphicx}
\usepackage[left=1.00in, right=1.00in, top=1.00in, bottom=1.00in]{geometry}
\author{Tommaso Severini}
\title{Funzione omografica}
\date{}

\parindent 0ex
\begin{document}
	\maketitle
	
	\thispagestyle{empty}
	
	Per definizione, una funzione omografica è una funzione rappresentata dalla funzione analitica:
	
	\begin{equation*}
		y = \frac{ax+b}{cx+d}	\qquad	\text{con} \quad a,b,c,d \in \mathbb{R} 
	\end{equation*}

	Nonostante questa funzione possa sia rappresentare sia una retta che un'iperbole, questa relazione si concentrerà sulla rappresentazione di un'iperbole e, in particolare, della sua derivazione.
	
	\section*{Derivazione}
	
	\subsection*{Ipotesi}
	
	In questo testo è presentata la dimostrazione del fatto che la funzione omografica non sia altro che la traslazione di un'iperbole equilatera nella forma $xy = k$.
	
	\subsection*{Dimostrazione}
	
	Sia data l'equazione di un'iperbole equilatera riferita ai propri assi di equazione:
	
	\begin{equation}
		xy = k
	\end{equation}

	Sia dato un vettore $\hat{V} (-\frac{d}{c}; \frac{a}{c})$ e il relativo sistema di traslazione:
	
	\begin{equation*}
		\left\{\begin{array}{@{}l@{}}
			x' = x-\frac{d}{c}\\
			y' = y+\frac{a}{c}
		\end{array}\right.\,
	\end{equation*}

	Da cui:
	
	\begin{equation}
		\left\{\begin{array}{@{}l@{}}
			x = x'+\frac{d}{c}\\
			y = y'-\frac{a}{c}
		\end{array}\right.\,
	\end{equation}

	Sostituendoi nuovi valori di x e y del sistema (2) nell'equazione (1), otteniamo:
	
	\begin{equation*}
		\left(x+\frac{d}{c}\right)\left(y-\frac{a}{c}\right)=k
	\end{equation*}

	\begin{equation*}
		xy + \frac{d}{c} y - \frac{a}{c} x - \frac{ad}{c^2}=k
	\end{equation*}

	Raccogliendo a fattor comune il termine $y$, otteniamo:
	
	\begin{equation*}
		\left( x+\frac{d}{c} \right) y = \frac{a}{c} x + \frac{ad}{c^2} + k
	\end{equation*}

	\begin{equation*}
		y = \frac{\frac{a}{c} x + \frac{ad}{c^2} + k}{x+\frac{d}{c}}
	\end{equation*}

	Moltiplicando sia il numeratore che il denominatore per $c$, otteniamo:
	
	\begin{equation*}
		y = \frac{ax + \frac{ad}{c} + kc}{cx+d}
	\end{equation*}

	Ponendo $\frac{ad}{c} + kc = b$, otteniamo la funzione omografica nella sua forma canonica:
	
	\begin{equation}
		y = \frac{ax+b}{cx+d}
	\end{equation}
	
	\hfill $\blacksquare$
	
\end{document}