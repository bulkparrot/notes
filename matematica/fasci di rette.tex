\documentclass[10pt,a4paper]{article}
\usepackage[utf8]{inputenc}
\usepackage[T1]{fontenc}
\usepackage{amsmath}
\usepackage{amsfonts}
\usepackage{amssymb}
\usepackage{makeidx}
\usepackage{graphicx}
\usepackage{mdframed}
\usepackage{xcolor}
\usepackage[left=1.00in, right=1.00in, top=1.00in, bottom=1.00in]{geometry}
\author{Tommaso Severini}
\title{Matematica - Fasci di rette}
\begin{document}
	
	%% mdframed definizione
\mdfdefinestyle{theoremstyle}{%
	linecolor=orange,linewidth=2pt,%
	frametitlerule=true,%
	frametitlebackgroundcolor=gray!20,
	innertopmargin=\topskip,
}
\mdtheorem[style=theoremstyle]{definition}{Definition}
	
	\maketitle
	
	Sappiamo come descrivere una retta conoscendo il suo coefficiente angolare ed un punto per cui questa retta passa attraverso la formula:
	\begin{equation*}
			y - y_0 = m(x - x_0)
	\end{equation*} 
	Se supponiamo che m possa variare e assumere qualsiasi valore appartenente a $\mathbb{R}$, essa diventerà l'equazione che descrive tutte le rette passanti per il punto $P(x_0; y_0)$. Questo insieme prende il nome di \textbf{fascio di rette proprio} si centro P.m L'unica retta non descritta da questa equazione è la retta $x = x_0$, in quanto il coefficiente angolare di questa retta tende a infinito.
	
	\begin{definition}[Fascio proprio di rette]
		L'equazione 
		\begin{equation*}
				y - y_0 = m(x - x_0)
		\end{equation*}
	dove $m$ è un elemento dei numeri reali, definisce il fascio di rette proprio di centro $P(x_0; y_0)$, esclusa l'equazione $x = x_0$
	\end{definition}

	
\end{document}