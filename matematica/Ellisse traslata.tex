%%% SETUP %%%%%%%%%%%%%%%%%%%%%%%%%%%%%%%%%%%%%%%%%%%%%%%%%%%%%%%%%%%%%%%%%%%%%%

	\documentclass[10pt, a4paper]{article}
	
	%%% PACKAGES %%%%%%%%%%%%%%%%%%%%%%%%%%%%%%%%%%%%%%%%%%%%%%%%%%%%%%%%%%%%%%%%%%%
	
	% Encoding
	
	\usepackage[utf8]{inputenc}
	\usepackage[T1]{fontenc}
	
	% Geometry
	
	\usepackage{geometry} % edit margins of paper
	\usepackage{setspace} % edit line spacing
	\usepackage{fancyhdr} % header, footer
	\usepackage{titlesec} % edit format of titles
	
	% Visual
	
	\usepackage[dvipsnames]{xcolor} % colors
	\usepackage{tikz} % graphics
	\usepackage[framemethod=tikz]{mdframed} % frames, better theorems
	
	% Math
	
	\usepackage{amsmath} % math tools
	\usepackage{amssymb} % math symbols
	\usepackage{amsthm} % thereoms
	\usepackage{mathtools} % math tools
	
	% Referencing
	
	\usepackage{nameref}
	\usepackage{hyperref}
	\usepackage{cleveref}
	
	% Useful
	
	\usepackage[shortlabels]{enumitem} % enumerations
	
	% Other
	
	\usepackage{lastpage} % get number of last page
	
	%%% MARGINS %%%%%%%%%%%%%%%%%%%%%%%%%%%%%%%%%%%%%%%%%%%%%%%%%%%%%%%%%%%%%%%%%%%%
	
	\geometry{a4paper, left=20mm, right=20mm, top=20mm, bottom=10mm, includehead}
	
	%%% TITLES %%%%%%%%%%%%%%%%%%%%%%%%%%%%%%%%%%%%%%%%%%%%%%%%%%%%%%%%%%%%%%%%%%%%%
	
	%%% SINGLE SYMBOLS %%%%%%%%%%%%%%%%%%%%%%%%%%%%%%%%%%%%%%%%%%%%%%%%%%%%%%%%%%%%
	
	% Logic
	
	% \forall exists
	% \exists exists
	% \lnot exists
	% \lor exists
	% \land exists
	\newcommand*{\limp}{\rightarrow}
	\newcommand*{\limps}{\; \limp \;} % \limp with some space around
	\newcommand*{\leqv}{\leftrightarrow}
	\newcommand*{\leqvs}{\; \leqvs \;} % \leqv with some space around
	
	% Meta Logic
	
	% \implies exists
	% \iff exists
	
	% Colon Stuff
	
	\newcommand*{\cl}{\colon}
	\newcommand*{\cleq}{\coloneqq}
	\newcommand*{\eqcl}{\eqqcolon}
	
	% Sets
	
	\newcommand*{\N}{\mathbb{N}} % natural numbers
	\newcommand*{\Z}{\mathbb{Z}} % integers
	\newcommand*{\Q}{\mathbb{Q}} % rational numbers
	\newcommand*{\R}{\mathbb{R}} % real numbers
	\newcommand*{\C}{\mathbb{C}} % complex numbers
	
	%%% MATH OPERATORS %%%%%%%%%%%%%%%%%%%%%%%%%%%%%%%%%%%%%%%%%%%%%%%%%%%%%%%%%%%%%
	
	% General
	
	\DeclareMathOperator{\id}{id}
	\DeclareMathOperator{\sgn}{sgn}
	
	%%% TEMPLATES %%%%%%%%%%%%%%%%%%%%%%%%%%%%%%%%%%%%%%%%%%%%%%%%%%%%%%%%%%%%%%%%%%
	
	% General
	
	% write a set definition like: { #1 | #2 }
	\newcommand*{\setdefinition}[2]{
		\left\{ #1 \mathrel{}\middle|\mathrel{} #2 \right\}
	}
	
	% write a nice map definition
	\newcommand*{\mapdefinition}[5]{
		\begin{align*}
			#1 \cl #2 &\to     #3 \\
			#4 &\mapsto #5
		\end{align*}
	}
	
	\colorlet{color-section}                {BrickRed}
	\colorlet{color-subsection}             {BrickRed}
	
	%%% MATH BOXES %%%%%%%%%%%%%%%%%%%%%%%%%%%%%%%%%%%%%%%%%%%%%%%%%%%%%%%%%%%%%%%%%
	
	\colorlet{color-definition}             {SpringGreen!30}
	\colorlet{color-theorem}                {Apricot!30}
	\colorlet{color-proposition}            {Apricot!30}
	\colorlet{color-corollary}              {Apricot!30}
	\colorlet{color-lemma}                  {Apricot!30}
	\colorlet{color-remark}                 {Gray!5}
	\colorlet{color-example}                {Lavender!7}
	% \colorlet{color-proof}                  {FILL COLOR HERE}
	
	\newcommand*{\definitionname}{Definition}
	\newcommand*{\theoremname}{Theorem}
	\newcommand*{\propositionname}{Proposition}
	\newcommand*{\corollaryname}{Corollary}
	\newcommand*{\lemmaname}{Lemma}
	\newcommand*{\remarkname}{Remark}
	\newcommand*{\examplename}{Example}
	
	%%% HEADER, FOOTER %%%%%%%%%%%%%%%%%%%%%%%%%%%%%%%%%%%%%%%%%%%%%%%%%%%%%%%%%%%%%
	
	\pagestyle{fancy}
	\fancyhf{} % clear everything
	\lhead{TRASLAZIONE DELL'ELLISSE}
	\rhead{Tommaso Severini}
	\lfoot{}
	\cfoot{}
	\rfoot{}
	
	%%% TITLE FORMAT %%%%%%%%%%%%%%%%%%%%%%%%%%%%%%%%%%%%%%%%%%%%%%%%%%%%%%%%%%%%%%%
	
	\setcounter{secnumdepth}{2}
	
	\titleformat{\chapter}[display]
	{\normalfont\huge\bfseries}{\chaptertitlename\ \thechapter}{20pt}{\Huge}
	\titleformat{\section}
	{\normalfont\Large\bfseries\color{color-section}}{\thesection}{1em}{}
	\titleformat{\subsection}
	{\normalfont\large\bfseries\color{color-subsection}}{\thesubsection}{1em}{}
	\titleformat{\subsubsection}
	{\normalfont\normalsize\bfseries}{\thesubsubsection}{1em}{}
	\titleformat{\paragraph}[runin]
	{\normalfont\normalsize\bfseries}{\theparagraph}{1em}{}
	\titleformat{\subparagraph}[runin]
	{\normalfont\normalsize\bfseries}{\thesubparagraph}{1em}{}
	
	%%% SPACING %%%%%%%%%%%%%%%%%%%%%%%%%%%%%%%%%%%%%%%%%%%%%%%%%%%%%%%%%%%%%%
	
	% Titles
	
	\titlespacing*{\chapter}{0pt}{50pt}{40pt}
	\titlespacing*{\section}{0pt}{3.5ex plus 1ex minus .2ex}{2.3ex plus .2ex}
	\titlespacing*{\subsection}{0pt}{3.25ex plus 1ex minus .2ex}{1.5ex plus .2ex}
	\titlespacing*{\subsubsection}{0pt}{3.25ex plus 1ex minus .2ex}{1.5ex plus .2ex}
	\titlespacing*{\paragraph}{0pt}{3.25ex plus 1ex minus .2ex}{1em}
	\titlespacing*{\subparagraph}{\parindent}{3.25ex plus 1ex minus .2ex}{1em}
	
	% Text, Paragraphs
	
	%\setstretch{1.05} % scaling of space between lines
	\setlength{\parindent}{0pt} % indentation of paragraphs
	%\setlength{\parskip}{4.0pt plus 1.0pt minus 1.0pt} % space between paragraphs
	\setlength{\parskip}{0.8ex}
	\setlength{\topsep}{0pt}
	
	%%% SYMBOLS USED BY NUMBERINGS, ENVIRONMENTS, ... %%%%%%%%%%%%%%%%%%%%%%%%%%%%%%
	
	% \renewcommand*\qedsymbol{$\blacksquare$} % alternative QED symbol
	\renewcommand{\thefootnote}{\arabic{footnote}} % normal footnotes on page
	\renewcommand{\thempfootnote}{\fnsymbol{mpfootnote}} % footnotes on minipages, e.g. in mdframed environments
	
	%%% LISTS, ENUMERATIONS %%%%%%%%%%%%%%%%%%%%%%%%%%%%%%%%%%%%%%%%%%%%%%%%%%%%%%%%
	
	% 'itemize'
	
	\setlist[itemize]{noitemsep, topsep=0pt}
	
	% 'enumerate'
	
	% no special settings at the moment
	
	% 'description'
	
	% no special settings at the moment
	
	% 'axioms'
	
	\newlist{axioms}{enumerate}{2}
	\setlist[axioms]{itemsep=0pt,label*=\arabic*.}
	
	%%% MDFRAMED PATCH %%%%%%%%%%%%%%%%%%%%%%%%%%%%%%%%%%%%%%%%%%%%%%%%%%%%%%%%%%%%%
	
	\usepackage{xpatch}
	
	\makeatletter
	\xpatchcmd{\endmdframed}
	{\aftergroup\endmdf@trivlist\color@endgroup}
	{\endmdf@trivlist\color@endgroup\@doendpe}
	{}{}
	\makeatother
	
	%%% MDFRAMED STYLES %%%%%%%%%%%%%%%%%%%%%%%%%%%%%%%%%%%%%%%%%%%%%%%%%%%%%%%%%%%%
	
	% thick frame and bar for title
	
	\mdfdefinestyle{style-box}{
		skipabove=1.5ex plus .5ex minus .2ex,
		skipbelow=1ex plus .2ex minus .2ex,
		linewidth=2pt,
		linecolor=Gray!20,
		%   roundcorner=3pt,
		innerleftmargin=0.5\baselineskip,
		innerrightmargin=0.5\baselineskip,
		innertopmargin=0.4\baselineskip,
		innerbottommargin=0.4\baselineskip,
		frametitlebackgroundcolor=Gray!20,
		frametitleaboveskip=0.3pt,
		frametitlebelowskip=0.3pt,
		theoremseparator=,
		theoremspace=\hfill,
		theoremtitlefont=\mdseries\scshape,
		nobreak=true
	}
	
	% highlighted background
	
	\mdfdefinestyle{style-background}{
		skipabove=1.5ex plus .5ex minus .2ex,
		skipbelow=1ex plus .2ex minus .2ex,
		hidealllines=true,
		backgroundcolor=Gray!5,
		innerleftmargin=0.5\baselineskip,
		innerrightmargin=0.5\baselineskip,
		innertopmargin=0.4\baselineskip,
		innerbottommargin=0.4\baselineskip,
	}
	
	% thin frame
	
	\mdfdefinestyle{style-leftline}{
		skipabove=1.5ex plus .5ex minus .2ex,
		skipbelow=1ex plus .2ex minus .2ex,
		linewidth=1pt,
		linecolor=Gray!50,
		topline=false,
		bottomline=false,
		rightline=false,
		innerleftmargin=0.5\baselineskip,
		innerrightmargin=0,
		innertopmargin=0.2\baselineskip,
		innerbottommargin=0.0\baselineskip,
	}
	
	%%% ENVIRONMENTS %%%%%%%%%%%%%%%%%%%%%%%%%%%%%%%%%%%%%%%%%%%%%%%%%%%%%%%%%%%%%%%
	
	% Definition
	
	\mdtheorem[
	style=style-box,
	linecolor=color-definition,
	frametitlebackgroundcolor=color-definition
	]{definition}{\definitionname}[section]
	
	% Theorem
	
	\mdtheorem[
	style=style-box,
	linecolor=color-theorem,
	frametitlebackgroundcolor=color-theorem,
	font=\itshape
	]{theorem}{\theoremname}[section]
	
	% Proposition
	
	\mdtheorem[
	style=style-box,
	linecolor=color-proposition,
	frametitlebackgroundcolor=color-proposition,
	font=\itshape
	]{proposition}[theorem]{\propositionname}
	
	% Corollary
	
	\mdtheorem[
	style=style-box,
	linecolor=color-corollary,
	frametitlebackgroundcolor=color-corollary,
	font=\itshape
	]{corollary}[theorem]{\corollaryname}
	
	% Lemma
	
	\mdtheorem[
	style=style-box,
	linecolor=color-lemma,
	frametitlebackgroundcolor=color-lemma,
	font=\itshape
	]{lemma}[theorem]{\lemmaname}
	
	\theoremstyle{remark}
	
	% Remark
	
	\newtheorem*{remark}{\remarkname}
	\surroundwithmdframed[
	style=style-background,
	backgroundcolor=color-remark
	]{remark}
	
	% Example
	
	\newtheorem*{example}{\examplename}
	\surroundwithmdframed[
	style=style-background,
	backgroundcolor=color-example
	]{example}
	
	% Proof
	
	\surroundwithmdframed[
	style=style-leftline
	]{proof}
	
	%%% TEXT FORMATTING %%%%%%%%%%%%%%%%%%%%%%%%%%%%%%%%%%%%%%%%%%%%%%%%%%%%%%%%%%%%
	
	% definitions
	
	\newcommand*{\df}[1]{\textbf{#1}}

%%% DOCUMENT %%%%%%%%%%%%%%%%%%%%%%%%%%%%%%%%%%%%%%%%%%%%%%%%%%%%%%%%%%%%%%%%%%%

	\title{Traslazione dell'ellisse}
	\author{Tommaso Severini}
	\date{}

\begin{document}
	
	\maketitle
	
	\section*{Elementi teorici}
	
	\subsection*{Definizione}
	
	L'ellisse è il luogo geometrico dei punti del piano per i quali è costante la somma delle distanze da due punti fissi detti fuochi. In termini più generali un'ellisse è una conica non degenere.
	
	Partiamo dalla definizione di ellisse anticipata e spieghiamone il significato:
	
	\begin{definition*}[Ellisse]
		Si definisce \textbf{ellisse} il luogo geometrico dei punti del piano per cui è costante la somma da due punti fissi $F_1 e F_2$, detti fuochi.\\
		
		Indicando con $P$ uno dei punti appartenenti all'ellisse, possiamo tradurre la definizione data in formule:
		
		\begin{equation*}
			\overline{\rm PF_1} + \overline{\rm PF_2} = 2a \qquad \text{dove $a$ rappresenta il semiasse maggiore dell'ellisse}
		\end{equation*}
	\end{definition*}

	Questa condizione, una volta espansa attraverso l'utilizzo della formula di disanza tra 2 punti, si traduce analiticamente nella seguente:
	
	\begin{equation}
		\frac{x^2}{a^2} + \frac{y^2}{b^2} = 1 \qquad \text{con $a>b>0$}
	\end{equation}

	\subsection*{Parametri}
	
	\subsubsection*{Assi principali}
	
	In questo articolo, tutte le considerazioni riguardanti il \textbf{semiasse maggiore} e il \textbf{semiasse minore} sono indicati rispettivamente con le lettere $a$ e $b$. In particolar modo, in questo articolo è fatta l'assunzione $a>b$, in modo da semplificare di molto i calcoli. 
	
	Nonostante ciò, è possibile che si verifichi il caso in cui $a<b$. In tale situazione, i fuochi dell'ellisse saranno situati lungo l'asse y.
	
	\subsubsection*{Eccentricità}

	La deformazione di un ellisse è misurata attraverso la sua eccentricità $e$, che può assumere valori compresi tra 0, nel caso in cui l'ellisse degenera in una circonferenza, e 1, nel caso in cui l'ellisse degenera in un segmento. 
	
	Questo valore è espresso dal rapporto tra la distanza focale ed il semiasse maggiore dell'ellisse, ovvero:
	
	\begin{equation*}
		e = \frac{2c}{2a} = \frac{c}{a} = \sqrt{1- \frac{b^2}{a^2}}
	\end{equation*}

	\subsubsection*{Rette tangenti}
	
	Dato un punto dell'ellisse di coordinate $P(x_0;y_0)$, la retta tangente all'ellisse nel punto P avrà equazione:
	
	\begin{equation*}
		\frac{xx_0}{a^2} + \frac{yy_0}{b^2} = 1
	\end{equation*}

	\subsection*{Applicazioni pratiche}
	
	Gli ellissi sono comuni in ambiti come la fisica, l'astronomia e l'ingegneria. Per esempio, l'orbita di ogni pianeta del sistema solare, secondo la prima legge di Keplero, è un ellisse in cui uno dei fuochi è rappresentato dal Sole.
	
	Lo stesso ragionamento risulta corretto anche per molte lune che orbitano i rispettivi pianeti e tutti gli altri sistemi astronomici costituiti da due corpi celesti.
	
	Oltre a ciò, la forma di pianeti e stelle può essere approssimata da un ellissoide, solido ottenuto attraverso la rotazione di un ellisse attorno ad uno dei propri assi.

	


	
	\section*{Interpretazione geometrica}
	
	In questa sezione di articolo osserveremo quale modifiche subisce l'equazione dell'ellisse nell'ipotesi in cui il centro sia un punto diverso dall'origine degli assi.\\
	
	Consideriamo un ellisse $\gamma$, di semiassi $a$ e $b$, avente centro nell'origne O. Definiamo quindi  il sistema di traslazione che sposti il centro dell'ellisse da O ad un punto $C(x_C;y_C)$:
	
	\begin{equation}
		\begin{cases} x'=x+x_C \\ y'=y+y_C \end{cases} \qquad \Rightarrow \qquad \begin{cases}
			x=x'-x_C \\ y= y'-y_C
		\end{cases}
	\end{equation}

	Applicando la traslazione (2) all'equazione (1), otterremo la formula dell'ellisse traslata di centro C.
	
	\begin{equation}
		\frac{(x-x_C)^2}{a^2} + \frac{(y-y_C)^2}{b^2} =1
	\end{equation}
	
	\section*{Interpretazione algebrica}
	
	In geometrai analitica, ogni sezione conica può essere rappresentata mediante l'utilizzo di uno strumento dell'algebra lineare noto come \textbf{rappresentazione matriciale delle sezioni coniche}. Questa metodologia consente di studiare dati elementi matematici senza ridurre essi ad una forma canonica condizionata da rotazioni o traslazioni, rendendo lo studio molto più semplice. \\
	
	Le sezioni coniche possono essere rappresentate come l'insieme dei punti del piano che rispettano la seguente equazione di secondo grado in 2 incognite:
	
	\begin{equation}
		Q(x,y) = Ax^2 + Bxy + Cy^2 + Dx + Ey + F =0
	\end{equation}

	I valori A, B e C sono influenzati e possono essere usati per ricavare un possibile angolo di rotazione dell'ellisse(nel caso in cui $B=0$, l'ellisse è privo di rotazione), mentre i valori D, E ed F sono influenzati e possono essere utilizzati per ricavare le coordinate del centro dell'ellisse.\\

	I principali strumenti che ci permettono di determinare la tipologia di sezione conica rappresentata dalla formula sovracitata sono le matrici $A_Q$ e $A_{33}$:
	$$A_Q = \begin{vmatrix}
		A & B/2 & D/2 \\
		B/2 & C & E/2 \\
		D/2 & E/2 & F
	\end{vmatrix}
	\qquad
	\qquad
	A_{33}= \begin{vmatrix}
		A & B/2 \\
		B/2 & C
	\end{vmatrix}$$

	Dove $A_Q$ è nota come equazione dell'equazione quadratica e $A_{33}$ è nota come matrice della forma quadratica, rappresentata dalla prima minore di $A_Q$.	
	
	In particolare, il determinante della matrice $A_Q$ è utilizzata per distinguere le sezioni coniche degeneri da quelle proprie, mentre il determinante della matrice $A_{33}$ permette di individuare la tipologia di sezione conica che si sta studiando. 
	
	Nel caso di un ellisse non degenere, le condizioni necessarie sono $det(A_Q) \ne 0$, affinchè non si ottenga un ellisse degenere, e $det(A_{33} > 0)$, che identifica l'ellisse.
	
	\begin{theorem*}
		Un'equazione della forma $Ax^2 + Bxy + Cy^2 + Dx + Ey + F =0 \: con B=0$ rappresenta un ellisse se e solo se è verificata la \textbf{condizione di realtà}:
		
		$$ \frac{D^2}{4A} + \frac{E^2}{4C^2} - F >= 0 $$
		
		Il centro di tale ellisse sarà dato dal punto di coordinate:
		
		$$ \left( -\frac{D}{2A}; -\frac{E}{2C} \right) $$ 
		
	\end{theorem*}

	\begin{proof}
		Per dimostrare quale sia la condizione di realtà di un ellisse, utilizzeremo la proprietà\footnote{Lawrence, J. Dennis, A Catalog of Special Plane Curves, Dover Publ., 1972. pag. 63} che indica come in un ellisse reale e non degenere, il prodotto $C \cdot det(A_Q) < 0$  (5).\\
		
		$$
			det(A_Q) = \left( AC - \frac{B^2}{4} \right) F - \frac{BED}{4} - \frac{CD^2}{4} - \frac{AE^2}{4}
		$$
		
		Poichè B=0 per ipotesi, è possibile semplificare ulteriormente il valore del determinante:
		
		$$
			det(A_Q) = ACF - \frac{CD^2}{4} - \frac{AE^2}{4}
		$$
		
		Sostituendo il valore ottenuto nell'equazione (5), otteniamo:
		
		$$
		 C \cdot ACF - \frac{CD^2}{4} - \frac{AE^2}{4} < 0
		$$
		$$
		AC^2F - \frac{C^2D^2}{4} - \frac{AE^2C}{4} < 0
		$$
		
		Dividendo per $-AC^2$ entrambi i membri, otteniamo la condizione di realtà:
		
		$$
		\frac{D^2}{4A} + \frac{E^2}{4C^2} - F >= 0
		$$
		
	\end{proof}

	\begin{proof}
		Il centro di una conica, se esso esiste, è il punto medio di tutte le corde dell'ellisse che attraversano il centro stesso. Questa proprietà\footnote{Ayoub, A. B. (1993), "The central conic sections revisited", Mathematics Magazine} può essere usata per calcolare le coordinate del centro, che può essere rappresentato come il punto in cui il gradiente della funzione funzione di secondo grado $Q$ diviene 0: 
		
		$$\nabla Q = \left[ \frac{\partial Q}{\partial x} ; \frac{\partial Q}{\partial y} \right] = [0;0] $$
		
		Svolgendo le rispettive derivate parziali mettendo a sistema le equazioni ottenute, è facile constatare come le coordinate del centro risulteranno essere:
		
		$$ \left( \frac{2CD-BE}{B^2-4AC}; \frac{2AE-BD}{B^2-4AC} \right) $$
		
		Poichè stiamo considerando il caso in cui B=0, le coordinate assumeranno la forma:
		
		$$ \left( -\frac{D}{2A}; -\frac{E}{2C} \right) $$
		
	\end{proof}
	
\end{document}