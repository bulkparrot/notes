\documentclass[10pt,a4paper]{article}
\usepackage[utf8]{inputenc}
\usepackage[T1]{fontenc}
\usepackage{amsmath}
\usepackage{amsfonts}
\usepackage{amssymb}
\usepackage{commath}
\usepackage{makeidx}
\usepackage{hyperref}
\hypersetup{
	colorlinks=true,
	linkcolor=blue,
	filecolor=magenta,
	urlcolor=red,
}

\usepackage{graphicx}
\usepackage{wrapfig}
\usepackage{mdframed}
\usepackage{xcolor}
\usepackage{tikz}
\usepackage{pgfplots}
\pgfplotsset{width=3in,compat=1.9}
\usepackage[left=1.00in, right=1.00in, top=1.00in, bottom=1.00in]{geometry}
\author{Tommaso Severini}
\title{Geometria analitica - Parabola}

\begin{document}
	\maketitle

  %% mdframed definizione
\mdfdefinestyle{theoremstyle}{%
  linecolor=orange,linewidth=2pt,%
  frametitlerule=true,%
  frametitlebackgroundcolor=gray!20,
  innertopmargin=\topskip,
}
\mdtheorem[style=theoremstyle]{definition}{Definition}

\begin{definition}[Parabola]
  Dati nel piano una retta $d$ e un punto $F \notin d$, si dice \textbf{parabola} di fuoco F e direttrice d il luogo geometrico dei punti dep piano equidistanti da F e da d.
\end{definition}

\section{Paraboal con vertice nell'origine}

Consideriamo il caso di una parabola avente vertice nell'origine degli assi e asse parallelo all'asse y. Sia la distanza tra il fuoco F e l'origine O uguale a $k$. Le coordinate del fuoco saranno quindi $F(0; k)$. Poichè sappiamo che anche la direttrice dovrà avere distanza $k$ dall'origine e dovrà essere perpendicolare all'asse y, essa avrà equazione $y=-k$.\\

Detto ciò consideriamo un punto generico $P(x; y)$ appartenente alla parabola e poniamo la sua distanza da F uguale a quella dalla direttrice $d$.

\begin{equation}
  d(P;F) = d(P;d)
\end{equation}
\begin{equation}
  \sqrt{x^2+(y-k)^2}=\abs{y+k}
\end{equation}

\begin{equation}
  x^2+y^2-2ky+k^2= y^2+2ky+k^2
\end{equation}

Da ciò si ottiene \textbf{l'equazione di una paarbola con vertice nell'origine e parallela all'asse y}:

\begin{equation}
    y= \frac{1}{4k}x^2
\end{equation}

\end{document}
