\documentclass[10pt,a4paper]{article}
\usepackage[utf8]{inputenc}
\usepackage[T1]{fontenc}
\usepackage{amsmath}
\usepackage{amsfonts}
\usepackage{amssymb}
\usepackage{makeidx}
\usepackage{graphicx}
\usepackage{mdframed}
\usepackage[left=1.00in, right=1.00in, top=1.00in, bottom=1.00in]{geometry}
\author{Tommaso Severini}
\title{Chimica - Le teorie atomiche}
\begin{document}
		%% mdframed definizione rosso
	\mdfdefinestyle{theoremstylered}{%
		linecolor=red,linewidth=2pt,%
		frametitlerule=true,%
		frametitlebackgroundcolor=gray!20,
		innertopmargin=\topskip,
	}
	\mdtheorem[style=theoremstylered]{definitionred}{Definition}
	
		%% mdframed definizione blu
	\mdfdefinestyle{theoremstyleblue}{%
		linecolor=blue,linewidth=2pt,%
		frametitlerule=true,%
		frametitlebackgroundcolor=gray!20,
		innertopmargin=\topskip,
	}
	\mdtheorem[style=theoremstyleblue]{definitionblue}{Definition}
	
		%% mdframed definizione verde
	\mdfdefinestyle{theoremstylegreen}{%
		linecolor=green,linewidth=2pt,%
		frametitlerule=true,%
		frametitlebackgroundcolor=gray!20,
		innertopmargin=\topskip,
	}
	\mdtheorem[style=theoremstylegreen]{definitiongreen}{Definition}
	
	\maketitle
	
	Grazie al lavoro svolto da vari chimici nel corso del XVIII e XIX secolo come \textbf{Thomson, Rutherford, Moesly e Chadwick}, oggi conosciamo le caratteristiche delle particelle subatomiche che compongono gli atomi: elettroni, protoni e neutroni.
\end{document}