\documentclass[10pt,a4paper]{article}
\usepackage[utf8]{inputenc}
\usepackage[T1]{fontenc}
\usepackage{amsmath}
\usepackage{amsfonts}
\usepackage{amssymb}
\usepackage{makeidx}
\usepackage{graphicx}
\usepackage{mdframed}
\usepackage{xcolor}
\usepackage[left=1.00in, right=1.00in, top=1.00in, bottom=1.00in]{geometry}
\author{Tommaso Severini}
\title{Chimica - Le teorie atomiche}
\begin{document}
		%% mdframed definizione rosso
	\mdfdefinestyle{theoremstylered}{%
		linecolor=red,linewidth=2pt,%
		frametitlerule=true,%
		frametitlebackgroundcolor=gray!20,
		innertopmargin=\topskip,
	}
	\mdtheorem[style=theoremstylered]{definitionred}{Definition}
	
		%% mdframed definizione blu
	\mdfdefinestyle{theoremstyleblue}{%
		linecolor=blue,linewidth=2pt,%
		frametitlerule=true,%
		frametitlebackgroundcolor=gray!20,
		innertopmargin=\topskip,
	}
	\mdtheorem[style=theoremstyleblue]{definitionblue}{Definition}
	
		%% mdframed definizione verde
	\mdfdefinestyle{theoremstylegreen}{%
		linecolor=green,linewidth=2pt,%
		frametitlerule=true,%
		frametitlebackgroundcolor=gray!20,
		innertopmargin=\topskip,
	}
	\mdtheorem[style=theoremstylegreen]{definitiongreen}{Definition}
	
	\maketitle
	
	Grazie al lavoro svolto da vari chimici nel corso del XVIII e XIX secolo come \textbf{Thomson, Rutherford, Moesly e Chadwick}, oggi conosciamo le caratteristiche delle particelle subatomiche che compongono gli atomi: elettroni, protoni e neutroni.
	
	\begin{definitionblue}[L'elettrone]
		L'elettrone è una particella con carica elettrica negativa. Il suo \textbf{simbolo è $e^-$} e ha \textbf{carica} pari a $-1.6 \cdot 10^{-19} C$. La sua massa corrisponde ad 1/1836 dell'\textbf{uma}, unità di massa atomica.
	\end{definitionblue}

	\begin{definitionred}[Il protone]
		Il protone è una particella con carica elettrica positiva. Il suo \textbf{simbolo} è $p^+$ e carica pari all'elettrone, ma di segno opposto. La sua massa è di poco superiore ad 1 uma (1.007276 uma).
	\end{definitionred}

	\begin{definitiongreen}[Il neutrone]
		Il protone è una particella priva di carica elettrica. Il suo \textbf{simbolo} è $n$. La sua massa è di poco superiore ad 1 uma ed è equivalente alla somma della massa di un protone o di un elettrone(1.008665 uma).
	\end{definitiongreen}

\section{Numeri quantici}

In meccanica quantistica un numero quantico esprime il valore di una quantità conservata nella dinamica di un sistema. I numeri quantici permettono di quantificare le proprietà di una particella e di descrivere la struttura elettronica di un atomo.

Convenzionalmente si usa caratterizzare un sistema con quattro numeri quantici principali:

\begin{itemize}


\item L'autovalore dell'energia simbolo n, detto anche \textbf{numero quantico principale} o di Bohr, che assume valori interi n=1,2,3... e che dipende dalla sola distanza tra l'elettrone ed il nucleo.

\item Il modulo quadro del momento angolare orbitale \textit{l}, detto anche \textbf{numero quantico secondario}, che può assumere valori interi compresi tra 0 e n-1. Esso definisce la forma dell'orbitale atomico.

\item La componente lungo un asse (simbolo m) del momento angolare orbitale, detto \textbf{numero quantico magnetico}, che assume valori interi tra -\textit{l} e \textit{+l}.

\item La componente dello spin, detto \textbf{numero quantico di spin}, che può assumere valori semi interi che sono da +1/2 e -1/2.	
\end{itemize}

\section{Orbitali}
	
\end{document}