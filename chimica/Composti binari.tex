\documentclass[10pt,a4paper]{article}
\usepackage[utf8]{inputenc}
\usepackage[T1]{fontenc}
\usepackage{amsmath}
\usepackage{amsfonts}
\usepackage{amssymb}
\usepackage{makeidx}
\usepackage{graphicx}
\usepackage[left=1.00in, right=1.00in, top=1.00in, bottom=1.00in]{geometry}
\author{Tommaso Severini}
\title{Chimica - Composti binari}

\parindent 0ex
\begin{document}
	\maketitle
	
	I composti binari, ovvero la classe di composti inorganici contenenti unicamente 2 specie chimiche, si possono classificare a seconda del legame che unisce le specie chimiche:\\
	
\textbf{	Legame ionico:}
	
	\begin{itemize}
		\item Idruri metallici (gruppi 1 e 2)
		\item Ossidi basici (metallici)
		\item Sali binari
	\end{itemize}

\textbf{	Legame covalente:}
	
	\begin{itemize}
		\item Idruri covalenti (non metalli/semimetalli gruppi 14, 15, 16)
		\item Ossidi acidi (non metalli/semimetalli)
		\item Idracidi (non metalli)
	\end{itemize}

\section*{Composti dell'idrogeno}

 \subsection*{Idracidi}
 
 	Gli idracidi sono composti binari dell'idrogeno con i non metalli (gruppi 16 e 17). L'idrogeno in questi composti ha \textbf{numero di ossidazione +1}.\\
 	
 	\begin{tabular}{|c|c|c|}
 		\hline
 		Formula chimica & Nomenclatura tradizionale & Nomenclatura IUPAC \\
 		\hline
 		Idrogeno + non metallo & acido + nonmetallo -idrico & nonmetallo -uro + di + n-idrogeno \\
 		\hline
 	\end{tabular} \\
 
 	Ad esempio, il composto HCl prende, tradizionalmente il nome "acido cloridrico", il nome di "cloruro di (mono)idrogeno".
 	
 	Una delle più notabili eccezioni è costituita dai composti del cianuro (CN). Infatti, nonostante esso si composto da 2 specie chimiche elementari, è spesso considerato come una specie unica. Per questo motivo il composto HCN prende, tradizionalmente il nome "acido cianidrico", il nome "cianuro di idrogeno".
 	
 \subsection*{Idruri}
 
	Gli idruri sono composti in cui l'idrogeno si lega con uno dei metalli (o semimetalli o non metalli) dei gruppi 1 a 15. In questi composti, l'idrogeno tende ad avere \textbf{numero di ossidazione -1}.
	
	\begin{tabular}{|c|c|c|}
		\hline
		Formula chimica & Nomenclatura tradizionale & Nomenclatura IUPAC \\
		\hline
		Metallo*  + idrogeno & Idruro di + metallo*  & N-idruro + di metallo*  \\
		\hline
	\end{tabular}
	
	*per i semimetalli ed i non metalli la nomenclatura rimane invariata.
	
\end{document}