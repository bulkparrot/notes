\documentclass[11pt]{article}
\usepackage[margin=1in]{geometry}
\usepackage{amsfonts,amsmath,amssymb}
\usepackage[italian]{babel}
\usepackage{fancyhdr}

\usepackage{pgfplots} %grafici e colori
\usepackage{xcolor}
\usepackage{tikz}

\usepackage{graphicx} %figure
\usepackage{wrapfig}

\usepackage{isotope} %equazioni chimiche

\usetikzlibrary{ shapes.geometric } %liberie tikz
\usetikzlibrary{calc}
\usepackage{anyfontsize}
\pgfplotsset{width=10cm,compat=1.9}

\usepackage{hyperref}

\title{Trattamenti e cure per il cancro}
\author{Tommaso Severini}

\pagestyle{fancy}
\fancyhead{}
\fancyfoot{}
\fancyhead[L]{TRATTAMENTI E CURE}
\fancyhead[R]{Tommaso Severini}
\fancyfoot[C]{\thepage}

\parindent 0ex

\begin{document}
	
	\maketitle
	
Il trattamento e le cure contro il cancro hanno subito un grande processo di evoluzione man mano che la nostra conoscenza dei processi biologici è aumentata nel tempo. Le ultime scoperte in questo campo riguardano la chemioterapia e la terapia immunitaria, sviluppati nel XX secolo. 

\section*{I trattamenti disponibili}

\subsection*{Chirurgia}

La rimozione chirurgica di tumori è stata documentata a partire dall'antico  Egitto ed è sicuramente una tecnica utile ed efficace per molte tipologie di piccoli tumori, non liquidi (come la leucemia) e privi di metastasi.

\subsection*{Radioterapia}

Questa metodologia si avvale di radiazioni ionizzanti per uccidere le cellule neoplastiche e ridurre le dimensioni della massa tumorale. Queste causano dei danni al DNA delle cellule tumorali, che, non essendo in grado di riparare in modo efficiente questi errori, ne impediscono la riproduzione.

\subsection*{Chemioterapia}

Questa tecnica si avvale di alcuni farmaci, detti chemioterapici, che attaccano selettivamente le cellule che si riproducono frequentemente, generalmente impedendo la duplicazione del loro DNA o la separazione di cromosomi appena formati. Per questo motivo i chemioterapici hanno la possibilità di attaccare anche tessuti sani le cui cellule si dividono spesso, come i tessuti intestinali. Nonostante ciò, questi tessuti si riparano dopo la fine del trattamento, non provocando danni permanenti.

Oggi però esistono farmaci più mirati e specifici per determinate categorie di cellule, come farmaci ormonali o a bersaglio molecolare, diminuendo così gli effetti collaterali.

Nuove informazioni riguardanti la biologia dei tumori continueranno ad emergere in futuro; nuovi trattamenti saranno sviluppati e modificati, aumentando la loro efficacia, precisione, tasso di sopravvivenza e qualità di vita dei malati.
	
\end{document}