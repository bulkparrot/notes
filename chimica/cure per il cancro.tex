\documentclass[11pt]{article}
\usepackage[margin=1in]{geometry}
\usepackage{amsfonts,amsmath,amssymb}
\usepackage[italian]{babel}
\usepackage{fancyhdr}

\usepackage{pgfplots} %grafici e colori
\usepackage{xcolor}
\usepackage{tikz}

\usepackage{graphicx} %figure
\usepackage{wrapfig}

\usepackage{isotope} %equazioni chimiche

\usetikzlibrary{ shapes.geometric } %liberie tikz
\usetikzlibrary{calc}
\usepackage{anyfontsize}
\pgfplotsset{width=10cm,compat=1.9}

\usepackage{multicol}

\usepackage{hyperref}

\title{Trattamenti e cure per il cancro}
\author{Tommaso Severini}
\date{}

\pagestyle{fancy}
\fancyhead{}
\fancyfoot{}
\fancyhead[L]{TRATTAMENTI E CURE}
\fancyhead[R]{Tommaso Severini}
\fancyfoot[C]{\thepage}

\parindent 0ex

\begin{document}
	
	\maketitle
	
	\begin{abstract}

Il trattamento e le cure contro il cancro hanno subito un grande processo di evoluzione man mano che la nostra conoscenza dei processi biologici è aumentata nel tempo. Le ultime scoperte in questo campo riguardano la chemioterapia e la terapia immunitaria, sviluppati nel XX secolo. 

	\end{abstract}


	


\section*{I trattamenti disponibili}

\subsection*{Chirurgia}

La rimozione chirurgica di tumori è stata documentata a partire dall'antico  Egitto ed è sicuramente una tecnica utile ed efficace per molte tipologie di piccoli tumori, non liquidi (come la leucemia) e privi di metastasi.



\subsection*{Radioterapia}

Questa metodologia si avvale di radiazioni ionizzanti per uccidere le cellule neoplastiche e ridurre le dimensioni della massa tumorale. Queste causano dei danni al DNA delle cellule tumorali, che, non essendo in grado di riparare in modo efficiente questi errori, ne impediscono la riproduzione.

\subsection*{Chemioterapia}

Questa tecnica si avvale di alcuni farmaci, detti chemioterapici, che attaccano selettivamente le cellule che si riproducono frequentemente, generalmente impedendo la duplicazione del loro DNA o la separazione di cromosomi appena formati. Per questo motivo i chemioterapici hanno la possibilità di attaccare anche tessuti sani le cui cellule si dividono spesso, come i tessuti intestinali. Nonostante ciò, questi tessuti si riparano dopo la fine del trattamento, non provocando danni permanenti.

\subsection*{Immunoterapia}

L'immunoterapia è una tipologia di trattamento che utilizza componenti del sistema immunitario per eliminare cellule tumorali. Queste strategie che sfruttano le difese immunitarie dello stesso malato sono permettono di attaccare un particolare tipologia di tumore, offrendo, teoricamente, la possibilità di sviluppare una memoria immunologica specifica per le cellule cancerose.\cite{bworld} Ciò potrebbe causare una regressione a lungo termine e un'importante misura preventiva per il paziente.



\section{La ricerca}

L'immunoterapia ha sostanzialmente migliorato il tasso di sopravvivenza e la qualità di vita dei pazienti. Nonostante ciò, non tutti i tumori sono uguali e ci sono ancora poche prove cliniche che possano determinare la tossicità e la reazione che nuovi farmaci possano avere sui malati. Nonostante i suoi numerosi processi, l'immuno-oncologia si trova ancora in uno stato embrionale, con numerose sfide ancora da superare. \cite{esfahani2020review}



\bibliographystyle{naturemag}
\bibliography{cancro.bib}
	
\end{document}