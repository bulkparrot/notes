\documentclass[10pt,a4paper]{article}
\usepackage[utf8]{inputenc}
\usepackage[T1]{fontenc}
\usepackage{amsmath}
\usepackage{amsfonts}
\usepackage{amssymb}
\usepackage{makeidx}
\usepackage{graphicx}
\usepackage[left=1.00in, right=1.00in, top=1.00in, bottom=1.00in]{geometry}
\author{Tommaso Severini}
\title{Filosofia e storia - Dalla rivoluzione inglese all'illuminismo}
\begin{document}
	\maketitle
	
	\section*{I teorici della politica}
	
	\subsection*{Hobbes: il teorico dell'assolutismo}
	
	Thomas Hobbes è stato un filosofo britannico, ideatore di ciò che può essere definito assolutismo politico. Egli, nel descrivere la sua teoria politica, parte da una semplice constatazione: l'uomo che vive nello stato di natura agisce in modo egoista e competitiva. Questa visione dell'ordine naturale può essere riassunta con l'espressione \textit{Homo homini lupus}, ogni uomo è un lupo per l'altro uomo.
	
	Ciò, secondo Hobbes, porta gli uomini ad organizzarsi in uno stato secondo un \textbf{contratto}, attraverso il quale gli uomini, ora sudditi, rinunciano ai loro diritti illimitati per concentrarle in un solo individuo. Portando questa entità a rispettare la \textbf{volontà di tutti} coloro che sottoscrivono il suddetto contratto, ovvero il popolo. Questa figura, che può essere fisica o giuridica, prende il nome di \textbf{Leviatano}, riprendendo la simbologia biblica che vede questo mostro come l'essere terreno più potente e terribile mai esistito. 
	
	\subsubsection*{Le caratteristiche dell'assolutismo hobbesiano}
	
	\begin{itemize}
		\item La prima caratteristica da evidenziare è il fatto che il contratto radatto dai sudditi è \textbf{unilaterale e irreversibile}. Essenzialmente ciò che Hobbes vuole affermare è che i patti stabiliti dalla società vanno rispettati. In caso contrario si creerebbe una situazione contraddittoria in cui un suddito che ha deciso di rinunciare ai propri diritti prova a ribellarsi alla persona a cui ha concesso i suoi diritti.
		\item Seconda caratteristica dell'assolutismo di Hobbes è che il potere si presenta \textbf{indivisibile}. Infatti una divisione del potere non porterebbe a nessun miglioramento, ma, nella peggiore delle ipotesi, anche alla guerra civile.
		\item In terzo luogo, il giudizio morale appartiene allo Stato. Ogni suddito, infatti, rinuncia alla propria volontà concentrandola nello Stato che quindi agirà sempre secondo il volere del popolo. Per questo motivo lo Stato ha la capacità di distinguere il bene dal male. Oltre a ciò, la cosa più importante da ricordare è che \textbf{la legge stabilisce la morale}, perchè la volontà di tutti si concentra nella legge dettata dal Leviatano.
		\item Quarto punto, \textbf{ogni ordine del sovrano va eseguito}, anche se essi sono ritenuti peccaminosi. L'unica eccezione è costituita dagli ordini che vanno contro il \textbf{diritto alla vita}.
		\item Il tirannicidio è inaccettabile.
		\item Lo Stato non è soggetto alle leggi dettate dallo Stato stesso. Ciò poichè lo Stato non ha nessun obbligo verso i cittadini, dato dal fatto che il patto è unilaterale, nè verso se stesso. 
	\end{itemize}

	\subsubsection*{I limiti del potere statale}
	
	Il sovrano (nella concezione hobbesiana il re o l'imperatore, ma anche lo Stato stesso, in qualunque sua forma) ha un potere assoluto e unitario e non ha alcun dovere nei confronti dei sudditi (tranne di proteggere la Nazione) perché essi stessi gli hanno dato i loro diritti ed è impossibile quindi che egli violi i patti e possa essere deposto, a meno che non ordini ai sudditi di uccidersi o danneggiare la loro persona o quella di un proprio caro (poiché quello dell'autoconservazione della propria vita è l'unico diritto che non gli è stato trasferito) con una guerra suicida. Quando egli cioè non sarà capace o non avrà la forza di difendere dai nemici interni ed esterni lo Stato assicurando, secondo i patti stipulati, l'ordine e la pace, allora, solo in questo caso, egli potrà essere deposto (come accadde a Carlo I) e sostituito con un nuovo sovrano.
	
	\subsection*{Locke: il teorico del liberalismo}
	
	John Locke è stato un filosofo inglese, considerato l'ideatore della teoria politica oggi conosciuta come liberalismo. Nelle sue opere, Locke descrive un teoria politica e etica basata sulla ragione, in quanto ciò che risulta giusto deve avere una motivazione razionale. Le caratteristiche principali dello Stato di Locke sono la libertà dei cittadini, la tolleranza religiosa e la libertà delle Chiese. 
	
	Sarà proprio la ragione che regola il comportamento umano a definire la forma che assume lo stato di natura nel pensiero di Locke. Egli, infatti, vede questa condizione caratterizzata da una legge di natura che corrisponde alla ragione stessa. Ciò permette di garantire la pacifica convivenza tra gli uomini in una società priva di Stato. Tutti gli uomini, che sono provvisti di ragione, si ritrovano quindi uguali dal punto di vista dei diritti naturali e le uniche limitazioni che incorrono, regolate dalla "legge della ragione", sono esattamente le libertà fondamentali di ogni uomo: la propria vita, la libertà e la proprietà(quest'ultima intesa in senso lato).
	
	\subsubsection*{Stato e libertà}
	
	La presenza di uno Stato, quindi, si presenta solo nella situazione in cui i cittadini vogliano che i loro diritti siano tutelati da un'entità in grado di far rispettare la legge con la forza.
	
	\subsection*{Rousseau}
	
\end{document}