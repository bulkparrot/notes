\documentclass[10pt,a4paper]{article}
\usepackage[utf8]{inputenc}
\usepackage[T1]{fontenc}
\usepackage{amsmath}
\usepackage{amsfonts}
\usepackage{amssymb}
\usepackage{makeidx}
\usepackage{graphicx}
\usepackage[left=1.00in, right=1.00in, top=1.00in, bottom=1.00in]{geometry}
\author{Tommaso Severini}
\title{Filosofia - I sofisti}
\begin{document}
	\maketitle
	
	\section{Introduzione}
	
	I sofisti, formati nel V secolo a.C., prendono il loro nome dalla parola greca per sapienza. Essi, infatti, furono i primi a fare della sapienza un mestiere, insegnandola a pagamento. Per questo motivo, sono stati spesso giudicati no come filosofi, ma più primitivi insegnati. Nonostante ciò, bisogna enfatizzare il fatto che i sofisti portarono a una vera rivoluzione in ambito filosofico, spostando l'interessa della ricerca filosofica \textbf{dalla natura sull'uomo}.
	
	\section{La politica}
	
	I sofisti, come già detto, concentrano le loro riflessioni sull'uomo e, in particolare, sulla società in cui vive e come si debba vivere in essa, ragionando sulla politica, la lingua, l'educazione... Ciò accadde soprattutto perchè ormai la riflessione naturalistica non era più una via percorribile, poichè tutte le possibili idee erano state esplorato nei minimi dettagli. Dal punto di vista \textbf{sociologico}, l'aristocrazia era in crisi e la potenza dei cittadini si accresceva, i commerci si stavano ampliando e i lavoratori si stavano specializzando nei loro mestieri. Tutti questi fattori portarono alla nascita del principio fondante della filosofia sofistica, la \textbf{democrazia}. Secondo i sofisti, l'uomo che \textbf{partecipa attivamente} alla vita politica della città, sa \textbf{far prevalere le proprie opinioni} per il benessere della società e sa padroneggiare \textbf{l'arte dell'eloquenza}, vive attivamente in democrazia. 
	
	\section{Caratteristiche principali}
	
	Da molti, la sofistica è considerata una forma di "illuminismo greco", in quanto la ragione prevale in ogni ambito della vita, permettendo il raggiungimento di una società migliore. Ciò permette di osservare la realtà senza pregiudizi e lontano dalle tradizioni. Ciò provoca la \textbf{predita del passato in favore della ragione}.\\
	
	 Altra caratteristica fondamentale dei sofisti è l'importanza che la \textbf{formazione di un individuo} assume. Essi, infatti, credono che la sapienza non debba essere solamente un insieme di conoscenze specialistiche che permettono di lavorare, ma un formazione generale dell'individuo che permette di vivere attivamente nella società.\\
	 
	 Ultima caratteristica fondante dei sofisti è la loro attitudine nei confronti di \textbf{nuove culture}. Infatti, dovendo viaggiare molto per lavoro, questi filosofi sono usciti dalla chiusa e limitata realtà della polis e si fecero portatori di nuove culture e conoscenze provenienti da altri popoli.
	
	\section{Protagora}
	
	Protagora è considerato da molti il più importante esponente tra tutti i sofisti per la sua incredibile eloquenza e argute intuizioni. Egli, nei suoi scritti, tratta principalmente dell'uomo, dell'essere (riprendendo il pensiero di Eraclito) e della politica. 
	
	\subsection{L'uomo}
	
	Il pensiero di Protagora si basa sulla seguente affermazione: "l'uomo è misura di tutto, delle cose che sono in quanto sono e delle cose che non sono in quanto non sono". Quindi, per Protagora, l'uomo è il \textbf{soggetto di giudizio} in base al quale il mondo è interpretato. Nonostante ciò, l'affermazione di Protagora può essere interpretata diversamente secondo le definizioni che si danno alle parole "uomo" e "cose".\\
	
	\subsubsection{Prima interpretazione}
	
	L'interpretazione che Platone ci fornisce è sicuramente quella più semplice, in cui ogni elemento assume il suo significato letterale. Ciò ci induce a pensare che \textbf{ogni individuo percepisce il mondo in modo differente}.
	
	\subsubsection{Seconda interpretazione}
	
	Questa interpretazione identifica l'"uomo" come l'umanità e le "cose" come la realtà. Secondo questa teoria, la realtà è interpretata \textbf{attraverso principi comuni a tutte le civiltà}.
	
	\subsubsection{Terza interpretazione}
	
	In questo ultimo punto di vista, l'"uomo" indica le civiltà umane, mentre le "cose" si riferiscono ai valori su cui quella società si basa. Questa interpretazione ci fa capire che \textbf{ogni civiltà giudica il mondo secondo i suoi principi fondanti}.
	
	Tutte queste riflessioni, indifferentemente da quale si ritenga più corretta, fanno capire che i sofisti si basavano sul principio dell'\textbf{umanesimo}, secondo cui non è il mondo esterno ad influenzare la visione dle mondo, ma l'uomo. Da ciò ne deriva anche che i sofisti credono nel fenomenismo, ovvero che le loro esperienze non riflettono oggettivamente la realtà, ma siano semplicemente interpretazioni di essa.
	
	\section{Il relativismo}
		   
	
\end{document}