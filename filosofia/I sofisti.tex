\documentclass[10pt,a4paper]{article}
\usepackage[utf8]{inputenc}
\usepackage[T1]{fontenc}
\usepackage{amsmath}
\usepackage{amsfonts}
\usepackage{amssymb}
\usepackage{makeidx}
\usepackage{graphicx}
\usepackage[left=1.00in, right=1.00in, top=1.00in, bottom=1.00in]{geometry}
\author{Tommaso Severini}
\title{Filosofia - I sofisti}
\begin{document}
	\maketitle
	
	\section{Introduzione}
	
	I sofisti, formati nel V secolo a.C., prendono il loro nome dalla parola greca per sapienza. Essi, infatti, furono i primi a fare della sapienza un mestiere, insegnandola a pagamento. Per questo motivo, sono stati spesso giudicati no come filosofi, ma più primitivi insegnati. Nonostante ciò, bisogna enfatizzare il fatto che i sofisti portarono a una vera rivoluzione in ambito filosofico, spostando l'interessa della ricerca filosofica \textbf{dalla natura sull'uomo}.
	
	\section{La politica}
	
	I sofisti, come già detto, concentrano le loro riflessioni sull'uomo e, in particolare, sulla società in cui vive e come si debba vivere in essa, ragionando sulla politica, la lingua, l'educazione... Ciò accadde soprattutto perchè ormai la riflessione naturalistica non era più una via percorribile, poichè tutte le possibili idee erano state esplorato nei minimi dettagli. Dal punto di vista \textbf{sociologico}, l'aristocrazia era in crisi e la potenza dei cittadini si accresceva, i commerci si stavano ampliando e i lavoratori si stavano specializzando nei loro mestieri. Tutti questi fattori portarono alla nascita del principio fondante della filosofia sofistica, la \textbf{democrazia}. Secondo i sofisti, l'uomo che \textbf{partecipa attivamente} alla vita politica della città, sa \textbf{far prevalere le proprie opinioni} per il benessere della società e sa padroneggiare \textbf{l'arte dell'eloquenza}, vive attivamente in democrazia. 
	
	\section{Caratteristiche principali}
	
	Da molti, la sofistica è considerata una forma di "illuminismo greco", in quanto la ragione prevale in ogni ambito della vita, permettendo il raggiungimento di una società migliore. Ciò permette di osservare la realtà senza pregiudizi e lontano dalle tradizioni. Ciò provoca la \textbf{predita del passato in favore della ragione}.\\
	
	 Altra caratteristica fondamentale dei sofisti è l'importanza che la \textbf{formazione di un individuo} assume. Essi, infatti, credono che la sapienza non debba essere solamente un insieme di conoscenze specialistiche che permettono di lavorare, ma un formazione generale dell'individuo che permette di vivere attivamente nella società.\\
	 
	 Ultima caratteristica fondante dei sofisti è la loro attitudine nei confronti di \textbf{nuove culture}. Infatti, dovendo viaggiare molto per lavoro, questi filosofi sono usciti dalla chiusa e limitata realtà della polis e si fecero portatori di nuove culture e conoscenze provenienti da altri popoli.
	
	\section{Protagora}
	
	Protagora è considerato da molti il più importante esponente tra tutti i sofisti per la sua incredibile eloquenza e argute intuizioni. Egli, nei suoi scritti, tratta principalmente dell'uomo, dell'essere (riprendendo il pensiero di Eraclito) e della politica. 
	
	\subsection{L'uomo}
	
	Il pensiero di Protagora si basa sulla seguente affermazione: "l'uomo è misura di tutto, delle cose che sono in quanto sono e delle cose che non sono in quanto non sono". Quindi, per Protagora, l'uomo è il \textbf{soggetto di giudizio} in base al quale il mondo è interpretato. Nonostante ciò, l'affermazione di Protagora può essere interpretata diversamente secondo le definizioni che si danno alle parole "uomo" e "cose".\\
	
	\subsubsection{Prima interpretazione}
	
	L'interpretazione che Platone ci fornisce è sicuramente quella più semplice, in cui ogni elemento assume il suo significato letterale. Ciò ci induce a pensare che \textbf{ogni individuo percepisce il mondo in modo differente}.
	
	\subsubsection{Seconda interpretazione}
	
	Questa interpretazione identifica l'"uomo" come l'umanità e le "cose" come la realtà. Secondo questa teoria, la realtà è interpretata \textbf{attraverso principi comuni a tutte le civiltà}.
	
	\subsubsection{Terza interpretazione}
	
	Secondo questo ultimo punto di vista, l'"uomo" indica le civiltà umane, mentre le "cose" si riferiscono ai valori su cui quella società si basa. Questa interpretazione ci fa capire che \textbf{ogni civiltà giudica il mondo secondo i suoi principi fondanti}.
	
	Tutte queste riflessioni, indifferentemente da quale si ritenga più corretta, fanno capire che i sofisti si basavano sul principio dell'\textbf{umanesimo}, secondo cui non è il mondo esterno ad influenzare la visione del mondo, ma l'uomo. Da ciò ne deriva anche che i sofisti credano nel fenomenismo, ovvero che le loro esperienze non riflettano oggettivamente la realtà, ma siano semplicemente interpretazioni di essa.
	
	\section{Il relativismo}
	
	Altro aspetto centrale della filosofia sofistica è il relativismo, evidenziamo nello scritto anonimo "I ragionamenti doppi". Secondo questo testo, non esiste un'unica ed universale verità, ma che tutto sia relativo al punto di vista della persona che ragiona, dai suoi ideali... 
	
	Come esempi, il testo fornisce diversi punti di vista su vari argomenti, facendo vedere come spesso questi vadano al vantaggio di qualcuno e a svantaggio di qualcun'altro, come la morte, terribile per il defunto e per i suoi cari, ma essenziale affinchè le pompe funebri e i becchini lavorino.
	
	\subsection{Relativismo culturale}
	
	Oltre all'impossibilità di poter definire un concetto oggettivo di bene, male, giustizia, ingiustizia etc, i sofisti pongono la loro attenzione anche sulle differenze culturali tra i vari popoli e come differenti valori fondanti possano risultare in atteggiamenti e mentalità molto diverse.  Ad esempio, i comportamenti nei confronti delle donne e le loro funzioni nella società.
	
	\section{L'utile}
	
	Nonostante Protagora creda che non esista una verità assoluta e e universale, egli crede che ogni decisione degli uomini debbano essere influenzati da un principio di scelta, ovvero \textbf{l'utilità pubblica e privata}, inteso come il bene dell'individuo e della comunità. Nonostante ciò possa sembrare contraddittorio poichè per individuare l'utile servirebbe un principio di scelta preesistente, ciò che Protagora voleva trasmettere è \textbf{l'interesse per la sopravvivenza della specie e per la vita pubblica}.
	
	\section{Gorgia}
	
	Gorgia fu un sofista vissuto nel V secolo a.C. caratterizzato da una filosofia molto pessimistica e tra le sue opere ricordiamo il "Sul non essere" e "Encomio di Elena".
	
	\subsection{L'impensabilità e l'inesprimibilità}
	
	Nella sua opera più importante, "Sul non essere", troviamo le 3 tesi alla base del pensiero di Gorgia:
	\begin{itemize}
		\item Nulla esiste
		\item Se anche qualcosa esiste non è conoscibile dall'uomo, ovvero pensabile
		\item Se anche fosse conoscibile, non sarebbe comunicabile
	\end{itemize}
	Nonostante tutto ciò venisse considerato solo una geniale espressione di eloquenza, il messaggio di Gorgia è molto più profondo. Egli vuole esprimere come sia impossibile parlare in modo logico dell'essere. Con la sua prima affermazione egli asserisce che l'essere è inesistente. Con la seconda, che è impossibile poter pensare all'essere perchè i nostri pensieri dovrebbero essere "fotografie" della realtà. Ma ciò non è vero, basti pensare all'inesistente per realizzare che la ragione non rappresenta la realtà. Con la terza affermazione, Gorgia vuole affermare che il linguaggio spesso si discosta dalla realtà, quindi non è in grado di rappresentarla. 
	
	\subsection{Scetticismo e agnosticismo}
	
	Le tesi fondanti della filosofia di Gorgia possono essere interpretati in modo più profondo se all'essere viene sostituita una divinità. Con la prima tesi, Gorgia esprime un ideali ateistico, che afferma l'inesistenza di alcun divinità. Mentre, con la seconda e la terza, Gorgia si appella agli ideali dello scetticismo e dell'agnosticismo, affermando che non è possibile raggiungere una verità certa e che gli uomini non possegono abbastanza strumenti per poter rispondere a questo tipo di quesiti. Molti altri filosofi sofisti guardano alla religione con un occhio di riguardo, come \textbf{Prodico di Ceo}, che afferma che \textbf{le divinità non siano altro che elementi necessari alla vita divinizzati dagli uomini}. Oppure, \textbf{Crizia} che ritiene che la religione sia semplicemente \textbf{utilizzato dai potenti per ottenere il potere}, riprendendo filosofie come il marxismo e l'illuminismo.
	
	\subsection{L'irrazionalità della vita}
	
	Come è possibile intuire dai ragionamenti di Gorgia, è facile comprendere come egli creda che la vita sia irrazionale e misteriosa. Infatti, egli sostiene che tutte le azioni degli uomini siano dettate da fattori esterni e incontrollabili, lasciando l'individuo in balia degli eventi che lo circondano. Ciò serve per mostrare la consapevolezza di Gorgia della \textbf{fragilità e nullità della vita umana}.
	
	\subsection{Le leggi}
	
	I sofisti furono i primi a discostarsi dalla tradizione che vedeva le leggi come volute dagli dei, cominciando a considerarle elementi del \textbf{carattere umano e sociale}. Il primo a pronunciarsi significativamente sulla questione fu Protagora, che fece notare che l'individuo diviene uomo facendo parte di una società; ma poichè le società hanno bisogno di regole che le governino, egli fu il primo a sostenere \textbf{l'importanza del dover rispettare le leggi per essere un uomo}. Altri filosofi, invece, come \textbf{Antifone}, credono che solo le leggi naturali sia vere e che quelle umane siano opinabili. Queste leggi naturali sono quelle che puntano verso la concordia e l'utile per la società. 
	
	Nonostante ciò, molti altri sofisti sostennero l'inutilità e addirittura la malvagità delle leggi, che, secondo loro, sono nate solamente per tutelare i potenti.
	
	こんにちは
	
	\subsection{La retorica}
	
	Per i sofisti, la parole, oltre ad essere strumento per partecipare alla vita sociale, era uno strumento di persuasione utilizzato principalmente in due modi:
	\begin{itemize}
		\item Il discorso lungo doveva essere articolato e strutturato nei minimi dettagli, per persuadere le altre persone ed evitare obiezioni.
		\item Il discorso breve doveva essere aggressivo e pungente, mirato a distruggere i ragionamenti degli avversari.
	\end{itemize}

	Gli strumenti principalmente usati erano:
	\begin{itemize}
		\item La dialettica, ovvero l'arte del discutere per far prevalere le proprie opinioni.
		\item La eristica, ovvero l'arte dell'attacco, mirato a far prevalere le proprie opinioni, giuste o sbagliate che fossero.
		\item L'antilogica, ovvero la capacità di poter usare più tesi affinche si giunga ad una contraddizione.
	\end{itemize}
	
\end{document}