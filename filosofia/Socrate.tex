\documentclass[10pt,a4paper]{article}
\usepackage[utf8]{inputenc}
\usepackage[T1]{fontenc}
\usepackage{amsmath}
\usepackage{amsfonts}
\usepackage{amssymb}
\usepackage{makeidx}
\usepackage{graphicx}
\usepackage[left=1.00in, right=1.00in, top=1.00in, bottom=1.00in]{geometry}
\author{Tommaso Severini}
\title{Filosofia - Socrate}
\begin{document}
	\maketitle
	
	Socrate, vissuto nel V secolo a.C., fu uno dei filosofi più significativi della filosofia occidentale, rendendo la filosofia la sua missione di vita, \textbf{esaminando costantemente se stesso e gli altri}.
	
	Egli, nonostante sia riconosciuto come l'antisofista per eccellenza, presenta diversi punti in comune con la filosofia sofistica:
	\begin{itemize}
		\item L'attenzione per l'uomo
		\item La tendenza a cercare nell'uomo le cause delle decisioni e delle azioni
		\item Il suo atteggiamento anticonformista e antitradizionalista
		\item L'importanza della dialettica e, soprattutto, del paradosso e delle contraddizioni.
	\end{itemize}

	Nonostante ciò, ci sono molti sostanziali elementi che differenziano i sofisti e Socrate:
	\begin{itemize}
		\item La voglia di cercare la verità e di non partecipare ad una discussione fine a se stessa.
		\item La voglia di volersi distaccare dal relativismo e cercare di trovare risposte che accomunino tutti gli uomini.
	\end{itemize}

	Dopo essersi appassionato agli ultimi lavori dei naturalisti, egli si concentra sulla ricerca di se stesso, adottando una filosofia che fa proprio il motto dell'oracolo di Delfi \textbf{"Conosci te stesso"}. Seguendo questa filosofia, Socrate comincia a capire che il rapporto con gli altri è alla base della ricerca di se e della verità e comincia a passare le sue giornate avendo discussioni molto profonde con i passanti, acquisendo nuove conoscenze, ma anche una nuova reputazione.
	
	\section{Il pensiero socratico}
	
	\subsection{Sapiente è chi sa di non sapere}
	
	Il primo passo che Socrate compie verso la ricerca della verità è la comprensione di non sapere. Quando egli riceve il responso dell'oracolo di Delfi che lo definisce come l'uomo più saggio della Grecia, egli crede che questo ragionamento sia alla base di tutto.
	
	Nonostante, da un canto, possa sembrare che questa affermazione riveli solamente la natura scettica e agnostica della filosofia socratica, fornendo i limiti della ricerca umana, essa funge da invito ad imparare. Socrate, infatti, capisce che colui che si crede sapiente non ha il bisogno di ricercare la verità, mentre chi ha coscienza della propria ignoranza comincierà a porsi domande e a ricercare la verità.
	
	\subsection{L'ironia}
	
	Uno degli strumenti maggiormente usati da Socrate è la sua famosa ironia. Tecnica che consisteva nel sembrare ignorante riguardo ad un argomento e porgere domande all'interlocutore per ottenere risposte incoerenti e deboli. Dopo di che, Socrate cominciava a porre una serie di quesiti studiati per confutare le affermazioni dell'interlocutore e provocare irritazione e imbarazzo. Nonostante ciò, il vero intento del dialogo era quello di far comprendere all'interlocutore la sua ignoranza e a invogliarlo a trovare la verità.
	
	\subsection{La maieutica}
	
	L'arte di far "partorire"(termine ripreso dal mestiere della madre di Socrate, ostetrica) nuove idee e invogliare il proprio interlocutore a trovare la verità utilizzando l'ironia divenne in seguito nota come \textbf{maieutica}. Ciò fa diventare la verità una scoperta personale e fa diventare la filosofia un viaggio della mente, fondando le basi della pedagogia.
	
	\subsection{La definizione}
	
	Altro strumento largamente utilizzato da Socrate è la domanda \textbf{"ti esti"} (che cos'è), ovvero la richiesta di una definizione precisa e formale, non accettando le prime risposte superficiali dell'interlocutore. Il significato più profondo che Socrate vuole trasmettere con questa domanda è l'idea di \textbf{"essenza"}, ovvero il fattore che accomuna anche i concetti più diversi ma che ricadono tutti sotto una unica classificazione, come ad esempio le virtù. Quindi, questa domanda presenta il lato negativo del mettere in crisi il proprio interlocutore, ma il lato positivo di poter trovare una definizione di un determinato argomento, creando opportunità per discussioni più approfondite.
	
\end{document}