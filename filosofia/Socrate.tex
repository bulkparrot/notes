\documentclass[10pt,a4paper]{article}
\usepackage[utf8]{inputenc}
\usepackage[T1]{fontenc}
\usepackage{amsmath}
\usepackage{amsfonts}
\usepackage{amssymb}
\usepackage{makeidx}
\usepackage{graphicx}
\usepackage[left=1.00in, right=1.00in, top=1.00in, bottom=1.00in]{geometry}
\author{Tommaso Severini}
\title{Filosofia - Socrate}
\begin{document}
	\maketitle
	
	Socrate, vissuto nel V secolo a.C., fu uno dei filosofi più significativi della filosofia occidentale, rendendo la filosofia la sua missione di vita, \textbf{esaminando costantemente se stesso e gli altri}.
	
	Egli, nonostante sia riconosciuto come l'antisofista per eccellenza, presenta diversi punti in comune con la filosofia sofistica:
	\begin{itemize}
		\item L'attenzione per l'uomo
		\item La tendenza a cercare nell'uomo le cause delle decisioni e delle azioni
		\item Il suo atteggiamento anticonformista e antitradizionalista
		\item L'importanza della dialettica e, soprattutto, del paradosso e delle contraddizioni.
	\end{itemize}

	Nonostante ciò, ci sono molti sostanziali elementi che differenziano i sofisti e Socrate:
	\begin{itemize}
		\item La voglia di cercare la verità e di non partecipare ad una discussione fine a se stessa.
		\item La voglia di volersi distaccare dal relativismo e cercare di trovare risposte che accomunino tutti gli uomini.
	\end{itemize}

	Dopo essersi appassionato agli ultimi lavori dei naturalisti, egli si concentra sulla ricerca di se stesso, adottando una filosofia che fa proprio il motto dell'oracolo di Delfi \textbf{"Conosci te stesso"}. Seguendo questa filosofia, Socrate comincia a capire che il rapporto con gli altri è alla base della ricerca di se e della verità e comincia a passare le sue giornate avendo discussioni molto profonde con i passanti, acquisendo nuove conoscenze, ma anche una nuova reputazione.
	
\end{document}