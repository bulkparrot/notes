\documentclass[10pt,a4paper]{article}
\usepackage[utf8]{inputenc}
\usepackage[T1]{fontenc}
\usepackage{amsmath}
\usepackage{amsfonts}
\usepackage{amssymb}
\usepackage{makeidx}
\usepackage{graphicx}
\usepackage[left=1.00in, right=1.00in, top=1.00in, bottom=1.00in]{geometry}
\author{Tommaso Severini}
\title{Filosofia - Umanesimo e rinascimento}
\begin{document}
	\maketitle
	
	\section*{Marsilio Ficino}
	
	Marsilio Ficino fu un filosofo neoplatonista vissuto principalmente a Firenze
	
	Egli è famoso per aver fondato l'Accademia platonica fiorentina in cui è possibile enfatizzare l'influenza della religione nel pensiero del Ficino. Essi infatti fecero risalire l'origine del pensiero platonico a Mosè.
	
	L'unità di religione e filosofia porta ad una concezione completamente nuova del mondo: centro di tutto.\\
	
	In generale, il fulcro della filosofia di Ficino può essere riassunto attraverso le varie visioni del mondo che l'uomo può sperimentare: corpo, qualità, anima, angelo e Dio. L'anima rappresenta ciò che permette la transizione tra Uomo e Dio e viceversa, diventando la terza essenza. Quest'ultima trova il suo riscontro più concreto nella sua azione mediatrice dell'\textbf{Amore}.
	
	\section*{Giordano Bruno}
	
	
	
\end{document}