\documentclass[10pt,a4paper]{article}
\usepackage[utf8]{inputenc}
\usepackage[T1]{fontenc}
\usepackage{amsmath}
\usepackage{amsfonts}
\usepackage{amssymb}
\usepackage{makeidx}
\usepackage{graphicx}
\usepackage[left=1.00in, right=1.00in, top=1.00in, bottom=1.00in]{geometry}
\author{Tommaso Severini}
\title{Filosofia - Platone}

\begin{document}
	\maketitle

  Platone, filosofo ateniese nato nel 427 a.C. da una famiglai aristocratica, vive in un contesto politico molto travagliato. Atene, infatti, è sconvolta dalla sconfitta della guerra del peloponneso, gli aristocratici stanno perdendo il loro potere e tutto ciò è segnato decisivamente dalla morte del maestro di Platone, Socrate. Platone crede che la situazione sia talmente negativa da dover radicalizzare questo periodo di crisi e farlo diventare una metafora della \textbf{crisi dell'uomo nella sua totalità}. Egli estremizza anche la figura di Socrate, tanto da renderlo al contempo sia figura della crisi sia fonte di speranza.

  \section{Le caratteristiche del pensiero platonico}

   Di sicuro la fedeltà alla figura del suo maestro è uno degli aspetti che ha influenzato maggiormente il pensiero di Platone, anche se le basi di esso si distacchino molto dalle teorie socratiche. In definitiva, però, la \textbf{ricerca platonica è lo sforzo che si compie per interpretare la personalità di Socrate}. Altro elemento ripreso da Socrate è sicuramente il mezzo di comunicazione. Infatti, anche se Platone ha deciso di scrivere i suoi pensieri in diverse raccolte, egli crede che il filosofeggiare come dialogo sia fondamentale nella vita dell'uomo. Ciò, nonostante sia in contrasto con la voglia di Platone di trovare una verità assoluta, fa in modo che la più importante delle idee della filosofia platonica si sviluppasse: la ricerca è inesauribile e mai conclusa, uno \textbf{sforzo infinito che conduce a una verità che gli uomini non potranno mai comprendere pienamente}.

   \section{Il mito}

   Oltre al dialogo, l'altra principale forma di comunicazione che Platone adotta è quella del mito. Fa ciò per due ragioni:
   \begin{itemize}
     \item Ciò rende la comunicazione di concetti filosofici molto più semplice, specialmente a chi non si intende di filosofia, riempiendo le lacune linguistiche della comunicazione umana.
     \item Ciò permette di ragionare su concetti che arrivano al limite (e possibilmente anche oltre) la conoscenza umana. In particolare ciò si rivela utile quando si ha a che fare con il trascendentale e per colmare le lacune della ricerca filosofica.
   \end{itemize}

   \section{La difesa si Socrate}

   Due delle opere più importanti scritte da Socrate sono \textbf{L'apologia di Socrate} e il \textbf{Critone}.
   La prima esalta il ruolo di Socrate e, di conseguenza, la ricerca filosofica a cui la vita si deve dedicare (\textbf{Una vita senza ricerca non è degna di essere vissuta dall'uomo}).

   Nella seconda, al contrario, Platone si concentra sulla decisione finale di Socrate: morire rispettando i propri ideali oppure scappare, però smentendo la sostanza del proprio inseganmento.

   Queste due opere presentano 3 punti cardine in comune:
   \begin{itemize}
     \item La virtù è una sola e si identifica con la scienza
     \item La virtù è insegnabile, ma solo come scienza
     \item La felicità dell'uomo consiste nella virtù come scienza
   \end{itemize}

   \section{Il sofismo}

   Platone, nel suo \textit{Protagora}, pone una forte critica alla pretsa dei sofisti di essere degli inseganti. Infatti, dal punto di vista di Platone, la conoscenza dei sofisti non può essere considerata scienza, ma solo una serie di abilità acquisite. Protagora, infatti, non crede che la virtù possa essere insegnata coem scienza perchè, secondo lui, la scienza è solo una delle virtù. L'unico inseganmento che mostra per intero il suo valore è quello di Socrate. Altre critiche che Platone rivolge ai sofisti si riferiscono all'eristica e alla retorica.

	 \section{La dottrina delle idee}

	 Platone, cercando di approfondire la scienza di Socrate, perfeziona quest'ultima e la rende idea. Da ciò, però, Platone si chiede quale sia la realtà rappresentata dalla scienza. Nonostante una risposta esatta esista, secondo Platone, egli esclude immediatamente tutto ciò che appartiene ai sensi poichè esso è mutabile e imperfetto. Ciò che appartiene alla sfera dei sensi prende il nome di \textbf{doxa}.

	 La risposta che Platone da alla domanda "quale la realtà rappresentata dalla scienza?" è l'idea. Essa è una sostanza perfetta e immutabile che rappresenta una caratteristica. Esse però non appartengono all'uomo e risiedono in un mondo paragonabile all'empireo dantesco detto \textbf{iperuranio}.\\

	 L'opinione, mutevole e imperfetta, rispecchia le cose, mutevoli e imperfette.

	 La scienza, immutabile e perfetta, rispecchia le idee, immutabili e perfette.\\

	 Da ciò è facile intuire come il mondo, abitato dalle cose, sia in continuo cambiamento, riprendendo l'idea del \textbf{panta rei} da Eraclito. L'ultimo tratto fondamentale della teorie delle idee è il dualismo: sia gnoseologico tra razionalità ed esperienza e ontologico, tra cose ed essere.

	 \subsection{Le idee}

	 Le idee si distinguono in due categorie:
	 \begin{itemize}
	 	\item idee valori: concetti astratti che appartengono, in un modo o in un altro, a ciò che oggi chiameremmo un insieme di ideali. Ad esempio, "Bellezza", "Bene"  e "Giustizia".
		\item idee matematiche: corrispondono ad entità matematiche e geometriche che modellano il mondo in cui viviamo.
	 \end{itemize}

	 Queste idee rappresentano sia \textbf{criteri di giudizio}, che ci permettono di analizzare le cose, sia \textbf{causa delle cose}, in quanto queste usano le idee come modelli, assumendo la funzione di \textbf{paradigma}.

	 Nonostante tutto ciò, rimaneva un problema fondameentale. Come può l'uomo, che è in contatto solo con le cose, riuscire a comprendere le idee, immutabili e perfette? Platone giustifica ciò attraverso l'\textbf{anamnesi}, ovvero la reminescenza. Egli sosteneva che l'anima, prima di prendere possesso di un corpo, coesistesse nel mondo delle idee e le esperienze della vita terrena permettono di ricordare ciò che l'anima aveva incontrato. Ciò non significa che l'esperienza crei conoscenza, ma che essa la solleciti solamente.

	 \subsection{Immortalità dell'anima}

	 La reminiscenza implica l'Immortalità dell'anima, argomento tratttato in una delle opere più famose di Platone: il \textbf{Fedone}.

	 In quest'opera, Platone giustifica l'immortalità dell'anima in due modi: per prima cosa, Platone afferma che, come la morte nasca dalla vita, anche la vita nasca dalla morte. Seconda argomentazione che Platone utilizza è detta della "\textbf{somiglianza}": l'anima, essendo affine alle idee, che sono eterne, sarà anch'essa tale. La terza e ultima argomentazione, detta della "\textbf{vitalità}", afferma che l'anima sia vita e, di conseguenza, che accolga in se l'idea della vita. Da ciò si ottiene che l'anima non può accogliere in se l'idea della morte.

	 \section{Contro il relativismo sofistico}

	 Una delle migliori definizioni che possiamo dare della filosofia platonica è l'antitesi del sofismo. Egli, infatti, crede che il relativismo attuato da questi filosofi, poichè si fonda sull'accettazioni didiversi punti di vista discordanti, non possa portare l'uomo ad ottenere risultati tangibili. Perciò, Platone si vede costretto a restaurare una sorta di \textbf{"assolutismo intellettuale e conoscitivo"} che permetta all'uomo di basarsi su certezze concrete e inamovibili per eliminare le piaghe sociali e politiche dalla società. Da ciò si ottiene una delle massime che riassumono l'avversione di Platone al relativismo sofistico: \textbf{le idee come misura dell'uomo}.

	 \section{La dottrina dell'amore}

	 \subsection{Il simposio}

	 Questa opera ha come protagonista Eros, il "dio" (più propriamente un demone) dell'amore, che cerca ciò che non possiede, la conoscneza. Questo desiderio lo rende filosofo e lo rende simbolo dell'umanità. Infatti, per Platone, l'amore rappresenta \textbf{il desiderio di sapienza e bellezza}, che quindi funge da strumento ad una conoscenza superiore.

	 \subsection{Il Fedro}

	 Questa opera si concentra sui mezzi attraverso i quali l'anima umana può raggiungere la bellezza suprema. Per fare ciò, Platone descrive le varie parti dell'anima e ogni loro ruolo:
	 \begin{itemize}
	 	\item La parte razionale: che guida gli altri due cavalli attraverso la logica
		\item La parte desiderante: il cavallo bianco che obbedisce alla ragione
		\item La parte irascibile: il cavallo nero ricalcitante che cerca di soddisfare gli istinti dell'uomo
	 \end{itemize}

	 \section{Lo stato}

	 Il tema fondamentale della opus magna di Platone, la \textbf{Repubblica}, è la politica. In particolar modo, si concentra su come organizzare una società perfetta.
	 Platone crede che la comunità politica debba essere governata dai filosofi. Ciò, però, genera due domande fondamentali: Su cosa si dovrebbe basare la società? Chi sono i filosofi?\\

	 Alla prima domanda Platone risponde con la giustizia. Platone considera la giustizia la condizione fondamentale affinchè uno stato nasca e possa continuare a vivere. Inoltre, Platone sostiene che lo stato debba essere diviso in tre classi sociali: i governanti, caratterizzati dalla \textbf{saggezza}, i guerrieri,caratterizzati dal \textbf{coraggio}, e i lavoratori, caratterizzati dalla \textbf{temperanza}.

	 Per cercare di mantenere questo equilibrio, Platone propone di eliminare tutti gli elementi di diseguaaglianza sociale, come la proprietà privata e il lusso, creando una sorta di forma di governo comunista.

	 Ovviamente il filosofo rinconosce l'inattuabilità del suo pensiero, ma crede comunque che esso possa essere utilizzato come modello di riferimento per giudicare le alterazioni dello stato. Tra di esse troviamo:
	 \begin{itemize}
	 	\item La timocrazia, che si basa sull'onore. Ad esso corrisponde l'uomo ambizioso, amante del comando e diffidente verso i sapienti.
		\item L'oligarchia, basato sul censo. A ciò corrisponde l'uomo avaro, laborioso e parsimonioso.
		\item La democrazia, basata sulla libertà. Ad esso si associa l'uomo che tende ad abbandonarsi ai suoi istinti.
		\item La tirannide, basata si basa sull'eccessiva libertà. A ciò si ricollega l'uomo schiavo delle passioni e che è più infelice degli uomini.
	 \end{itemize}

Da ciò è facile intuire come Platone sia contrario alla "troppo libera" democrazia e sia, al contario, sostenitore dello statalismo, che prevede l'intervento dello stato nella cosa pubblica. Ciò perchè l'arte di governare è riservata solamente  alla classe di sapienti. Da ciò si intuisce come Platone giustifichi la sua forte divisione della società in classi sociali 

\end{document}
