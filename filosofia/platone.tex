\documentclass[10pt,a4paper]{article}
\usepackage[utf8]{inputenc}
\usepackage[T1]{fontenc}
\usepackage{amsmath}
\usepackage{amsfonts}
\usepackage{amssymb}
\usepackage{makeidx}
\usepackage{graphicx}
\usepackage[left=1.00in, right=1.00in, top=1.00in, bottom=1.00in]{geometry}
\author{Tommaso Severini}
\title{Filosofia - Platone}

\begin{document}
	\maketitle

  Platone, filosofo ateniese nato nel 427 a.C. da una famiglai aristocratica, vive in un contesto politico molto travagliato. Atene, infatti, è sconvolta dalla sconfitta della guerra del peloponneso, gli aristocratici stanno perdendo il loro potere e tutto ciò è segnato decisivamente dalla morte del maestro di Platone, Socrate. Platone crede che la situazione sia talmente negativa da dover radicalizzare questo periodo di crisi e farlo diventare una metafora della \textbf{crisi dell'uomo nella sua totalità}. Egli estremizza anche la figura di Socrate, tanto da renderlo al contempo sia figura della crisi sia fonte di speranza.

  \section{Le caratteristiche del pensiero platonico}

   Di sicuro la fedeltà alla figura del suo maestro è uno degli aspetti che ha influenzato maggiormente il pensiero di Platone, anche se le basi di esso si distacchino molto dalle teorie socratiche. In definitiva, però, la \textbf{ricerca platonica è lo sforzo che si compie per interpretare la personalità di Socrate}. Altro elemento ripreso da Socrate è sicuramente il mezzo di comunicazione. Infatti, anche se Platone ha deciso di scrivere i suoi pensieri in diverse raccolte, egli crede che il filosofeggiare come dialogo sia fondamentale nella vita dell'uomo. Ciò, nonostante sia in contrasto con la voglia di Platone di trovare una verità assoluta, fa in modo che la più importante delle idee della filosofia platonica si sviluppasse: la ricerca è inesauribile e mai conclusa, uno \textbf{sforzo infinito che conduce a una verità che gli uomini non potranno mai comprendere pienamente}.

   \section{Il mito}

   Oltre al dialogo, l'altra principale forma di comunicazione che Platone adotta è quella del mito. Fa ciò per due ragioni:
   \begin{itemize}
     \item Ciò rende la comunicazione di concetti filosofici molto più semplice, specialmente a chi non si intende di filosofia, riempiendo le lacune linguistiche della comunicazione umana.
     \item Ciò permette di ragionare su concetti che arrivano al limite (e possibilmente anche oltre) la conoscenza umana. In particolare ciò si rivela utile quando si ha a che fare con il trascendentale e per colmare le lacune della ricerca filosofica.
   \end{itemize}

   \section{La difesa si Socrate}

   

\end{document}
